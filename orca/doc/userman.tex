%
% (c) copyright 1995 by the Vrije Universiteit, Amsterdam, The Netherlands.
% For full copyright and restrictions on use see the file COPYRIGHT in the
% top level of the Orca distribution.
%

\documentclass[10pt]{article}

\usepackage{html}

\setlength{\textheight}{9in}
\setlength{\textwidth}{6.75in}
\setlength{\oddsidemargin}{-0.19in}
\setlength{\evensidemargin}{-0.19in}
\setlength{\topmargin}{-0.25in}

\begin{document}
\title{User's Manual of the Programming Language
{\html {\htmladdnormallink {Orca} {http://www.cs.vu.nl/proj/orca/}}}
{\latex {Orca}}}

\maketitle

\section{Introduction}

This manual describes how to compile and run Orca programs.
\begin{htmlonly}
It is also available in
\htmladdnormallink{postscript}{http://www.cs.vu.nl/proj/orca/userman.ps.gz}.
\end{htmlonly}
The definition of Orca and pointers to the literature are given in
the {\html {\htmladdnormallink{Report on the Programming Language Orca}{http://www.cs.vu.nl/proj/orca/refman/refman.html}}}{\latex {\em Report on the Programming Language Orca}}.
The Orca system is stored in some directory,
for instance
{\em /usr/local/orca},
or
{\em /usr/proj/orca};
we will refer to this directory as
{\em \$ORCA}.
Example Orca programs can be found in the directory
{\em \$ORCA/examples}.
Orca compilation scripts can be found in {\em \$ORCA/bin}.

The Orca system consists of a compiler and several run-time systems (RTSs).
The compiler will be described in Section \ref{sec:compiler}.
The RTSs will be described in Section \ref{sec:rts}.
Section \ref{sec:using} will describe how to use the Orca system.
Section \ref{sec:example} discusses how to compile and run an example program.
Impatient users may immediately proceed to this section and try out the system.
The run-time error messages will be described in Section \ref{sec:errors}.
Section \ref{sec:standard}
will describe a number of useful standard modules, object types, and
generic types. Section \ref{sec:interface} will discuss the interface to the C programming
language, as far as needed to implement a module in C.

\section{The Orca compiler}\label{sec:compiler}

The Orca compiler consists of

\begin{itemize}
\item
a {\em driver}, called {\bf oc},
which will normally run all phases
of the compilation.
\item
a {\em front end} that translates Orca into (ANSI) C.
\item
a {\em C compiler} that translates C into assembly code for the
target machine.
\item
an {\em assembler} that translates assembly code into object code.
\item
a {\em loader} that links the object code together with the run-time system
to form an executable file, which can be run on multiple processors.
\end{itemize}

Only the first two items are Orca-specific; all other phases are part of the
C compiler and will not be described here.
Currently, the GNU C compiler is used, but any ANSI C compiler should do.

The compiler should be given files containing one or more
{\bf implementation}
or
{\bf specification}
units (modules or objects) as input.
Usually, specification units need not be compiled, but they must be if there
is no accompanying implementation unit (for instance when it is implemented
in C).
The compiler's 
{\em --I}
flag can be used to specify the directories in which
files with specification units and generic units are stored.
A number of useful standard object types and generic types can be found
in the directory
{\em \$ORCA/lib/std},
which
{\em oc}
searches automatically.

\subsection{Compiler optimizations done at present}

The compiler optimizes input parameters of procedures.
A data structure that is passed by value and that is not changed in the
called function is implemented as if it were passed by reference.
Hence, if a complex data structure needs to be passed by value to
a procedure, the formal parameter need not be declared as a reference
({\bf shared}) parameter, as in Pascal. In fact, if the parameter
is declared as 
{\bf shared},
the compiler assumes that the procedure will
change the actual parameter and will not do certain other optimizations
described below.

The compiler distinguishes between
{\em read}
and
{\em write}
operations on objects.
A read operation does not change the local data of the object it is
applied to; all other operations are write operations.
The run-time system implements read operations much
more efficiently than write operations.
On replicated objects, read operations are applied to local copies of
objects, without doing any interprocess communication.

To implement this optimization,
the compiler needs to have information about the 
{\em implementation}
of objects.
It maintains a file called 
.{\em module name}.db"
with information about object implementations and functions.
If an implementation uses an object type whose implementation part
has not yet been compiled, the compiler assumes all operations on the
object do not modify it and do not block.
If this assumption turns out not to be true, or
if the implementation part of an object type is changed, all other
modules and objects using it have to be recompiled.
The 
{\em oc}
program will take care of this automatically.

There are also some (more or less stable) optimizers which are enabled
through compiler switches:
\begin{itemize}
\item
The {\em {\em --}--O} flag enables a combined strength-reduction/common
sub-expression elimination/code motion optimizer.
\item
The {\em {\em --}--V} flag enables a compiler pass that attempts to reduce
the number of temporary variables by combining them.
\item
The {\em {\em --}--f} flag enables a compiler pass that generates two
versions of any function that has an object as shared parameter: one which
treats it as a local object, and the other that treats it as a shared object.
Also, at each call site the compiler determines which version to call.
\item
The {\em {\em --}--n} flag enables some very simple minded loop-unrolling
(this phase is only useful if you have loops which are executed a fixed
small number of times).
\end{itemize}

\subsection{Deviations from the language definition}

The compiler deviates in the following ways from the language definition.

\begin{itemize}
\item
In addition to the standard types ``real'' and ``int'',
the compiler also supports
the types ``longreal'' and ``longint'', and
``shortreal'' and ``shortint''.
\item
In addition to the letter 'E', the compiler also accepts the letter 'D' as the
start of a scale factor, for backwards compatibility with an earlier
Orca compiler, which used this to distinguish floating point constants of
type ``longreal'' from those of type ``real''.
\item
The compiler recognizes the built-in call
\begin{quote}
\verb+Strategy(+{\em object}\verb+, +{\em replicate}\verb+, +{\em cpu}\verb+)+
\end{quote}
which tells the RTS to replicate {\em object} if {\em replicate} is non-zero,
or else store the object on CPU {\em cpu}.
\end{itemize}

\section{The run-time system}\label{sec:rts}

The run-time system takes care of management of processes and shared objects.
Currently, the following runtime systems exist or are in development:
\begin{itemize}
\item
the {\bf unixproc} RTS. This RTS runs in a Unix process.
It is currently available for Sun Sparcstations running SunOS 4.1 or Solaris 2,
and for Intel x86 processors running BSD/OS or Linux.
the {\bf Panda}
RTS. This RTS runs on top of the Panda platform. Several implementations of
this platform are available, a.o. for Amoeba, Solaris2, Linux, and BSD/OS,
on different networks.
In fact, there are several versions of Panda. The currently used ones are
Panda3.0 and Panda4.0.
\end{itemize}

\subsection{The unixproc RTS}

The unixproc RTS runs an Orca program within one Unix process, so
the parallelism of the Orca program is simulated using threads.
Hence, this RTS can be used to debug Orca programs, but not to measure
speedups.
Orca processes are implemented as threads, so the RTS depends on the
availability of a suitable threads package.
The compiler and RTS make sure that every now and then a thread switch takes
place, so Orca processes are more or less time-sliced.

The number of processors that is to be simulated must be passed as an argument.

\subsection{The Panda RTS}

This RTS is built on top of the Panda platform. It can be
used on any system that is supported by Panda. The Panda platform offers
primitives for RPC, group communication, and threads.

A number of processes are started, and these processes
are numbered. Preferably, each of these processes runs on a different CPU.
Each Orca process is represented as a Panda thread.
The Orca statement
\begin{quote}
\begin{verbatim}
FORK name(parameters) ON (P);
\end{verbatim}
\end{quote}
creates a new thread within the Panda RTS process
{\em P}.
(If the ON part is omitted, the new thread is created in the current process.)

\section{Using Orca}\label{sec:using}

This section explains how to compile Orca programs on a Unix system and execute
them on the run-time systems.

\subsection{Compiling Orca programs}

The first step in running an Orca program is compiling the program.
The program
{\em oc}
invokes the different passes of the compiler.
There are scripts
in
{\em \$ORCA/bin}
which call 
{\em oc}
with parameters set for the different RTSs.
For instance,
{\em \$ORCA/bin}
contains
{\html {\htmladdnormallink {oc\_unixproc} {http://www.cs.vu.nl/proj/orca/oc_unixproc.1.html}}}
{\latex {\em oc\_unixproc}},
{\html {\htmladdnormallink {oc\_panda} {http://www.cs.vu.nl/proj/orca/oc_panda.1.html}}}
{\latex {\em oc\_panda}},
and others.
These should be given the implementation unit containing
{\em OrcaMain}
as parameter, for example:
\begin{quote}
\begin{verbatim}
oc_unixproc GordonBellAward.imp
\end{verbatim}
\end{quote}
The
{\em oc}
program will find any other modules automatically, and compile them
when needed, and will then, in the absence of errors,
produce an
{\em a.out}
file.

The
{\em oc}
program invokes the compiler with the
{\em --I\$ORCA/lib/std}
flag,
so modules and objects from the
{\em \$ORCA/lib/std}
directory can be directly
imported and used.
For more details about the
{\em oc}
program, the reader is referred to the
{\html {\htmladdnormallink {oc} {http://www.cs.vu.nl/proj/orca/oc.1.html}}}
{\latex {\em oc}}
manual page.

\subsection{Orca command-line arguments}

Usually, the RTS accepts some command-line options. 
Following these RTS options, one 
can pass command-line arguments to the Orca program using the "--OC"
option.
All options after this "--OC" option are ignored by the RTS and are
available to the Orca program through the {\em args} module, as described
in Section \ref{mod:args}.

\subsection{Running Orca programs using the unixproc RTS}

To run an Orca program using the
{\em unixproc}
RTS, 
compile the Orca program with the
{\html {\htmladdnormallink {oc\_unixproc} {http://www.cs.vu.nl/proj/orca/oc_unixproc.1.html}}}
{\latex {\em oc\_unixproc}}
command.

The generated
{\em a.out}
file can now be executed.
It should be given the number of simulated CPUs as first and only parameter.
For example:
\begin{quote}
\begin{verbatim}
a.out 10
\end{verbatim}
\end{quote}
will run the program, simulating 10 CPUs.
If no parameter is given, 1 CPU is simulated.
The
{\em NCPUS()}
standard function returns
the parameter given in the command line.
The
{\em MYCPU()}
standard function returns the number of the simulated CPU on which the
calling Orca process is run.

\subsection{Running Orca programs using the Panda RTS}

\subsubsection{On Solaris 2, BSD/OS or Linux}

To run an Orca program using the Panda RTS under Solaris 2, BSD/OS or Linux, it
must be compiled on the corresponding system using the
{\html {\htmladdnormallink {oc\_panda} {http://www.cs.vu.nl/proj/orca/oc_panda.1.html}}}
{\latex {\em oc\_panda}}
command.
Additional compile flags determine exactly which version of the Panda RTS
and Panda library are chosen.
This will create an executable file, which can be run using the
{\em prun} command, as follows:
\begin{quote}
\verb+prun+ [ flags ] {\em exec\_file nr\_cpus} ...
\end{quote}
See the
{\html {\htmladdnormallink {prun manual page} {http://www.cs.vu.nl/das/prun/prun.1.html}}}
{\latex {{\em prun} manual page}}
for more details, and the
{\html {\htmladdnormallink {Panda documentation} {http://www.cs.vu.nl/proj/panda/}}}
{\latex {Panda documentation}}
for the
currently available Panda implementations.

\subsubsection{On SunOS4}

To run an Orca program using the Panda RTS under SunOS4, it
must be compiled on the corresponding system using the
{\html {\htmladdnormallink {oc\_panda} {http://www.cs.vu.nl/proj/orca/oc_panda.1.html}}}
{\latex {\em oc\_panda}}
command.
This will create an executable file, which can be run using the
{\em prun} command, as follows:
\begin{quote}
\verb+prun+ [ flags ] {\em exec\_file nr\_cpus} ...
\end{quote}
This command requires the existence of an environment variable PANDA\_HOSTS,
which should contain a space-separated list of hosts.
Alternatively, the option "--{\bf h} {\em hostfile}"
specifies a file containing host names.

{\em Prun} accepts the following options:
\begin{description}
\item[--{\bf c} {\em directory}]
This option sets {\em prun}'s idea of the current directory.
This may be needed to make sure that the path to the directory where the
binary is executed exists on all hosts.
\item[--{\bf t} {\em timeout}]
This option sets the maximum number of timeout checks to {\em timeout}.
{\em Prun} kills the program when the limit is exceeded.
The default value for {\em timeout} is 60.
\item[--{\bf d} {\em delta}]
This option sets {\em delta} (the timeout check interval). 
The default value is 10 seconds.
{\em Prun}
checks every {\em delta} seconds whether the program has finished.
\item[--{\bf h} {\em hostfile}]
This option specifies a file containing host names.
\item[--{\bf f} {\em first\_host}]
This option specifies that this call of {\em prun} starts the program on hosts
numbered from {\em first\_host} upwards. The default value is 0.
\item[--{\bf l} {\em last\_host}]
This option specifies that this call of {\em prun} starts the program on hosts
numbered up to {\em last\_host}. The default value is {\em nr\_cpus}. The options
--{\bf f} {\em first\_host} and --{\bf l} {\em last\_host} 
allow parallel execution of
different binaries, for instance on a heterogeneous system.
\end{description}
For example,
\begin{quote}
\begin{verbatim}
prun -h hosts -t 120 -c /usr/proj/orca/examples/asp a.out 4
\end{verbatim}
\end{quote}
will run the a.out binary in /usr/proj/orca/examples/asp with a timeout
of 20 minutes (checked every 10 seconds), on the first 4 hosts listed in
the file "hosts".

\subsubsection{On Amoeba}

To run an Orca program using the Panda RTS under Amoeba, it
must be compiled on Unix using oc\_panda --amoeba.
The executable can be installed on Amoeba using the
"ainstall" command:
\begin{quote}
\begin{verbatim}
ainstall a.out /home/<filename>
\end{verbatim}
\end{quote}
Alternatively, when the Orca program is compiled on Amoeba, oc\_panda
must be used.
In this case, the --amoeba flag is not needed (but does no harm).
The program can now be executed by logging in under Amoeba and using
the
{\em gax} (U)
command:
\begin{quote}
\verb+gax -p +{\em capability-dir filename nr-cpus} ...
\end{quote}
For example, to run a program called
{\em orca/tsp}
on three processors whose capabilities are stored in the
directory
{\em hosts},
execute:
\begin{quote}
\begin{verbatim}
gax -p hosts orca/tsp 3
\end{verbatim}
\end{quote}

After the number of CPUs, RTS options can be passed to the Panda
RTS. These options are described in the following
subsection.

\subsubsection{Panda RTS options}

The Panda RTS recognizes many options, some of which are primarily of
interest of the developers of the Panda RTS. These options are marked
with the phrase {\bf [Developers only]}. Some users
may want to experiment with these options anyway.

A number of the options listed below relate to {\em object
distribution}. By this we mean the way the RTS places shared objects
on processors. Objects are either replicated on all processors that
run an Orca process that references the object, or they are not
replicated at all. The distribution decisions are taken by the RTS,
but can be influenced through \verb+Strategy+ calls in Orca programs
and through command-line options to the RTS (--{\bf no\_strategy},
--{\bf dynamic}, and --{\bf combined}). Users who want to know eaxctly
what decisions were taken by the RTS should use --{\bf v}.
\begin{description}
\item[--{\bf v}]
Prints object distribution information. When the
RTS decides to migrate, replicate, or unreplicate some shared object,
this is printed on standard output.
\item[--{\bf no\_strategy}]
The RTS ignores all \verb+Strategy+ calls made by the Orca program. All
object distribution decisions are taken by the RTS itself. This is
used to avoid unnecessary recompilation.
\item[--{\bf dynamic}]
Normally, the Panda RTS bases its object distribution decisions on
compiler-generated information. With this option, the
compiler-generated information is ignored. Instead, the RTS uses
runtime statistics to take decisions.  Users that use arrays of shared
objects should probably use this option or else --{\bf combined} (see below).
\item[--{\bf combined}]
With this option, the RTS uses both the compiler-generated information
and runtime statistics to take its object distribution decisions.
\item[--{\bf dedicated}]
Use a dedicated processor to sequence group messages. No Orca
processes will be run on this processor. Although this option sacrifices one
processor, it speeds up write operations on replicated objects.
\item[--{\bf tlevel} {\em level}]
Only valid when the program was compiled with --{\bf trc}. With this
option, only events with a trace level greater than {\em level} are
written to the trace file.
\item[--{\bf p} {\em replication\_policy}]
To have the RTS always replicate shared objects,
this option should be used with {\em replication\_policy} 2. To never
replicate shared objects, use {\em replication\_policy} 1. Finally,
{\em replication\_policy} 3 gives the default behavior, where the RTS takes
all decisions (using compile-time information).
\item[--{\bf warm\_start} {\em rpc\_flag group\_flag}]
{\bf [Developers only]}. If {\em rpc\_flag} is nonzero,
all processors send an RPC to all other 
processors before the Orca program is started.
If {\em group\_flag} is nonzero, then all processors send a group message.
These flags are used to "warm up" the Panda communications.
The option really only
makes sense on Amoeba, where it avoids slow startups of Orca programs.
\item[--{\bf statistics}]
{\bf [Developers only]} Print group communication statistics.
\item[--{\bf g} {\em message\_size}]
{\bf [Developers only]} PB/BB switch. Switch to BB for group messages that
are larger than {\em message\_size} bytes, otherwise use BB.
\item[--{\bf threads}]
{\bf [Developers only]} Use popup threads instead of creating
continuations.
\item[--{\bf account}]
{\bf [Developers only]} This option is used to collect statistics about
operations. Each processor creates an output file that contains its
statistics. The statistics tell how many operations, and of what type,
were performed on shared objects. Some work needs to be done before
this output is meaningful to non-developers.
\end{description}

\section{Running an example program}\label{sec:example}

Several example Orca programs are contained in subdirectories
of {\em \$ORCA/examples}.
Each subdirectory contains the sources of one Orca program.

We will describe the execution of an example program, for the
All-pairs Shortest Paths (ASP) problem.
The sources for this program are in the directory
{\em \$ORCA/examples/asp}.

\subsection{Running ASP using the Panda RTS, on Amoeba}

To run the ASP
program under Amoeba, first compile it (under Unix) by typing
\begin{quote}
\begin{verbatim}
oc_panda -amoeba $ORCA/examples/asp/asp.imp
\end{verbatim}
\end{quote}
This creates an {\em a.out} file.
Next, install this {\em a.out} file under Amoeba by typing
\begin{quote}
\begin{verbatim}
ainstall a.out /home/orca/asp
\end{verbatim}
\end{quote}
(assuming that there is an "orca" directory).
Next, log in on the Amoeba system.
You can now start ASP using the
{\em gax}
command, for example
\begin{quote}
\begin{verbatim}
gax orca/asp 4
\end{verbatim}
\end{quote}
will run ASP on 4 processors.

\section{Run-time errors}\label{sec:errors}

If the run-time system detects an error, it prints a message
containing the file name and line number where the error occurred
(unless file name and line number administration were disabled),
and an explanation of the error.
Most error messages speak for themselves.
A list is given below.

The following error messages exist:
\begin{itemize}
\item
FROM on empty set or bag
\item
array bound error
\item
assertion failed
\item
divide by 0
\item
range error in CHR or VAL
\item
modulo by $<$= 0
\item
local deadlock;
This error occurs when a non-nested blocking operation is performed on
a local object, and the operation would block.
In this case, there is no other process that could change the value of the
object, so the process is in a deadlock.
\item
bad nodename;
This error occurs when a node of a graph is selected with an incorrect
nodename.
\item
union error;
This error occurs when a union field is accessed that does not correspond
with the current union tag value, or if the union is not initialized.
\item
illegal cpu number;
This error occurs in a FORK with a cpu number outside the range 0 .. NCPUS()-1.
\item
case error;
This error occurs in a CASE statement if none of the labels contains the
case value, and there is no ELSE part.
\item
illegal alias;
This error occurs if a shared parameter is an alias of another shared or
output parameter.
\item
no RETURN from function procedure/operation;
This error occurs when a function procedure or operation should return
a value, but does not ("falls through").
\end{itemize}

Other (RTS dependent) error messages exist, for instance ``out of memory''.

\section{Standard modules and object types}\label{sec:standard}

The specifications for the standard modules and object types can be found
in the directory 
{\em \$ORCA/lib/std}.
Currently, some of the available modules and object types are described
in the following subsections.

\subsection{InOut}

The {\em InOut} module contains a number of routines for reading and writing
the values of basic Orca types, on standard input/output.
For reading, the following routines exist:
\begin{quote}
\begin{verbatim}
FUNCTION ReadChar(c: OUT char);
FUNCTION ReadInt(i: OUT integer);
FUNCTION ReadLongInt(i: OUT longint);
FUNCTION ReadReal(r: OUT real);
FUNCTION ReadLongReal(r: OUT longreal);
FUNCTION ReadString(s: OUT string);
\end{verbatim}
\end{quote}
In addition, the user can write his own routines with the help of the
following routines:
\begin{quote}
\begin{verbatim}
FUNCTION Ahead(): char;
FUNCTION Eof(): boolean;
FUNCTION Eoln(): boolean;
\end{verbatim}
\end{quote}
The \verb+Eof+ and \verb+Eoln+ test for end-of-file and end-of-line
respectively, and \verb+Ahead+ returns the next input character
without actually accepting it (it does not move the read pointer).

For writing, the following routines exist:
\begin{quote}
\begin{verbatim}
FUNCTION WriteChar(c : char);
FUNCTION WriteInt(i : integer);
FUNCTION WriteLongInt(i : longint);
FUNCTION WriteReal(r : real);
FUNCTION WriteLongReal(r : longreal);
FUNCTION WriteString(str: string);
FUNCTION WriteLn();
\end{verbatim}
\end{quote}
In addition, input and output can be redirected using the following routines:
\begin{quote}
\begin{verbatim}
FUNCTION OpenInputFile(f: string): boolean;
FUNCTION OpenOutputFile(f: string): boolean;
FUNCTION CloseInput();
FUNCTION CloseOutput();
\end{verbatim}
\end{quote}
After a successful call to \verb+OpenInputFile+ the input will be read from
the file indicated by \verb+f+.
After calling \verb+CloseInput+, further input will again be read from
standard input.

There is also a routine
\begin{quote}
\begin{verbatim}
FUNCTION Flush();
\end{verbatim}
\end{quote}
which flushes the currently buffered output. Usually, the output is
unbuffered or line-buffered when writing on a terminal, but buffered
when writing on a file.

\subsection{Conversions}

The {\em conversions} module contains some string-to-number and number-to-string routines:
\begin{quote}
\begin{verbatim}
FUNCTION StringToReal(s: string; eaten: OUT integer): longreal;
FUNCTION StringToInt(s: string; eaten: OUT integer): longint;
FUNCTION RealToString(r: longreal; p: integer): string;
FUNCTION IntToString(i: longint): string;
\end{verbatim}
\end{quote}
The StringToXxx routines convert the string \verb+s+. After completion,
\verb+eaten+ indicates the number of characters actually used in the
conversion, including leading spaces. Leading spaces are skipped, but the count
is included in \verb+eaten+. If the conversion fails, \verb+eaten+ is set to 0
and 0 is returned.

\subsection{Random}

The {\em Random} object type enables the user to draw random numbers.
There are two operations:
\begin{quote}
\begin{verbatim}
OPERATION val() : integer;
OPERATION val01() : real;
OPERATION init(seed : integer);
\end{verbatim}
\end{quote}
The operation \verb+init+ initializes the object.
Different values for \verb+seed+ give different sequences of random numbers.
The operation \verb+val+ returns the next number in the random number
sequence.
The operation \verb+val01+ returns a random number between 0 and 1.
If \verb+init+ is not called, a default initialization is chosen.

\subsection{Finish}

The {\em Finish} module is needed for some runtime systems which don't detect when the
program is finished.
In this case, the last process alive should call the following routine:
\begin{quote}
\begin{verbatim}
FUNCTION Finish();
\end{verbatim}
\end{quote}

\subsection{GenericBin}

The {\em genericBin} generic object type must be instantiated with a type \verb+T+.
There are two operations:
\begin{quote}
\begin{verbatim}
OPERATION put(e: T);
OPERATION get(e: OUT T);
\end{verbatim}
\end{quote}
The \verb+put+ operation blocks until the object is empty, and then copies
\verb+e+ to it.
The \verb+get+ operation blocks until the object contains something, and
then takes the contents and puts it in \verb+e+, thereby emptying the
object.

\subsection{GenericBinJobQueue}

The {\em GenericJobQueue} generic object type implements a queue.
It must be instantiated with a type \verb+T+.
Items of type \verb+T+ can be added to the end of the queue, or taken
from the start of the queue.
There are three operations:
\begin{quote}
\begin{verbatim}
OPERATION AddJob(job: T);
OPERATION NoMoreJobs();
OPERATION GetJob(job: OUT T): boolean;
\end{verbatim}
\end{quote}
The operation \verb+AddJob+ adds an item \verb+job+ of type \verb+T+
to the queue.
The operation \verb+GetJob+ blocks until either the queue contains an
item or \verb+NoMoreJobs+ is called.
It returns \verb+true+ if an item could be taken from the queue,
and in this case puts the item in \verb+job+.
When no more items will be added to the queue, the operation \verb+NoMoreJobs+
should be called, otherwise \verb+GetJob+ will ultimately block forever.

% Maybe the name for this object is wrong???
% How about GenericQueue, AddItem, NoMoreItems, GetItem.

\subsection{IntObject}

The {\em IntObject} object type exports the following operations:
\begin{quote}
\begin{verbatim}
OPERATION value(): integer;
OPERATION assign(v: integer);
OPERATION min(v: integer);
OPERATION inc();
OPERATION dec();
OPERATION AwaitValue(v: integer);
\end{verbatim}
\end{quote}
The operation \verb+value+ returns the current value of the
object, \verb+assign+ stores the value \verb+v+ in the object, \verb+min+
stores the minimum of the object and \verb+v+ in the object, \verb+inc+
increments the object, \verb+dec+ decrements the object, and \verb+AwaitValue+
blocks until the object has the value indicated by \verb+v+.

\subsection{PollSequence}

The {\em PollSequence} object type implements a sequence of simple polls.
If all voters in
a simple poll say yes, the result of the simple poll is yes; if anyone
says no, the result is no.
The following operations are exported:
\begin{quote}
\begin{verbatim}
OPERATION init(n: integer);
OPERATION vote(iter: integer; YesOrNo: boolean);
OPERATION AwaitDecision(iter: integer): boolean;
\end{verbatim}
\end{quote}
The operation \verb+init+ initializes the object for \verb+n+ voters.
The operation \verb+vote+ can be used to vote in the simple poll with
iteration number \verb+iter+.
The operation \verb+AwaitDecision+ blocks until the result of the
simple poll with iteration number \verb+iter+ is known, and returns it.
Once the result is yes, no more votes will be accepted.

\subsection{RowCollection}

The {\em RowCollection} object type maintains a collection of the following type:
\begin{quote}
\begin{verbatim}
TYPE RowType = ARRAY[integer] OF integer;
\end{verbatim}
\end{quote}
It exports the following operations:
\begin{quote}
\begin{verbatim}
OPERATION init(size: integer);
OPERATION AddRow(iter: integer; R: RowType);
OPERATION AwaitRow(iter: integer): RowType;
\end{verbatim}
\end{quote}
The operation \verb+init+ sets the size of the row collection, \verb+AddRow+
adds the row with iteration number \verb+iter+ to the collection,
and \verb+AwaitRow+ blocks until the row with iteration number
\verb+iter+ is available, and then returns it.

% Should this really be a standard object??? Maybe a generic object???

\subsection{Time}

The {\em Time} module exports the following functions:
\begin{quote}
\begin{verbatim}
FUNCTION GetTime() : integer;
FUNCTION SysMilli() : integer;
FUNCTION SysMicro() : real;
FUNCTION Sleep(sec, nanosec: integer);
\end{verbatim}
\end{quote}
The function \verb+GetTime+ returns the time (with respect to an unspecified
origin) in deciseconds, and \verb+SysMilli+ does the same, but in milliseconds,
and \verb+SysMicro+ does the same, but in microseconds.
\verb+Sleep+ suspends the current Orca process for the indicated amount of
time.

% Should be replaced by GetClock(sec, nanosec: OUT integer) or something like
% that.

\subsection{Barrier}

The {\em barrier} object type implements a barrier synchronization mechanism.
It exports an operation and a function:
\begin{quote}
\begin{verbatim}
OPERATION init(n_workers : integer);
FUNCTION sync(b: SHARED barrier);
\end{verbatim}
\end{quote}
The operation \verb+init+ initializes the object for \verb+n_workers+
processes. 
When these processes want to synchronize, they must all call the
function \verb+sync+, which blocks until it is called \verb+n_workers+
times, and then releases all threads blocked on it, and then re-initializes
the barrier.
(Unfortunately, \verb+sync+ cannot be an operation; internally, it consists
of two operations that both may block).

\subsection{Unix}

The {\em unix} module contains an interface to some of the more often used
Unix
system calls, a.o.:
\begin{quote}
\begin{verbatim}
TYPE iobuf = string;
FUNCTION open(name: string; mode: integer): integer;
FUNCTION creat(name: string; mode: integer): integer;
FUNCTION read(  fildes: integer;
                buffer: OUT iobuf;
                nbytes: integer) : integer;
FUNCTION write( fildes: integer;
                buffer: iobuf;
                nbytes: integer): integer;
FUNCTION close(fildes: integer): integer;
\end{verbatim}
\end{quote}

\subsection{Args}\label{mod:args}

The {\em args} module offers a unified way to pass arguments to Orca programs,
by using the following convention, which all RTSs must obey: all arguments
after a "--OC" are available to the Orca program.
The following functions are available:
\begin{quote}
\begin{verbatim}
FUNCTION Argc() : integer;
FUNCTION Argv(n: integer): string;
FUNCTION Getenv(s: string): string;
\end{verbatim}
\end{quote}
As an example, for the call
\begin{quote}
\begin{verbatim}
a.out 5 -OC arg1 arg2 arg3
\end{verbatim}
\end{quote}
\verb+Argc()+ will return 4, \verb+Argv(0)+ will return "a.out",
\verb+Argv(1)+ will return "arg1", et cetera.

\subsection{Math}

The {\em Math} module provides the following functions and constants:
\begin{quote}
\begin{verbatim}
CONST    PI = 3.1415926535;
CONST    E  = 2.71828183;

FUNCTION sin(x : longreal) : longreal;
FUNCTION cos(x : longreal) : longreal;
FUNCTION tan(x : longreal) : longreal;

FUNCTION asin(x : longreal) : longreal;
FUNCTION acos(x : longreal) : longreal;
FUNCTION atan(x : longreal) : longreal;
FUNCTION atan2(y, x : longreal) : longreal;

FUNCTION sinh(x : longreal) : longreal;
FUNCTION cosh(x : longreal) : longreal;
FUNCTION tanh(x : longreal) : longreal;

FUNCTION exp(x : longreal) : longreal;
FUNCTION log(x : longreal) : longreal;
FUNCTION log10(x : longreal) : longreal;
FUNCTION pow(x, y : longreal) : longreal;          # x ^ y

FUNCTION sqrt(x : longreal) : longreal;

FUNCTION ceil(x : longreal) : longreal;
FUNCTION floor(x : longreal) : longreal;

FUNCTION fabs(a : longreal) : longreal;

FUNCTION ldexp(a : longreal; n : integer) : longreal;
FUNCTION frexp(a : longreal; n : OUT integer) : longreal;

FUNCTION modf(a : longreal; n : OUT longreal) : longreal;
FUNCTION fmod(a,b : longreal) : longreal;

FUNCTION IntToLongReal(n : integer) : longreal;
FUNCTION LongRealToInt(r : longreal) : integer;
FUNCTION LongRealMin(a, b : longreal) : longreal;
FUNCTION LongRealMax(a, b : longreal) : longreal;
\end{verbatim}
\end{quote}

% Other modules/objects???

\section{Writing modules in C}\label{sec:interface}

Sometimes it is needed to write a module in C, for instance to interface
to a library which is not available for Orca, but is available for C.
If this module is called
{\em modnam},
the user must write a specification module
{\em modnam.spf}
which
contains Orca specifications for all functions that the user wants to write
in C.
The
{\em oc}
program will compile this specification, because it detects that there
is no corresponding module implementation in Orca.
It will produce a 
{\em modnam.c}
and a
{\em modnam.h}
file for it.

The user must of course also provide the C implementation of these functions.
These must reside in a .c file (with a different name!) that must
start with
\begin{quote}
\verb+#include <interface.h>+ \\
\verb+#include "+{\em modnam}\verb+.h"+
\end{quote}
Each function {\em f}
must have a corresponding implementation in C with the name
\begin{quote}
\verb+f_+{\em modnam}\verb+__+{\em f}
\end{quote}
If the function does not return a value, or it returns a complex type,
it must be declared
{\bf void}.
If it returns a simple type, the following table indicates which type
to use for each simple Orca type.

\begin{quote}
\begin{tabular}{|l|l|l|}\hline
{\bf Orca type} & {\bf C type} & {\bf usual C implementation} \\ \hline
{\em shortint} & \verb+t_shortint+ & \verb+short+ \\
{\em integer} & \verb+t_integer+ & \verb+int+ \\
{\em longint} & \verb+t_longint+ & \verb+long+ \\
{\em shortreal} & \verb+t_shortreal+ & \verb+float+ \\
{\em real} & \verb+t_real+ & \verb+double+ \\
{\em longreal} & \verb+t_longreal+ & \verb+double+ \\
{\em char} & \verb+t_char+ & \verb+unsigned char+ \\
{\em enum} & \verb+t_enum+ & \verb+unsigned char+ \\
{\em boolean} & \verb+t_boolean+ & \verb+unsigned char+ \\ \hline
\end{tabular}
\end{quote}

OUT or SHARED parameters are passed by reference (an address is passed),
IN parameters are passed by value if they have a simple type, otherwise
they are passed by reference as well.
(This means that the user has to be careful not to change IN parameters.)
In addition, if the function returns a complex type, a parameter should be
added as if it is an OUT parameter.

As a small example, if the user wants to use the
{\em exp}
function of C, he could have the following specification module math.spf:
\begin{quote}
\begin{verbatim}
MODULE SPECIFICATION math;
    FUNCTION exp(x: REAL): REAL;
END;
\end{verbatim}
\end{quote}
and the following exp.c file:
\begin{quote}
\begin{verbatim}
#include <interface.h>
#include "math.h"

double exp(double);

t_real f_math__exp(t_real x)
{
    return (t_real) exp((double) x);
}
\end{verbatim}
\end{quote}

The user should also explicitly add exp.c to the list of files offered to
{\em oc}.

\end{document}
