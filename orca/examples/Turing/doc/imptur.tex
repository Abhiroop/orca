\section{Implementing the Turing Ring}
\label{sImpTur}

Implementing the Turing Ring in a serial language is
straightforward. The ring can be represented by an array of records,
each of which contains the prey and predator population for a single cell.
The program then loops over all cells in the ring, determining for
each individual whether it gives birth, dies, or
migrates. For migration a second array is used, to which the processed
animals are moved. This avoids processing twice the animals which
migrate to a cell which is not yet processed.

\begin{figure}
\begin{tgrind}
\File{sertur.pseu},{14:42},{Sep 11 1994}
\L{\LB{Read parameters}}
\L{\LB{Initialise}}
\L{\LB{\K{FOR} N iterations \K{DO}}}
\L{\LB{    Empty the other array}}
\L{\LB{    \K{FOR} all cells \K{DO}}}
\L{\LB{        \K{FOR} all predators\/prey \K{DO}}}
\L{\LB{            Determine birth and death}}
\L{\LB{            Migrate them to the other array}}
\L{\LB{        \K{OD}}}
\L{\LB{    \K{OD}}}
\L{\LB{    Swap the arrays}}
\L{\LB{\K{OD}}}

\end{tgrind}
\caption{Serial Implementation of the Turing Ring}
\label{fSerialTuring}
\end{figure}

\begin{figure}
\epsfxsize=\textwidth
\epsfbox{migration.eps}
\caption{Representation of the Turing Ring}
\label{fRepresentation}
\end{figure}

The Turing Ring was implemented in ANSI~C using this method.
A simple graphical interface was added to inspect the results.
Profiling this program revealed that 90\% of the
execution time was spent generating random numbers. 

One way to reduce this cost would be to increase the efficiency of the
way the random numbers are used. To determine if one predator or prey
gives birth or dies does not require a 32-bit random number. The
program could split the random number into parts, thus generating only
one random number for a few probabilistic decisions. This was not
implemented, because it was not in the scope of this work.

The ANSI~C program was then translated into strictly sequential Orca, and 
run on one
processor. This way the speed of Orca could be compared with that of
ANSI~C. It was found that the Orca program was about 2.5 times slower
than ANSI~C. This was mostly due to the dynamic array bound checking done
by the Orca run-time system.

\subsection {Parallelising the Turing Ring}

To parallelise the Turing Ring the problem must be {\em decomposed}.
There are two obvious compositions of the Turing Ring: a spatial
decomposition or a decomposition per animal.

A spatial decomposition would have each processor take care
of a number of cells. Since animals only move between adjacent cells, it
would furthermore be practical to have adjacent cells be processed by
one processor. This way a processor would only have to exchange information
regarding its cells with two other processors.

The decomposition per animal implies that every processor takes care of
a number of animals, which may be anywhere in the ring. This
decomposition seemed impractical, since a processor needs
information about a cell's population to determine the behaviour of a
single animal. If the population of a cell is scattered over all 
processors, this information is not readily available.
Therefore, the first approach was taken.

\begin{figure}
\epsfxsize=\textwidth
\epsfbox{distributing.eps}
\caption{Spatial Decomposition of the Turing Ring}
\label{fSpatialDecomposition}
\end{figure}

Each processor holds a proper
interval of the ring. Most of the calculations for animal migration can
then be
processed on the processor itself, as in the serial program. Only the
animals migrating from a processor's border cells need to be sent to
another processor.

This means that each processor needs to know which
cells are located on which processor. This can be accomplished by using
a shared array which contains this information. A simpler way is also
possible. If every processor contains a proper interval of the total
number of cells, and these intervals are given to the processors in
sequence, processor $i$ only needs to communicate with processor
$i - 1$ and processor $i + 1$ (and processor $1$ with processor $n$ and
vice versa).

The first solution has higher communication overhead,
but might be better in the general
case. In a two dimensional grid for example, load balancing would be much
harder using the latter approach, since the processors could not keep track
of where cells are located (see Section \ref{sPrevWork}).
In a ring, this problem does not occur. Therefore, the first solution was
chosen for this program.

\subsubsection{Implementation of the Turing Ring}

The parallel program consists of two modules. The main program is
contained in the module {\tt Turing}.
The communication channel between processors is implemented by an
object called {\tt LinkObj}.

\begin{figure}
\begin{center}
\begin{tabular}{|l|l|}
\hline
DisplayFreq: integer	& Display frequency \\
\hline
NOCells: integer	& Number of cells \\
\hline
NOIterations: integer	& Number of iterations \\
\hline
RandomSeed: integer	& Seed to initialise random numbers \\
\hline
TimeStep: real		& time step used \\
\hline
Explode: integer	& maximum number of animals in a cell \\
\hline
PredPopInit: integer	& initial predator population \\
\hline
PredDeathA: real	& \\
PredDeathB: real	& predator coefficients \\
PredBirth: real		& \\
PredMigrate: real	& \\
\hline
PreyPopInit: integer	& initial prey population \\
\hline
PreyDeathA: real	& \\
PreyDeathB: real	& prey coefficients \\
PreyBirth: real		& \\
PreyMigrate: real	& \\
\hline
\end{tabular}
\end{center}
\caption{The {\tt Params} data structure}
\label{fParams}
\end{figure}

The main process, {\tt OrcaMain}, initialises the work. First, it
reads the user-specified parameters from the standard input. (The
current Orca system has no support for command line parameters.) It
then forks off as many worker processes as there are CPUs available
and blocks until the worker processes have finished.

In the main process two arrays of link objects are declared. Each consecutive 
pair
of worker processes gets one link object of each array passed as a shared
parameter. These two links act as read and write channels between
neighbouring worker processes.

Each worker initially has a range of cells assigned to it for
processing. At the start, all cells are distributed evenly over 
all processors.
Each worker process also gets a copy of the parameters
the main process has read.

The worker process uses two arrays to store its part of the ring.
It initialises its part using the user provided parameters and then
starts iterating. On each iteration, it performs the following
steps:
\begin{itemize}
	\item[1.] Accept the animals which migrated from the neighbouring
	processors during the previous iteration.
	\item[2.] Display the current state of its part of the ring,
	if necessary.
	\item[3.] Generate birth and death in its part of the ring.
	\item[4.] Handle migration within its part of the ring, and
	determine which animals move to the neighbouring CPUs.
	\item[5.] Send migrating animals to its neighbours.
	\item[6.] Wait for the other workers to finish this iteration.
\end{itemize}
For inter-processor communication (step 1 and 5) the link object is
used. This is an object which contains operations for writing and
reading data which denote the number of predators and prey which
migrate between processors.

The synchronisation in step 6 is necessary to ensure that each
worker actually receives the migration data sent to it in step 5. Not
every worker will finish step 3 and 4 at the same time, since the
amount of work done in these steps is determined by the number of
animals which is presently in its part of the ring. Since this number
can vary by hundreds, some processors may have to wait.

After all iterations have been performed, the workers inform the main
process that they are done using barrier synchronisation. The main
process can then terminate.

\subsubsection{Load Balancing}

As mentioned above, a processor may have to wait for other processors
to finish their work. If the variance of execution time of an iteration
between processors becomes too large, some processors will do nothing
most of the time while others have lots of work to do. To prevent this
situation, a load balancing strategy has to be used. It should ensure
that the workload is evenly distributed amongst processors during every
iteration.

Initially, all cells in the ring contain the same number of animals,
so an even distribution of workload can be achieved by distributing
the cells evenly amongst processors at startup. This is called static
load balancing.

However, since populations can vary dramatically, some dynamic load
balancing has to be done as well. This means a processor must be able to
pass some of its work, i.e.\ some of its cells, to another, lightly
loaded, processor during run-time. Devising a good dynamic load balancing
strategy was a major part of this work.

\begin{figure}
\epsfxsize=\textwidth
\epsfbox{shifting.eps}
\caption{Shifting a cell to a neighbour}
\label{fShifting}
\end{figure}


It would be possible to have a master process collect all information and
then redistribute the work. However, as a consequence of the spatial
decomposition as depicted in Figure \ref{fSpatialDecomposition}, a
processor should only shift cells to its neighbours. Furthermore, the
populations (thus workload) of a processor's cells are not likely to
change rapidly. Therefore, a
processor can decide whether to shift cells to its neighbours based on
local data. This takes away the need for a central master process. In
addition, the channels for information exchange between neighbouring
processors have already been defined for migration. The load balancing
information can therefore be piggy-backed with this data.

A naive load balancing algorithm would do the following. On each
iteration, it would compare its own workload with that of its
neighbours. If the difference between the loads exceeded a certain
threshold, cells would be shifted to or from the neighbour (Figure
\ref{fShifting}).

This load balancing scheme suffers from two
problems. First, if most of the load of a heavily loaded 
processor comes from
one cell, (one cell contains almost all of the predators
and prey it has to handle,) and this cell happens to be a border cell,
shifting the cell would result in the neighbouring processor becoming
heavily loaded while the processor itself would lose most of its load. So
in the next iteration the neighbour will shift the cell back in an
attempt to redistribute the total workload more evenly. This way a
heavily loaded cell would be shifted back and forth until its population
has died out. This phenomenon is called {\em thrashing}.

Secondly, the following situation might occur. Imagine that a processor's
neighbours are both heavily loaded, and both decide to shift cells to
that processor. Then that processor would be heavily loaded, maybe even more
than its two neighbours were, so the workload would not be evenly
distributed. This situation is called {\em overloading}.

To avoid these two problems, a better load balancing scheme is needed.
Since processors need to know more about their neighbours's load, a
{\em multi-iteration protocol} is used. A processor not only receives
its neighbours's workload, but also an estimate of what its workload
would be after shifting. Now it decides whether shifting cells will
result in a more evenly distributed workload or not and acts
accordingly.  The new control structure is outlined in Figure
\ref{fWorker}.

\begin{figure}
\begin{tgrind}
\File{worker.pseu},{15:04},{Sep 11 1994}
\L{\LB{\K{BEGIN}}}
\L{\LB{    Initialise}}
\L{\LB{    \K{FOR} a number of iterations \K{DO}}}
\L{\LB{        Get animals moving in}}
\L{\LB{        Get the cells sent here for load balancing}}
\L{\LB{        Display}}
\L{\LB{        Process birth and death in every cell}}
\L{\LB{        Process migration between cells}}
\L{\LB{        Negotiate}}
\L{\LB{        Drop cells if we are overloaded}}
\L{\LB{        Wait for neighbours}}
\L{\LB{    \K{OD}}}
\L{\LB{\K{END}}}

\end{tgrind}
\caption{The Worker Process}
\label{fWorker}
\end{figure}

The algorithm is implemented using two functions,
{\tt InitLoadBal} and {\tt Negotiate}. It uses two {\em status records} to
keep information about the current phase of negotiation with the
neighbouring CPUs. These records are initialised when {\tt
InitLoadBal} is invoked.

Each worker process calls {\tt Negotiate} on every iteration. This
function communicates with the neighbours according to a three stage
protocol:
\begin{itemize}
        \item[1.] Send current workload to both neighbours, along
        with border cell's load.
        \item[2.] Receive neighbours's load information, send expected load.
        \item[3.] Receive neighbours's expected load.
\end{itemize}
In stage 1 all workers send their current load to their neighbours.
Since the workload is determined by the number of animals a processor
has to manage, it is actually this number that is sent. The worker also
sends the number of animals in the cell bordering the neighbour's cell.
This is the amount of load the worker would lose, if it decided to
shift a cell to the neighbour.

The load information is written to the link structures which connect the
neighbouring workers. It is not necessary to create another
communication channel if {\em piggy-backing} can be used. This however prevents
the load balancing algorithm from being implemented as an object, since
Orca does not allow shared variables as parameters to an
operation. (The link structures have to be passed to {\tt Negotiate}.)

In stage 2 every worker receives load balancing information from its
neighbours. It can then calculate the difference between its own load
and that of its neighbours. If the difference exceeds a certain threshold, it
decides to shift cells to or from the neighbour. Since the neighbour
makes the same calculations, it makes the same decision.
Now the worker calculates what its future load will be if all the shifts
actually take place, and sends this number to its neighbours.

In stage 3 each worker receives its neighbours's expected load, so it
knows what load it will have and what load its neighbours will have. It
now reconsiders the shifting of cells as it was planned in the previous
step.

If the worker wanted to send, but this would result in a situation where
its neighbour would get so much to do that the difference in load would
exceed the threshold again, there is no use in sending, so the worker
does not do it. This way thrashing can never occur: a cell which
contains so much load that the neighbour will want to send it back
immediately will never be sent. Also overloading is prevented, since a
worker takes into account what its neighbour's other neighbour will
send.

Another unwanted situation now can occur. Suppose a cell wants to send
to both neighbours. Its expected load will be its current load minus
the load of its two border cells. If this load is low compared to its
neighbours', it will refrain from sending in both direction, even if
the load distribution would be better if it would send to one of its
neighbours (Figure \ref{fSendOne}).

\begin{figure}
\epsfxsize=\textwidth
\epsfbox{sendone.eps}
\caption{A Worker which should send to one side}
\label{fSendOne}
\end{figure}

If CPU~5 would want to send cells to both CPU~4 and CPU~6, its future
load would be very low. But sending its border cell to CPU~6 would not
happen anyway, because CPU~6 would send it back immediately. Sending to
CPU~4 still seems a good idea, but if the future workload of CPU~5 is
compared to that of CPU~4, this will not happen either.

The following solution to this problem is used in the current
implementation. When a worker
finds out it planned to send to both its neighbours, but after
reconsidering it does not send to either of them, it determines if it
is possible to send to only one of them. In Figure \ref{fSendOne}, CPU~5
would decide it could send to CPU~4 after all, since the difference in
load after shifting would be very small. If a worker can send to
both neighbours, but not at the same time, it sends the border cell
containing the least workload.

Using this strategy, a worker which has a high workload and a heavily
loaded border cell will drop its load on the other side instead of
thrashing the heavily loaded cell back and forth. Only if
a worker has heavily loaded border cells on {\em both} sides would load
balancing be impossible.
