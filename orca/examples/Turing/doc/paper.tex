\documentstyle[titlepage,11pt,a4,epsf,tgrind]{article}

\hyphenation{speed-up speed-ups}

\title{Parallelising the Turing Ring using Orca}

\author{Dion Nicolaas\\
	Department of Mathematics and Computer Science \\
        Vrije Universiteit \\
	Amsterdam, The Netherlands}
\date{August 31, 1994}

\begin{document}
\begin{titlepage}
%\vspace{1.5cm}
\hspace{2cm}
\parbox{10cm}{
	\begin{small}
	\begin{center}
		{\Large Parallelising the Turing Ring using Orca} \\
	\vspace{1cm}
		Dion Nicolaas\\
		Department of Mathematics and Computer Science \\
		Vrije Universiteit \\
		Amsterdam, The Netherlands \\
	\vspace{2cm}
		August 31, 1994
	\end{center}
	\end{small}
\vspace{12.2cm}
\hspace{7cm}
\epsfxsize=3.5cm
\epsfbox{grif.ps}
}
\end{titlepage}

\begin{abstract}
The Turing Ring is a discrete simulation of a chaotic dynamic system. It
is well suited for parallelisation, but balancing the load on processors
is difficult. A program was written in Orca, an ALGOL-like language for
parallel systems supporting shared data objects.
The load-balancing strategy used is described and its performance
reported. Some comments on the ease of programming in Orca are then
made.
\end{abstract}


\section{Introduction}
\label{sIntroduction}

\begin{sloppypar}
Parallel programming is becoming more important as the speed of
single-processor
systems approach the limit imposed by the speed of light. While
communication in multi-computers (such as hypercubes and transputer
grids) is still faster than communication on a local area network such
as an Ethernet, distributed
systems based on LANs of workstations are becoming more attractive for
coarse-grained parallel applications because of their wide availability
and because they can easily be expanded with off-the-shelf components.
\end{sloppypar}

Parallel programming, however, raises new difficulties for the
programmer. Since multiple CPUs work at the same time, a program's
behaviour cannot simply be deduced by reading its lines of code
step-by-step. It is difficult to determine which processor is doing
what at a certain moment. So synchronisation of the
processes is something the programmer has to think about. Another
problem is keeping data consistent between processors. As different
processors all apply different operations to copies of the same data,
these copies become different. To ensure that the copies remain consistent,
the same operations should be applied to all copies in the same
order.

These problems seem to be more easily solved on systems which offer
shared memory than on systems which use message passing as a model.
Distributed systems, however, do not have shared memory. This is why the
Orca programming language \cite{tse92,bal} provides
{\em logically shared data}, even on
systems that do not have physical shared memory. This model is called
{\em Distributed Shared Memory}.

Orca was designed at the Vrije Universiteit for implementing parallel
applications on distributed systems. It is not an extension to an
existing programming language, but is designed from scratch to integrate
sequential and parallel constructs cleanly. As part of an effort to
investigate the suitability of Orca for larger software projects, a
discrete simulation of a chaotic dynamic system, the Turing
Ring, was implemented. During the development of the program, special
attention was given to the ease of programming. Performance results
were then gathered to examine the performance of Orca-based programs.

This paper is organised as follows. First, the Turing Ring is described
in detail, and some background information is given. Next, the Orca
programming language and its implementation on top of the 
Amoeba operating system is described. Several possible
implementations of the Turing
Ring are considered, and the final program is described. Finally,
performance results are presented and the usability of Orca is
discussed.

The work presented here has been performed at the Vrije Universiteit,
Amsterdam as an afstudeerproject under the supervision of Gregory V.
Wilson \nocite{Cowichan}. Thanks for the other things he did, too.
Thanks to Marc Baehr for his numerous comments;
I would also like to thank everyone who listened to me and then sent me back to
work again.

\section{The Turing Ring}
\label{sTuringRing}
In 1952, Alan Turing wrote a paper in which he formulated a model
for describing the interaction of two chemicals in a ring of cells
\cite{Tur52}. His
work was later generalised by others to include many different kinds of
reaction-migration systems and is now of importance in many branches
of science, such as ecology \cite{Smith91}, epidemiology
\cite{Moll77} and petroleum engineering \cite{Whee88}.

Turing's model can be described as a spatial system in
which reactants interact within locations and migrate between the
locations. It can thus represent populations of predators and prey in
a series of habitats.

\begin{figure}
\epsfxsize=\textwidth
\epsfbox{introduction.eps}
\caption{The Turing Ring}
\label{fTuringRing}
\end{figure}

Turing used the following differential equations to model the
interactions of reactants and the migration to neighbouring cells.  Let
$X_i$ and $Y_i$ (where $i = 1, \ldots , n$) be the concentration of
reactants (or population of predators/prey) in each cell $i$ in a
ring-shaped world of size $n$ and let $\mu$ and $\nu$ be the migration
rates between neighbouring cells for $X$ and $Y$ respectively. Then
\begin{eqnarray}
	\frac{dX_i}{dt} & = & f(X_i, Y_i) + \mu(X_{i-1} -
		2X_i + X_{i+1}) \label{migrX} \\
	\frac{dY_i}{dt} & = & g(X_i, Y_i) + \nu(Y_{i-1} -
		2Y_i + Y_{i+1}) \label{migrY}
\end{eqnarray}
where $f(X, Y)$ and $g(X, Y)$ can be expressed as:
\begin{eqnarray}
	f(X, Y) & = & X(r_X + a_XX + b_XY) \\
	g(X, Y) & = & Y(r_Y + a_YX + b_YY)
\end{eqnarray}
Note that $i$ is `wraparound':  location $1$ borders location $n$.  By
using different values for the coefficients
$r_{\{X, Y\}}$, $a_{\{X, Y\}}$ and $b_{\{X, Y\}}$
different systems can be obtained, which for example 
can be used to describe
Volterra prey/predator systems or to model the mixing of water and oil
in porous rock, and the resulting flow through rock structure.

In this work, a predator/prey system in a ring-shaped world is
described; the system can be visualised as a model for a series of
locations around a
lake, in which foxes and rabbits live. Using this image, the equations
can be explained as follows.

$X$ denotes the predator population and $Y$ the prey population. The
migration rate $\mu$ ($\nu$) determines how many predators (prey) from
neighbouring locations migrate to location $i$, and how many predators
(prey) move to the locations $i + 1$ and $i - 1$.

$f(X, Y)$ and $g(X, Y)$ are the generation functions of the predators
and prey. For $f(X, Y)$, $r_X$ is the constant death rate, which denotes
how many predators die of age. $a_X$ determines how many predators die
of overpopulation. $b_X$ determines the birth rate, which depends further on
the prey population (which represents the food situation).

In $g(X,Y)$, $r_Y$ is the constant death rate for prey, as for
predators above. $a_Y$ determines the number of prey killed by
predators. $b_Y$ determines the birth rate, which increases if the
number of prey increases.

To calculate the evolution of the system with a computer the
differential equations are discretised. Iterations
over the populations of the locations are performed in small time
steps (typically $10^{-4}$). The populations of
the locations are updated probabilistically, i.e.\ the system decides
randomly whether each predator or prey gives birth, dies or migrates. If
the time step chosen is small enough this linear system nears the
behaviour of the equations above.

\begin{figure}
\epsfxsize=\textwidth
\epsfbox{population.eps}
\caption{Typical behaviour for predator/prey systems}
\label{fTypical}
\end{figure}

The system described above is chaotic. An expected typical behaviour
is depicted in Figure \ref{fTypical}. However, a very small change in the
coefficients can cause a large difference in the outcome. Therefore it is
possible that the populations of cells differ by large numbers.
Furthermore, parameters for the system which yield sensible results are
very hard to find.


\section{Previous Work}
\label{sPrevWork}

The most important previous work on this topic is Smith
\cite{Smith92}. His experiences are summarised here.

Smith used a generalisation of the Turing Ring in two dimensions
called the Turing Torus. 
Every cell had four neighbours, one in each of four directions. The
equations \ref{migrX} and \ref{migrY} 
from Section \ref{sTuringRing} were generalised
as follows:
\begin{eqnarray}
        \frac{dX_{i,j}}{dt} & = & f(X_{i, j}, Y_{i, j}) +
	\mu(X_{i \pm 1, j} - 2X_i + X_{i, j \pm 1}) \\
        \frac{dY_{i,j}}{dt} & = & g(X_{i, j}, Y_{i, j}) +
	\nu(Y_{i \pm 1, j} - 2Y_i + Y_{i, j \pm 1} )
\end{eqnarray}
Distributing the cells over a grid of 16k processors on a CM-200 
yielded a speed-up of 844 compared to a Sun 4/20
workstation. However, to avoid load imbalance the populations were not
updated probabilistically (using random numbers) but deterministically
(using fixed numbers).

The CM-200 is a SIMD architecture. This means that there is only one
stored program which is executed by every processor at the same time.
If one processor has to do more work than others, the other processors
have to do the same number of operations, even if they have no more
work to do. These operations are then made ineffective using `masking'
techniques, to prevent them affecting the results, but they obviously
affect execution time.

As stated in Section \ref{sTuringRing}, the (stochastic) Turing Ring (or
Torus) behaves
very chaotically. In order to use probabilistic updates, a load
balancing strategy had to be devised. Using spatial decomposition
(distributing the cells over a grid of processors) load
balancing is very impractical (Figure \ref{fShift2Dim}).

\begin{figure}
\epsfxsize=\textwidth
\epsffile{shift2dim.eps}
\caption{Shifting cells in a two-dimensional system}
\label{fShift2Dim}
\end{figure}

Initially, the CPU at location $(a,b)$ in the grid processes a block of
cells ($(i, j)$, $(i+1, j)$, $(i, j+1)$ and $(i+1, j+1)$.) The
small arrows in the figure denote the data exchange between cells
caused by migration. It is clear that CPU $(a,b)$ must communicate with
its neighbours to exchange information to and from its border cells.
(Two neighbouring CPUs are shown.)

If CPU $(a,b)$ has a very high load, it would want to get rid of some
of the work it does. It could decide to shift the cell at $(i+1, j+1)$
(shown highlighted) to one of its neighbours, either right
or below. It would then have less work to do, while the other CPU
would have more work. This way better load balance can be achieved.

However, one problem arises immediately. CPU $(a,b)$ will still have
information to pass to this cell, so it needs to know were it resides.
This is not as easily determined as before, particularly if cells
float away from their original processor further over time. After a while
it will be
very difficult to find out which cell is where, while in the
initial situation every CPU knew who to address.

As a result, either some global data (known to at least groups of CPUs) is
required, or CPUs should always shift a complete row or column to their
neighbours. This might make efficient load balancing very difficult.

Instead of trying to solve these problems, Smith took a completely different
approach. A different decomposition was used: the individuals
of each species were distributed amongst the processors instead of the
locations they lived in. Their location was kept in a local data
structure. This decomposition is depicted in Figure \ref{fSOS}.

\begin{figure}
\epsfxsize=\textwidth
\epsfbox{sos.eps}
\caption{Data structure to implement Scan-Order-Scan strategy}
\label{fSOS}
\end{figure}

This posed a new problem. In order to make the necessary calculations
about birth and death, a CPU must know the population of an animal's
location. This information is no longer readily available. A solution
to this is to store these
numbers in a local data structure for each animal.
However, to keep these numbers
up-to-date a lot of inter-processor communication might be needed,
since the animals of one cell are scattered all over the processors.

The final system was implemented using a technique called
Scan-order-scan (SOS). Each processor kept a
list of animals, sorted by location. Then birth and death was calculated
locally, after which the list was reorganised to remove the gaps caused
by death and to insert the newly born animals. Migration was also done locally,
as it only involved changing the location coordinates of migrating animals.
After this the list
was resorted. The new population numbers could then be determined by each
processor; they then spread their calculated fraction of the total
population among all processors who have at least one animal from the
location in question. This way the population
information for every animal's location could be updated.

Smith proposed some further improvements for efficiency. One would be
to have two arrays on each processor: one with
the animals as explained above, and one with the populations in every
location of the world. This way a CPU could look up the population
of an animal's home cell in the population matrix. A possible problem with
this scheme is that heavily loaded cells would become bottlenecks in
the population matrix, as the same data might be needed for many animals.


\section{The Orca Programming Language}
\label{sOrca}
Orca is a structured language, similar to Modula-2, designed to
implement parallel applications on a distributed system. To support
parallelism, it provides explicit dynamic process creation, i.e.\ a
program can fork off processes on different processors.
The different processes can communicate through shared data, even if the
system Orca runs on does not support shared memory. The shared data is
accessed by high-level user-specified operations, unlike most systems
with shared memory.

Orca is not an extension to an existing language. It was designed from
scratch so that the serial and parallel constructs of the language fit
tightly together. Typical constructs of serial languages which
are useless or dangerous
in parallel systems (such as pointers, which have no meaning
between machines) were avoided completely. In Orca, pointers do not
exist; neither does global data.

The serial constructs of the language are much like those of Modula-2.
Orca supports constructs such as functions, while loops and
if statements, though their syntax differs slightly from Modula-2.
Parallelism is expressed through fork statements, shared variables
and guards.

An Orca program is organised in one or more compilation units or
{\em modules}. A  module consists of two parts: a
specification part and an implementation part. Each module can import
entities from other modules's specifications.

An Orca program is run as a process, but it can fork off other processes
by using the fork statement. The fork statement starts
another process on any processor the programmer specifies; a process's
declaration is similar to a function declaration.

Arguments can be passed to functions in three mode: {\tt in}, which is
the default, {\tt out} and {\tt shared}. {\tt in}-parameters can only be
read by the function;  {\tt out}-parameters can only be written, thus
acting like a generalised return-mechanism. {\tt shared} parameters can
be read or written, much like {\tt var}-parameters in Modula-2.

The main concept in the parallel constructs of Orca is the {\em object}.
This is however not an object as in object-oriented programming,
but more like an {\em abstract data type} in Modula-2. Like a module,
an object
consists of a specification and an implementation part. Objects can be
manipulated by {\em operations}.

Arguments of a process should either be of type {\tt in} or 
{\tt shared}, but shared parameters must be an object.
Processes which run on different CPUs can communicate through
these shared objects.
An object can only
be manipulated by operations, and operations on the same object are
guaranteed to be {\em serialised}: operations take place one at a
time, in no particular order. Operations are indivisible.

The guard statement provides synchronisation for
processes. It allows operations on a certain (shared) object to block
until one of a set of conditions becomes true. A guarded statement
consists of a condition (the {\em guard}) and a block of statements;
if one or more guards in a block evaluate to {\tt true}, one of them is
chosen non-deterministically and its statements are executed. This way
a process can block until another process changes the object's data.

Note that it is difficult to keep serialisability while using guards;
the guarded statement may change the object, then wait until another
process changes the object and then make some more changes to the
object. Since operations should be indivisible the two processes's
changes should not be mixed.
The solution in Orca is that the guarded statement acts as if
it copies the object before blocking; it then repeats the changes it
made to the changed object. This way the final result is that the
other process's changes seem to have happened before the guarded
statement's changes. In reality, simple objects are seldom copied;
the system avoids it whenever possible for performance reasons.

\begin{figure}
\begin{small}
\begin{verbatim}
OBJECT SPECIFICATION SharedInt;
    OPERATION value(): integer;
    OPERATION assign(v: integer);
    OPERATION inc();
    OPERATION dec();
    OPERATION AwaitValue(v: integer);
END;



OBJECT IMPLEMENTATION SharedInt;

    x: integer;

    OPERATION value(): integer;
    BEGIN
        RETURN x;
    END;

    OPERATION assign(v: integer);
    BEGIN
        x := v;
    END;

    OPERATION inc();
    BEGIN
        x +:= 1;
    END;

    OPERATION dec();
    BEGIN
        x -:= 1;
    END;

    OPERATION AwaitValue(v: integer);
    BEGIN
        GUARD x = v DO OD;
    END;

BEGIN
    x := 0;
END;
\end{verbatim}
\end{small}
\caption{A sample object: SharedInt}
\label{fSharedInt}
\end{figure}

We will study a simple object as an example. The SharedInt object
is shown in Figure \ref{fSharedInt}. First the specification part is
shown: this is the part which can be imported by other modules. 

The implementation part contains the actual operations. The variable
{\tt x}, which contains the object's data, is declared here, and is only
visible within this implementation part. Most operations are
straightforward. 

The operation {\tt AwaitValue} contains a guard. If this
operation is called, the calling process blocks until the value of 
{\tt x} is set to {\tt v} by another process.
Then the caller is unblocked and may proceed.

The body of the implementation part contains the initialisation. It is
executed once for every SharedInt object that is declared, before any
operation takes place.



\section{Implementations of Orca}
\label{sOrcaImp}

Several implementations of Orca exist. The current compiler translates
Orca into ANSI~C, which is then compiled using {\tt gcc}.  The object
code is then linked with the Orca {\em run-time system}.
The run-time system implements the concepts of Orca as described in
Section \ref{sOrca}. It must ensure that objects
which are supposed to be shared between processes are kept
consistent, even on distributed memory.

Several different run-time systems have been implemented.
One, which runs on top of Unix, uses
threads (light-weight processes) to simulate parallelism. Since this
run-time system uses only one machine, its primary use is for testing
Orca programs on single workstations. This is the simplest of the
run-time systems, since 
implementing distributed shared memory is trivial within a single
address space.

\subsection{The Broadcast-RPC Run-Time System on Amoeba}

Another run-time system has been developed on top of Amoeba,
a distributed operating system which combines several
workstations connected with a network into a single virtual time-sharing
machine \cite{Amoeba}.
This Orca run-time system \cite{ieee92,cpe92} is the middle layer inside
a four layer
system. The lowest software layer is the {\em reliable broadcasting layer},
which
is provided by Amoeba. The top layer is the compiled application
program.

\begin{figure}
\epsfxsize=\textwidth
\epsfbox{orcastruct.eps}
\caption{Structure of the Orca system}
\label{fOrcaStructure}
\end{figure}

\subsubsection{Object Management}

The principal design choice for implementing distributed shared memory in the
Amoeba run-time system is {\em replication}.
This means that multiple copies of each shared object are
available, so that each processor that uses the object has a local copy. This 
way reading data from
an object can be done locally, without any network access.

Writing to an object is more complicated. First, every processor
which has a copy of the object needs to receive update information.
Secondly, each of these processors must appear to receive the different updates
in the same order, so that changes appear consistent across all
processors.

The run-time system places an object-manager on each processor. This
is a light-weight process that handles updating of the
objects on each processor. Each object's data is stored in an address space
shared by the user processes and the object-manager, so reading an object
can be done by the user application itself. An object can have been locked
by the object-manager, in which case the user process blocks. Locking
within one address space is easy, though.

If a write operation is requested, the object-manager marshals
the write operation and broadcasts it to all object-managers in the
system. The
object-managers then store the request in a FIFO-queue. Note that
reliable broadcasting is used: this means that every processor receives
all requests in the same order.

The object-managers now handle the queued requests in order. They
lock their local copy of the object, perform the update and release the
lock again. Each object-manager in the system handles the updates on its
local copy in the same order, so this way {\em serialisability} is
ensured. It is possible that one processor is ahead of another in updating
the object, however.

As an aside, {\em indivisibility} of operations now has become a local
affair. The object-manager locks its local copy of the object, so the
user process will block if it tries to read the object, and the write
operations are performed one at a time by the object manager.

Following the approach described above, every object is replicated on
every processor. However, full replication is impractical if an object
is shared by only a few processors, or if it has mostly write
operations. In the last case, a better system can be used, which does
not replicate objects.  Instead, the object is
stored on exactly one processor, where it can be accessed by other
processors using point-to-point messages.
Normally, the run-time system should decide on
which processor the object is stored, but the programmer can affect
this decision using {\em strategy calls}.

Most of the functionality of this system is put in the compiler. The
compiler analyses the code for processes to find out which objects
they access and how many times they read or write them. The run-time
system then decides whether to replicate objects or not. This is not 
done by the compiler, since the decision can depend on the hardware used
and the communication protocol. Leaving this decision to the run-time
system keeps the compiler architecture-independent.

Normally, the compiler cannot determine exactly how many times an object is read
or written without actually running the program. For example, it is often 
impossible to
determine how many times a loop's body will be executed or if
the body of an if-statement will be executed or not. The
compiler
therefore uses simple heuristics for these constructs, multiplying the
number of operations by a constant (such as 16 for loops and 0.5 for
if-statements.)

The compiler then passes two values to the run-time system for each
object processes can access: the estimated number of read
operations and the estimated number of write
operations. When the run-time system forks off the process, it broadcasts a
message to all processors telling them which process is forked off and
on which processor it runs. All processors now decide
whether the object should be replicated or not, based on the data they
received. Since they use the same data, they all come to the same conclusion.
An object may wind up being stored on one machine, or being replicated by
the processor which currently holds it. 
A non-replicated object can also be migrated to another processor (if that
processor accesses it more frequently.)

The decision whether to replicate or not is based on the number of
messages needed in either case. For a replicated object, each write
operation involves a broadcast. For a non-replicated object, each read
operation from another processor than the one which currently stores
the object requires a point-to-point message.

\subsubsection{Reliable Broadcasting}

The key to the run-time system is reliable broadcasting.
This is implemented in the distributed operating system
Amoeba. The main entity in the broadcasting scheme in Amoeba is the {\em
sequencer}.  The CPU on which an Orca program is initially located is chosen
the sequencer for this program. Other processes use this sequencer
for reliable
broadcast in one of two ways. The first method can be summarised as
follows:
\begin{itemize}
        \item[1.] The Orca run-time system traps to the kernel, passing
	the message to be broadcast.
        \item[2.] The kernel blocks the Orca run-time system.
        \item[3.] The kernel sends the message to the sequencer as a
        point-to-point message.
        \item[4.] The sequencer assigns a {\em sequence number} to the
        message and broadcasts it.
        \item[5.] The sending kernel receives the broadcast and unblocks
        the Orca run-time system.
\end{itemize}
If the sending kernel does not receive the broadcast message in a
fixed period of time, it assumes the message has been lost and retransmits it
to the sequencer. The sequencer on the other hand checks to see that it
does not broadcast a message twice.

The sequencer stores the broadcast message in a history buffer. If a
CPU misses a broadcast, it detects this by comparing the sequence
number of the next broadcast it receives with that of the last one it
got. If any number was skipped, the CPU sends a request for the missing
messages to the sequencer. It then passes the messages in the right
order to the Orca run-time system.

To ensure the history buffer of the sequencer does not grow too large, a
processor sending a {\em Request For Broadcast} attaches a piggy-backed
acknowledgement telling the sequencer the last sequence number it
received. Additionally it sometimes sends special messages with the
last sequence number
to the sequencer. Finally the sequencer can request this information from
every processor. If every processor received the broadcast messages up to
some sequencer number $n$, the sequencer can safely delete messages
$1 \ldots n$ from the history buffer.

A second method of reliable broadcasting is also used. It can be
summarised as follows:
\begin{itemize}
        \item[1.] The Orca run-time system traps to the kernel, passing the message to
        be broadcast.
        \item[2.] The kernel blocks the Orca run-time system.
        \item[3.] The kernel assigns a unique identifier to the
        message and broadcasts it.
        \item[4.] The sequencer receives the broadcast, and broadcasts an
        {\em Accept} message containing the identifier and the next
        sequencer number.
        \item[5.] The sending kernel receives the {\em Accept} broadcast
        and unblocks the Orca run-time system.
\end{itemize}
Processors which missed a broadcast can request it from the sequencer in
the normal way.

The difference between the two protocols is that in the first one, the
broadcast message appears on the network twice: once as a point-to-point
message to the sequencer and once as a broadcast. This uses a lot of
bandwidth, especially if the message is large. The second protocol, on
the other hand, uses two broadcasts, one for the actual message and one
for a short {\em Accept} message. This way every processor is interrupted
twice.

The first protocol is thus used for small messages and the second one
for large messages. This achieves greater efficiency.

\subsection{The Panda Platform}

A rather different approach was taken with {\em Panda} \cite{Panda},
a portable platform for parallel programming languages. Panda
is a virtual machine which offers the following abstractions:
\begin{itemize}
	\item threads
	\item Remote Procedure Call (RPC)
	\item totally-ordered group communication
\end{itemize}
These are the same constructions which made the Broadcast-RPC run-time
system on Amoeba possible. 
However, Panda can more easily be ported to non-Amoeba systems.

\begin{figure}
\epsfxsize=\textwidth
\epsfbox{panda.eps}
\caption{The Panda architecture}
\label{fPanda}
\end{figure}

Panda is implemented in two layers, the system layer and the Panda
layer (see Figure \ref{fPanda}).
The lowest layer is the system layer,
which 
can take full advantage of the features of the underlying operating
system. Above this layer is the Panda layer, which provides the
operations as described above.

To port Panda to a new architecture, only the system layer has to be
rewritten. The system layer is carefully designed to implement all
parts which are performance-critical: messages, threads and the underlying
network. This way these parts can all be implemented using fast,
low-level features of the specific operating system.

Panda run-time systems are currently implemented on Amoeba, a
T9000-based parallel machine, and the CM-5.

\section{Implementing the Turing Ring}
\label{sImpTur}

Implementing the Turing Ring in a serial language is
straightforward. The ring can be represented by an array of records,
each of which contains the prey and predator population for a single cell.
The program then loops over all cells in the ring, determining for
each individual whether it gives birth, dies, or
migrates. For migration a second array is used, to which the processed
animals are moved. This avoids processing twice the animals which
migrate to a cell which is not yet processed.

\begin{figure}
\begin{tgrind}
\File{sertur.pseu},{14:42},{Sep 11 1994}
\L{\LB{Read parameters}}
\L{\LB{Initialise}}
\L{\LB{\K{FOR} N iterations \K{DO}}}
\L{\LB{    Empty the other array}}
\L{\LB{    \K{FOR} all cells \K{DO}}}
\L{\LB{        \K{FOR} all predators\/prey \K{DO}}}
\L{\LB{            Determine birth and death}}
\L{\LB{            Migrate them to the other array}}
\L{\LB{        \K{OD}}}
\L{\LB{    \K{OD}}}
\L{\LB{    Swap the arrays}}
\L{\LB{\K{OD}}}

\end{tgrind}
\caption{Serial Implementation of the Turing Ring}
\label{fSerialTuring}
\end{figure}

\begin{figure}
\epsfxsize=\textwidth
\epsfbox{migration.eps}
\caption{Representation of the Turing Ring}
\label{fRepresentation}
\end{figure}

The Turing Ring was implemented in ANSI~C using this method.
A simple graphical interface was added to inspect the results.
Profiling this program revealed that 90\% of the
execution time was spent generating random numbers. 

One way to reduce this cost would be to increase the efficiency of the
way the random numbers are used. To determine if one predator or prey
gives birth or dies does not require a 32-bit random number. The
program could split the random number into parts, thus generating only
one random number for a few probabilistic decisions. This was not
implemented, because it was not in the scope of this work.

The ANSI~C program was then translated into strictly sequential Orca, and 
run on one
processor. This way the speed of Orca could be compared with that of
ANSI~C. It was found that the Orca program was about 2.5 times slower
than ANSI~C. This was mostly due to the dynamic array bound checking done
by the Orca run-time system.

\subsection {Parallelising the Turing Ring}

To parallelise the Turing Ring the problem must be {\em decomposed}.
There are two obvious compositions of the Turing Ring: a spatial
decomposition or a decomposition per animal.

A spatial decomposition would have each processor take care
of a number of cells. Since animals only move between adjacent cells, it
would furthermore be practical to have adjacent cells be processed by
one processor. This way a processor would only have to exchange information
regarding its cells with two other processors.

The decomposition per animal implies that every processor takes care of
a number of animals, which may be anywhere in the ring. This
decomposition seemed impractical, since a processor needs
information about a cell's population to determine the behaviour of a
single animal. If the population of a cell is scattered over all 
processors, this information is not readily available.
Therefore, the first approach was taken.

\begin{figure}
\epsfxsize=\textwidth
\epsfbox{distributing.eps}
\caption{Spatial Decomposition of the Turing Ring}
\label{fSpatialDecomposition}
\end{figure}

Each processor holds a proper
interval of the ring. Most of the calculations for animal migration can
then be
processed on the processor itself, as in the serial program. Only the
animals migrating from a processor's border cells need to be sent to
another processor.

This means that each processor needs to know which
cells are located on which processor. This can be accomplished by using
a shared array which contains this information. A simpler way is also
possible. If every processor contains a proper interval of the total
number of cells, and these intervals are given to the processors in
sequence, processor $i$ only needs to communicate with processor
$i - 1$ and processor $i + 1$ (and processor $1$ with processor $n$ and
vice versa).

The first solution has higher communication overhead,
but might be better in the general
case. In a two dimensional grid for example, load balancing would be much
harder using the latter approach, since the processors could not keep track
of where cells are located (see Section \ref{sPrevWork}).
In a ring, this problem does not occur. Therefore, the first solution was
chosen for this program.

\subsubsection{Implementation of the Turing Ring}

The parallel program consists of two modules. The main program is
contained in the module {\tt Turing}.
The communication channel between processors is implemented by an
object called {\tt LinkObj}.

\begin{figure}
\begin{center}
\begin{tabular}{|l|l|}
\hline
DisplayFreq: integer	& Display frequency \\
\hline
NOCells: integer	& Number of cells \\
\hline
NOIterations: integer	& Number of iterations \\
\hline
RandomSeed: integer	& Seed to initialise random numbers \\
\hline
TimeStep: real		& time step used \\
\hline
Explode: integer	& maximum number of animals in a cell \\
\hline
PredPopInit: integer	& initial predator population \\
\hline
PredDeathA: real	& \\
PredDeathB: real	& predator coefficients \\
PredBirth: real		& \\
PredMigrate: real	& \\
\hline
PreyPopInit: integer	& initial prey population \\
\hline
PreyDeathA: real	& \\
PreyDeathB: real	& prey coefficients \\
PreyBirth: real		& \\
PreyMigrate: real	& \\
\hline
\end{tabular}
\end{center}
\caption{The {\tt Params} data structure}
\label{fParams}
\end{figure}

The main process, {\tt OrcaMain}, initialises the work. First, it
reads the user-specified parameters from the standard input. (The
current Orca system has no support for command line parameters.) It
then forks off as many worker processes as there are CPUs available
and blocks until the worker processes have finished.

In the main process two arrays of link objects are declared. Each consecutive 
pair
of worker processes gets one link object of each array passed as a shared
parameter. These two links act as read and write channels between
neighbouring worker processes.

Each worker initially has a range of cells assigned to it for
processing. At the start, all cells are distributed evenly over 
all processors.
Each worker process also gets a copy of the parameters
the main process has read.

The worker process uses two arrays to store its part of the ring.
It initialises its part using the user provided parameters and then
starts iterating. On each iteration, it performs the following
steps:
\begin{itemize}
	\item[1.] Accept the animals which migrated from the neighbouring
	processors during the previous iteration.
	\item[2.] Display the current state of its part of the ring,
	if necessary.
	\item[3.] Generate birth and death in its part of the ring.
	\item[4.] Handle migration within its part of the ring, and
	determine which animals move to the neighbouring CPUs.
	\item[5.] Send migrating animals to its neighbours.
	\item[6.] Wait for the other workers to finish this iteration.
\end{itemize}
For inter-processor communication (step 1 and 5) the link object is
used. This is an object which contains operations for writing and
reading data which denote the number of predators and prey which
migrate between processors.

The synchronisation in step 6 is necessary to ensure that each
worker actually receives the migration data sent to it in step 5. Not
every worker will finish step 3 and 4 at the same time, since the
amount of work done in these steps is determined by the number of
animals which is presently in its part of the ring. Since this number
can vary by hundreds, some processors may have to wait.

After all iterations have been performed, the workers inform the main
process that they are done using barrier synchronisation. The main
process can then terminate.

\subsubsection{Load Balancing}

As mentioned above, a processor may have to wait for other processors
to finish their work. If the variance of execution time of an iteration
between processors becomes too large, some processors will do nothing
most of the time while others have lots of work to do. To prevent this
situation, a load balancing strategy has to be used. It should ensure
that the workload is evenly distributed amongst processors during every
iteration.

Initially, all cells in the ring contain the same number of animals,
so an even distribution of workload can be achieved by distributing
the cells evenly amongst processors at startup. This is called static
load balancing.

However, since populations can vary dramatically, some dynamic load
balancing has to be done as well. This means a processor must be able to
pass some of its work, i.e.\ some of its cells, to another, lightly
loaded, processor during run-time. Devising a good dynamic load balancing
strategy was a major part of this work.

\begin{figure}
\epsfxsize=\textwidth
\epsfbox{shifting.eps}
\caption{Shifting a cell to a neighbour}
\label{fShifting}
\end{figure}


It would be possible to have a master process collect all information and
then redistribute the work. However, as a consequence of the spatial
decomposition as depicted in Figure \ref{fSpatialDecomposition}, a
processor should only shift cells to its neighbours. Furthermore, the
populations (thus workload) of a processor's cells are not likely to
change rapidly. Therefore, a
processor can decide whether to shift cells to its neighbours based on
local data. This takes away the need for a central master process. In
addition, the channels for information exchange between neighbouring
processors have already been defined for migration. The load balancing
information can therefore be piggy-backed with this data.

A naive load balancing algorithm would do the following. On each
iteration, it would compare its own workload with that of its
neighbours. If the difference between the loads exceeded a certain
threshold, cells would be shifted to or from the neighbour (Figure
\ref{fShifting}).

This load balancing scheme suffers from two
problems. First, if most of the load of a heavily loaded 
processor comes from
one cell, (one cell contains almost all of the predators
and prey it has to handle,) and this cell happens to be a border cell,
shifting the cell would result in the neighbouring processor becoming
heavily loaded while the processor itself would lose most of its load. So
in the next iteration the neighbour will shift the cell back in an
attempt to redistribute the total workload more evenly. This way a
heavily loaded cell would be shifted back and forth until its population
has died out. This phenomenon is called {\em thrashing}.

Secondly, the following situation might occur. Imagine that a processor's
neighbours are both heavily loaded, and both decide to shift cells to
that processor. Then that processor would be heavily loaded, maybe even more
than its two neighbours were, so the workload would not be evenly
distributed. This situation is called {\em overloading}.

To avoid these two problems, a better load balancing scheme is needed.
Since processors need to know more about their neighbours's load, a
{\em multi-iteration protocol} is used. A processor not only receives
its neighbours's workload, but also an estimate of what its workload
would be after shifting. Now it decides whether shifting cells will
result in a more evenly distributed workload or not and acts
accordingly.  The new control structure is outlined in Figure
\ref{fWorker}.

\begin{figure}
\begin{tgrind}
\File{worker.pseu},{15:04},{Sep 11 1994}
\L{\LB{\K{BEGIN}}}
\L{\LB{    Initialise}}
\L{\LB{    \K{FOR} a number of iterations \K{DO}}}
\L{\LB{        Get animals moving in}}
\L{\LB{        Get the cells sent here for load balancing}}
\L{\LB{        Display}}
\L{\LB{        Process birth and death in every cell}}
\L{\LB{        Process migration between cells}}
\L{\LB{        Negotiate}}
\L{\LB{        Drop cells if we are overloaded}}
\L{\LB{        Wait for neighbours}}
\L{\LB{    \K{OD}}}
\L{\LB{\K{END}}}

\end{tgrind}
\caption{The Worker Process}
\label{fWorker}
\end{figure}

The algorithm is implemented using two functions,
{\tt InitLoadBal} and {\tt Negotiate}. It uses two {\em status records} to
keep information about the current phase of negotiation with the
neighbouring CPUs. These records are initialised when {\tt
InitLoadBal} is invoked.

Each worker process calls {\tt Negotiate} on every iteration. This
function communicates with the neighbours according to a three stage
protocol:
\begin{itemize}
        \item[1.] Send current workload to both neighbours, along
        with border cell's load.
        \item[2.] Receive neighbours's load information, send expected load.
        \item[3.] Receive neighbours's expected load.
\end{itemize}
In stage 1 all workers send their current load to their neighbours.
Since the workload is determined by the number of animals a processor
has to manage, it is actually this number that is sent. The worker also
sends the number of animals in the cell bordering the neighbour's cell.
This is the amount of load the worker would lose, if it decided to
shift a cell to the neighbour.

The load information is written to the link structures which connect the
neighbouring workers. It is not necessary to create another
communication channel if {\em piggy-backing} can be used. This however prevents
the load balancing algorithm from being implemented as an object, since
Orca does not allow shared variables as parameters to an
operation. (The link structures have to be passed to {\tt Negotiate}.)

In stage 2 every worker receives load balancing information from its
neighbours. It can then calculate the difference between its own load
and that of its neighbours. If the difference exceeds a certain threshold, it
decides to shift cells to or from the neighbour. Since the neighbour
makes the same calculations, it makes the same decision.
Now the worker calculates what its future load will be if all the shifts
actually take place, and sends this number to its neighbours.

In stage 3 each worker receives its neighbours's expected load, so it
knows what load it will have and what load its neighbours will have. It
now reconsiders the shifting of cells as it was planned in the previous
step.

If the worker wanted to send, but this would result in a situation where
its neighbour would get so much to do that the difference in load would
exceed the threshold again, there is no use in sending, so the worker
does not do it. This way thrashing can never occur: a cell which
contains so much load that the neighbour will want to send it back
immediately will never be sent. Also overloading is prevented, since a
worker takes into account what its neighbour's other neighbour will
send.

Another unwanted situation now can occur. Suppose a cell wants to send
to both neighbours. Its expected load will be its current load minus
the load of its two border cells. If this load is low compared to its
neighbours', it will refrain from sending in both direction, even if
the load distribution would be better if it would send to one of its
neighbours (Figure \ref{fSendOne}).

\begin{figure}
\epsfxsize=\textwidth
\epsfbox{sendone.eps}
\caption{A Worker which should send to one side}
\label{fSendOne}
\end{figure}

If CPU~5 would want to send cells to both CPU~4 and CPU~6, its future
load would be very low. But sending its border cell to CPU~6 would not
happen anyway, because CPU~6 would send it back immediately. Sending to
CPU~4 still seems a good idea, but if the future workload of CPU~5 is
compared to that of CPU~4, this will not happen either.

The following solution to this problem is used in the current
implementation. When a worker
finds out it planned to send to both its neighbours, but after
reconsidering it does not send to either of them, it determines if it
is possible to send to only one of them. In Figure \ref{fSendOne}, CPU~5
would decide it could send to CPU~4 after all, since the difference in
load after shifting would be very small. If a worker can send to
both neighbours, but not at the same time, it sends the border cell
containing the least workload.

Using this strategy, a worker which has a high workload and a heavily
loaded border cell will drop its load on the other side instead of
thrashing the heavily loaded cell back and forth. Only if
a worker has heavily loaded border cells on {\em both} sides would load
balancing be impossible.

\section{Experiences and Results}

The Turing Ring program described in Section \ref{sImpTur} was tested
using the Unix run-time system and run on the Amoeba system using
different numbers of processors. While the work was in progress, Panda
for Amoeba was still being developed. Therefore the program has not yet
been used with that system.

Programming in Orca proved to be relatively easy.
The syntax could be learned quickly because of the language's similarity
to Pascal and Modula-2.
A little more effort was required to parallelise the
problem using shared variables, but the model is clean and easy to
understand. The dynamic arrays used in Orca are a bit puzzling, but
come in very handy because of the lack of functions to allocate memory
at will (like C's {\tt malloc}).

Debugging the program sometimes was difficult, mainly because no
debugger programs exist. The only way to gain insight into the programs
behaviour was to write values to the standard output. But since
processors may work in parallel, output from different
processors can be mixed, which can be confusing.

\subsection{Re-designing and Tuning}
The early versions of the Turing Ring program included a simple
graphical interface to check the results. The graphical functions were
written in C and made available to Orca by a specification module.
Since the graphical module used a callback mechanism this was
difficult to use. A graphical module has been added to 
Orca's standard libraries now.

In the current program, the workers fill their own part
of the ring using the same parameters for each cell.
A more general way to fill the ring with data was considered, making
it possible to fill the ring with arbitrary data. This could either be
implemented by having the main process fill the ring with data
provided by the user, or have the worker process read in data some
way. The first solution would cause high startup costs, since the ring
would have to be shared with the workers. The second solution would be
difficult to implement cleanly, since many workers would have to read
data from one source.

A couple of small problems were encountered during implementation. One
had to do with memory allocation. If a worker process does not know how
many cells it will end up with (due to load balancing,) how much
memory should it allocate for it?
This was simply solved by having it allocate enough memory to store
the whole ring. The Amoeba processors currently have enough memory to
store thousands of cells, so that does not impose practical constraints on
the program.

The link object is the most frequently used shared object in the
program. During each iteration, the migrating animals are written to
the link and then read by the neighbouring CPU. Furthermore, the load
balancing negotiations write some data to the link. Initially the link
object contained one read operation and one write operation for each
data element it contains. However, some write operations (e.g.\
writing the migrating predators and writing the migrating prey) always
take place one after the other. As explained in Section
\ref{sOrcaImp}, every write operation requires a
point-to-point message and a broadcast. Combining the two write
operations into one operation thus reduces network traffic by a
factor of two. 

All write operations which always were executed one after the other
were combined. This reduced the execution time spent in communication
by 30\%. Further optimisation would be possible by writing single
operations for every set of operations which are frequently used together.
This however is difficult to implement cleanly, since it implies
operations are written which perform completely unrelated tasks.
Furthermore, the performance gain is system dependent.

Optimisation by combining operations could be done with read operations
as well, although this only yields speedup in case the run-time 
system decides not to replicate the link structures. Since this is not
the case on the Amoeba run-time system, this has not yet been implemented.

\subsection{Performance Results}

The program was run on the Amoeba processor pool, which consists of 80
SPARC~processors. Different problem sizes were used on different
numbers of processors; the number of iterations was kept the same (1000). 

\begin{figure}[p]
\epsfxsize=\textwidth
\epsfbox{serial.eps}
\caption{Execution time for 1000 iterations of the serial Orca
program}
\label{fSerial}
\end{figure}

Figure \ref{fSerial} shows the execution time for different problem
sizes of the serial Orca program discussed in Section
\ref{sImpTur}. 
The execution time is directly proportional to the number of
cells which have to be processed. This is reasonable since the amount
of calculation to be done depends on the number of cells in the ring,
provided the populations of the ring are more or less stable. For all
measurements, parameters were used which yielded such stable behaviour
most of the time. The parameters are shown in Figure \ref{fParamsUsed}.


\begin{figure}[p]
\begin{center}
\begin{tabular}{|l|l|}
	\hline
	TimeStep & 0.0001 \\
	\hline
	PredPop & 10 \\
	\hline
	PredDeathA & 1.3090 \\
	\hline
	PredDeathB & 0.250 \\
	\hline
	PredBirth & 0.220 \\
	\hline
	PredMigrate & 0.1 \\
	\hline
	PreyPop & 10 \\
	\hline
	PreyDeathA & 0.6545 \\
	\hline
	PreyDeathB & 0.130 \\
	\hline
	PreyBirth & 0.125 \\
	\hline
	PreyMigrate & 0.1 \\
	\hline
\end{tabular}
\end{center}
\caption{Parameters for the Turing Ring}
\label{fParamsUsed}
\end{figure}

\begin{figure}[p]
\epsfxsize=\textwidth
\epsfbox{extime.eps}
\caption{Execution time for 1000 iterations for different numbers of
processors}
\label{fExTime}
\end{figure}

The execution time of the parallel program using different numbers of
processors is shown in Figure \ref{fExTime}.
To test the consistency of the results the program was run 10
times on 10, 15 and 20 processors. This way the variance between runs
was determined. Figure \ref{fScatter} shows a scatter plot of these
runs. The standard deviation of the executions time is less than
$1/10$ of the mean.

\begin{figure}[p]
\epsfxsize = \textwidth
\epsfbox{scatter.eps}
\caption{Scatter plot using 10, 15 and 20 processors}
\label{fScatter}
\end{figure}

Note that for small problem sizes, adding processors adds to the
execution time. For larger problems, the execution time goes down till
a certain number of processors is reached. Then it goes up again.
To analyse these results, some definitions are introduced
here.

The {\em speedup} ${\cal S}$ achieved by running a program on more than one
processor is defined as follows:
\[
        {\cal S} = \frac{\tau_1}{\tau_{\cal P}}
\]
where $\tau_1$ is the time to run the program on one processor and
$\tau_{\cal P}$ the time to run it on ${\cal P}$ processors.

Perfect speedup is achieved if a program executes twice as fast when the
number of processors is doubled. Linear speedup is achieved if the
speedup is 
proportional to the number of processors, although the ratio is not
necessarily 1.

\begin{figure}[p]
\epsfxsize=\textwidth
\epsfbox{speedup.eps}
\caption{Speedup for different job sizes}
\label{fSpeedup}
\end{figure}

The speedup curves for different problem sizes are shown in Figure
\ref{fSpeedup}. $\tau_1$ is taken the execution time of the serial Orca program.
The parallel Orca program cannot be run on only one processor, since the
single link structure then is passed to the worker twice (aliasing),
which is not allowed in Orca.

The reason perfect speedup was not achieved, is that
some time is spent in communication between processors during each
iteration, which depends on the number of processors used.  Each
processor spends a fixed time in communication during each iteration, to
exchange migrating animals and load balancing information with its
neighbours. All processors want to use the network at nearly the same time
(if the workload is balanced well), but since an Ethernet is used
(which is a bus architecture) the communication is serialised. Adding
one processor thus adds to the communication time, which
reduces the speedup ratio. At a certain number of processors the
speedup curves drop, because the performance penalty due to
communication is then larger than the performance gained by better
parallelism. Note that the performance penalty for
communication depends only on the number of processors used: therefore
the speedup is closer to linear if larger problems are simulated.
The time spent
in calculating new generations on each processor then takes a larger
portion of the total time.

The {\em efficiency} of a program is defined as
\[
        \eta = \frac{\tau_1}{{\cal P}\tau_{\cal P}}
\]
where $\tau_1$ is the time to run on one
processor and $\tau_{\cal P}$ is the time to run on ${\cal P}$ processors
as before. It is a measure of hardware utilisation, as it compares the
effective time spent on calculating results and the overhead due
to communication. Figure \ref{fEfficiency} shows that the
efficiency of the program drops for a particular problem size as
processors are added. This is because of the sequentialisation of
communication: the processors use the Ethernet one at a time. However, if the
problem size becomes larger, for any number of processors the efficiency 
goes up.
This explains why the speedup curve for the largest problem size
measured goes up till 20 processors, while those for the smaller
problems start dropping for a smaller number of processors.
Figure \ref{fOptimal} shows the optimal number of processors for
different problem sizes.

\begin{figure}[p]
\epsfxsize=\textwidth
\epsfbox{effic.eps}
\caption{Efficiency for different problem sizes}
\label{fEfficiency}
\end{figure}

\begin{figure}[p]
\epsfxsize=\textwidth
\epsfbox{optimal.eps}
\caption{Optimal number of processors for different problem sizes}
\label{fOptimal}
\end{figure}

\begin{figure}[p]
\begin{center}
\begin{tabular}{|l|r|r|}
\hline
15 processors	& balanced load 	& unbalanced load \\
\hline
normal program	& 173			& 203 \\
\hline
no load balancing & 171			& 389 \\
\hline
\end{tabular}
\end{center}
\caption{Execution times for 1000 cells}
\label{fNoLoadBal}
\end{figure}

To see if the load balancing algorithm works well, the program was run
with 1000 cells on 15 processors, but
with a high initial load on one processor (predator and prey
population multiplied by 10 in every cell on that processor). The
results (Figure \ref{fNoLoadBal}) show that 
the execution time went up by 17\%, since it took many generations to
redistribute the load.

Then the load balancing functions were removed from the program and it
was tested with balanced and unbalanced input. With balanced input the
performance was the same as for the original program, so the load
balancing does not impose a large performance penalty. With unbalanced
input the execution time was more than doubled. The load balancing
strategy thus improves the performance of the program considerably.

\sectionLabel{Conclusions}

The main conclusion is that the Overlay Problem is hard to parallelize,
because of the bad ratio between communication and computation.
This make the Overlay Problem a killer application for parallel
programming systems.

The outcome of the experiments comparing the two algorithms for solving the
Overlay Problem is that Patchwised Overlay is better than 
Data-Parallel Overlay for the following reasons:
\begin{itemize}
  \item Patchwised Overlay is faster than Data-Parallel Overlay
  as the problem size increases, because sorting of the B map takes less time.
  In Patchwised Overlay the B map is sorted in parallel
  by the workers; each worker sorts a sub-map.
  In contrast, in Data-Parallel Overlay the B map is primarily
  sorted by the master; each worker only merges sorted parts of the
  B map.

  \item The workers in Patchwised Overlay use less memory than the
  workers in Data-Parallel Overlay, because the B map is partitioned in
  the former case, whereas it is replicated in the latter case. The
  difference in memory usage is especially important for large input
  maps.

  \item Performance tuning in Patchwised Overlay was easier
  than in Data-Parallel Overlay where it was hard to find a
  good solution for balancing the sorting load between Master
  and workers. Note, however, that
  the algorithm of Data-Parallel Overlay itself is less complex.
\end{itemize}

Writing the parallel Overlay programs in Orca was easy because of
the shared data objects, which supports communication and
synchronization in parallel programs at a high level. 
Performance tuning of the programs, however, was rather difficult since the
Orca system hides important low level details like communication
overhead, object distribution, and message fragmentation.

Orca programs perform well as was demonstrated by the comparison
between sequential solutions of the Overlay problem written both in
C and Orca; the observed difference between Orca and C was less than
3\%.


\bibliographystyle{plain}
\bibliography{refs}

\end{document}
