\section{The Turing Ring}
\label{sTuringRing}
In 1952, Alan Turing wrote a paper in which he formulated a model
for describing the interaction of two chemicals in a ring of cells
\cite{Tur52}. His
work was later generalised by others to include many different kinds of
reaction-migration systems and is now of importance in many branches
of science, such as ecology \cite{Smith91}, epidemiology
\cite{Moll77} and petroleum engineering \cite{Whee88}.

Turing's model can be described as a spatial system in
which reactants interact within locations and migrate between the
locations. It can thus represent populations of predators and prey in
a series of habitats.

\begin{figure}
\epsfxsize=\textwidth
\epsfbox{introduction.eps}
\caption{The Turing Ring}
\label{fTuringRing}
\end{figure}

Turing used the following differential equations to model the
interactions of reactants and the migration to neighbouring cells.  Let
$X_i$ and $Y_i$ (where $i = 1, \ldots , n$) be the concentration of
reactants (or population of predators/prey) in each cell $i$ in a
ring-shaped world of size $n$ and let $\mu$ and $\nu$ be the migration
rates between neighbouring cells for $X$ and $Y$ respectively. Then
\begin{eqnarray}
	\frac{dX_i}{dt} & = & f(X_i, Y_i) + \mu(X_{i-1} -
		2X_i + X_{i+1}) \label{migrX} \\
	\frac{dY_i}{dt} & = & g(X_i, Y_i) + \nu(Y_{i-1} -
		2Y_i + Y_{i+1}) \label{migrY}
\end{eqnarray}
where $f(X, Y)$ and $g(X, Y)$ can be expressed as:
\begin{eqnarray}
	f(X, Y) & = & X(r_X + a_XX + b_XY) \\
	g(X, Y) & = & Y(r_Y + a_YX + b_YY)
\end{eqnarray}
Note that $i$ is `wraparound':  location $1$ borders location $n$.  By
using different values for the coefficients
$r_{\{X, Y\}}$, $a_{\{X, Y\}}$ and $b_{\{X, Y\}}$
different systems can be obtained, which for example 
can be used to describe
Volterra prey/predator systems or to model the mixing of water and oil
in porous rock, and the resulting flow through rock structure.

In this work, a predator/prey system in a ring-shaped world is
described; the system can be visualised as a model for a series of
locations around a
lake, in which foxes and rabbits live. Using this image, the equations
can be explained as follows.

$X$ denotes the predator population and $Y$ the prey population. The
migration rate $\mu$ ($\nu$) determines how many predators (prey) from
neighbouring locations migrate to location $i$, and how many predators
(prey) move to the locations $i + 1$ and $i - 1$.

$f(X, Y)$ and $g(X, Y)$ are the generation functions of the predators
and prey. For $f(X, Y)$, $r_X$ is the constant death rate, which denotes
how many predators die of age. $a_X$ determines how many predators die
of overpopulation. $b_X$ determines the birth rate, which depends further on
the prey population (which represents the food situation).

In $g(X,Y)$, $r_Y$ is the constant death rate for prey, as for
predators above. $a_Y$ determines the number of prey killed by
predators. $b_Y$ determines the birth rate, which increases if the
number of prey increases.

To calculate the evolution of the system with a computer the
differential equations are discretised. Iterations
over the populations of the locations are performed in small time
steps (typically $10^{-4}$). The populations of
the locations are updated probabilistically, i.e.\ the system decides
randomly whether each predator or prey gives birth, dies or migrates. If
the time step chosen is small enough this linear system nears the
behaviour of the equations above.

\begin{figure}
\epsfxsize=\textwidth
\epsfbox{population.eps}
\caption{Typical behaviour for predator/prey systems}
\label{fTypical}
\end{figure}

The system described above is chaotic. An expected typical behaviour
is depicted in Figure \ref{fTypical}. However, a very small change in the
coefficients can cause a large difference in the outcome. Therefore it is
possible that the populations of cells differ by large numbers.
Furthermore, parameters for the system which yield sensible results are
very hard to find.

