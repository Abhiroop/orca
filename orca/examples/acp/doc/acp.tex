\documentstyle[a4, code, fullpage,11pt]{article}
\input{psfig}

% define fonts
%   phv = Helvetica
%   ptm = Times
\newfont{\fsectionfont}{phvb at 15pt}
\newfont{\ftitlefont}{phvb at 25pt}
\newfont{\fsubsectionfont}{phvr at 14pt}
\newfont{\fsubsubsectionfont}{phvr at 13pt}


% define new commands
%\newcommand{\fsection}[1]{\section {\fsectionfont #1}}
%\newcommand{\ftitle}[1]{\title {\ftitlefont #1}}
%\newcommand{\fsubsection}[1]{\subsection {\fsubsectionfont #1}}
%\newcommand{\fsubsubsection}[1]{\subsubsection {\fsubsubsectionfont #1}}

\newcommand{\fsection}[1]{\section { #1}}
\newcommand{\ftitle}[1]{\title { #1}}
\newcommand{\fsubsection}[1]{\subsection { #1}}
\newcommand{\fsubsubsection}[1]{\subsubsection { #1}}



%\newcommand{\inbox}[1]{\framebox[\textwidth][l]{ \parbox{\textwidth}{\verbatim #1 \endverbatim }}}





\title{Parallel Implementation of an Active Chart Parser in Orca}
\author{A. R. Sukul}
%\date

\begin{document}

\maketitle

\begin{abstract}
Active Chart Parsing is an efficient strategy used to
generate all possible parsings of a sentence given an ambiguous
grammar. This document describes several serial and parallel
implementations of an Active Chart
Parser. Timing results show that for this fine grained parallel application no
speedup can be achieved on a LAN of workstations. 
\end{abstract}


\section{Introduction}

{\em Active Chart Parsing} is a technique that is used to efficiently generate all
possible derivations of a sentence given an ambiguous grammar. For large
grammars and for large sentences this can be a very computationally
intensive task. This paper describes a parallel implementation of an Active
Chart Parser using the Orca language \cite {orca}.

This paper starts with a
short introduction to Active Chart Parsing. The next three sections
describe a serial ANSI C, a serial Orca and a parallel Orca implementation. In
section 6 we present some performance results. Finally, in section 7
we discuss our experiences with the Orca programming language.

\section{Active Chart Parsing}

Consider the sentence ``Jane saw Fred in the park with the green
telescope''. Depending upon how it it is parsed, it can have several
meanings:
\begin {itemize}
\item
Jane and Fred were both in the park and Jane was using a green telescope; or
\item
Jane was not in the park, but Fred was and Fred had a green telescope; or
\item
Jane saw Fred in the park in which a green telescope is located; or ...
\end{itemize}

The reason why we can parse this sentence in so many different ways is
that the grammar from which it is drawn (shown
in Figure~\ref{grammar}) is ambiguous. The purpose of the program is
to generate all possible derivations of such a sentence using a
given (ambiguous) grammar. 

\footnotesize
\begin{figure}
\begin{center} 
\begin{tabular}{l@{      $\rightarrow$      }l}
{\em Sentence} & {\em NounPhrase VerbPhrase} \\
{\em Sentence} & {\em VerbPhrase} \\
{\em NounPhrase} & {\em determiner NounPhrase2} \\
{\em NounPhrase} & {\em NounPhrase2} \\
{\em NounPhrase} & {\em NounPhrase PrepPhrase} \\
{\em NounPhrase2} & {\em Noun} \\
{\em NounPhrase2} & {\em Adjective NounPhrase2} \\
{\em PrepPhrase} & {\em NounPhrase} \\
{\em VerbPhrase} & {\em Verb} \\
{\em VerbPhrase} & {\em Verb NounPhrase} \\
{\em VerbPhrase} & {\em VerbPhrase PrepPhrase} \\ [3mm]

{\em Adjective} & {\sf green} \\
{\em Determiner} & {\sf the, to} \\
{\em Noun} & {\sf Jane, Fred, park, telescope } \\
{\em Preposition} & {\sf in, with} \\
{\em Verb} & {\sf saw}
\end{tabular}
\caption{Example ambiguous grammar}
\label{grammar}
\end{center}
\end{figure}
\normalsize

Solving this problem can be done in different ways \cite {terry},
of which the {\em bottom-up} and the {\em top-down} approach are the
best known. Each of these strategies has its own advantages and
disadvantages. 
In this section we will describe an algorithm called {\em active
chart parsing} which combines the advantages of
the strategies mentioned above. 

Basically the active chart parser follows a bottom-up strategy,
but differs from the usual bottom-up algorithms in that it saves
results computed in earlier stages of the algorithm. Thus, once the
algorithm knows a certain part of the sentence is a {\em NounPhrase},
this result is stored and can be used later on to determine
higher-order categories (e.g.\ a {\em sentence}). Furthermore, since
active chart parsing is 
a bottom-up strategy it avoids the backtracking overhead involved with
top-down strategies \cite {terry}.

The algorithm starts with inserting one node between each word and two extra
nodes respectively before the first and after the last word in the
sentence (Figure~\ref{example}). Note that this figure is not
complete (e.g\ the {\em Verb} arc between nodes 2 and 3 can be
expanded to a {\em VerbPhrase}, which leads to new arcs). In the initialization step
(level 1 in Figure~\ref{example}), the algorithm connects adjacent nodes with one
arc (or edge) for each of the categories to which the intervening word
belongs. In the next steps (levels 2-5) the algorithm tries to create
new categories from the existing edges in one of the following ways:
\begin{itemize}

\item  $Expanding$.
	If an edge of category $X$ connects nodes $n_{src}$ and $n_{dest}$ and
	the grammar contains the rule $Y$ {$\rightarrow$} $X$, then
	a new edge of category $Y$ may be inserted between nodes $n_{src}$
	and $n_{dest}$. An example of this rule is the expansion of a
	$Noun$ to a $NounPhrase2$.

\item	$Combining$.
	If a node $n$ has an incoming edge from node $n_{src}$ of
	category $X$ and an outgoing edge to node $n_{dest}$ of category
	$Y$ and the grammar contains the rule $Z$ {$ \rightarrow $} $X$
	$Y$, then a new edge of category $Z$ may be inserted between
	nodes $n_{src}$ and $n_{dest}$. An example of this rule is the
	combination of a $NounPhrase$ and a $VerbPhrase$ which can be
	combined to a $Sentence$.
\end{itemize}

This process continues until no further arcs can be added. Each time a
$Sentence$ arc spanning all nodes is found, the active chart
parser has found a new derivation of the given sentence.

\begin{figure}[tbh]
\centerline{
\psfig{figure=example.eps}
}
\caption{Example chart}
\label{example}
\end{figure}

\section{Serial implementation of ACP in ANSI C}

To implement the algorithm, typically the following data structures
are needed:

\begin{enumerate}

\item
A data structure which stores the reverse of the grammar shown in
Figure~\ref{grammar}. Instead of recording the rules by which a single
term may be expanded, this structure records the ways in which pairs
of terms might be generated. The reason for storing the grammar in
this order is to support the bottom-up strategy used by the active
chart parser. If there are $T$ terms in the grammar, a $T$ $\times$
$T$ matrix of lists of terms can be used. However, for large grammars,
this can be quite expensive in memory usage, so a list of lists or some
other irregular  structure is preferred, as it requires less memory
while increasing access time slightly.

\item
A data structure for the dictionary. This is stored as a list for each
terminal which shows the terms to which that terminal belongs.

\item
A data structure for the chart. The chart is represented as a list of
nodes, with each node containing information on incoming and outgoing
arcs (see below).

\item
Information for each arc, including:

	\begin {itemize}
	
	\item	The source and destination node of the arc.
	\item	The name of the terminal or non-terminal the arc
		represents.
	\item	The constituent edges. Once a sentence has been found,
		we need information on how the parse-tree is build in
		order to print that parse tree. Therefore each arc should contain
		information on the arcs it combines. 
	\end {itemize}

\item
A data structure for storing arcs which have
been generated by rule firings, but have not been added to the chart
yet. This may be managed as a stack, a queue or some form of priority
queue if the program is using heuristics to guide parsing.

\end {enumerate}

Most of the data structures above can be translated directly into C
data structures. The Pending Edges Table (the fifth data structure)
was implemented as a queue 
(hereafter called the {\em PEQ}).
The most notable data structure  in the C-implementation is the arc
data structure which is defined as shown in Figure~\ref{cstruct}.

\footnotesize
\begin{figure}
\begin{center}
\framebox{\parbox[t]{4.in} {
\hspace*{.10in}typedef struct edge\_struct{ \\
\hspace*{.25in}int \hspace{.80in} from\_node; \hspace{.10in} /* source node */  \\
\hspace*{.25in}int \hspace{.80in} to\_node; \hspace{.25in} /* destination node */ \\
\hspace*{.25in}int \hspace{.80in} symbol\_name; /* label of edge */ \\
\hspace*{.25in}struct edge\_struct *comp\_1; \hspace{.25in} /* first constituting component */ \\
\hspace*{.25in}struct edge\_struct *comp\_2; \hspace{.25in} /* second component */ \\
\hspace*{.10in}} Edge\_t;
}}
\end{center}
\caption{Arc data structure in ANSI-C}
\label{cstruct}
\end{figure}
\normalsize


Looking closely we see that this data structure is in fact a
representation of a tree, in our case the parse tree of the
derivation. The usage of the data structures is illustrated in
Figure~\ref{datas}. In this figure the contents of the incoming and
outgoing edge lists of each of the nodes is given after all the
possible edges for the sentence ``Jane saw'' have been added to the
chart . Note that the incoming edgelist of 
the first node and the outgoing edgelist of the last node will always
be empty. For the sake of simplicity we have represented the arcs as triplets 
\{ {\em name, src\_node, dest\_node} \} and have not included
information on their constituent edges. 

\begin{figure}[tbh]
\centerline{
\psfig{figure=datas1.eps}
}
\caption{Illustration of the data structures}
\label{datas}
\end{figure}

Given the data structures described above, the algorithm for a serial
implementation of the ACP is given in Figure~\ref{mainloop}
in pseudo code.

\footnotesize
\begin{figure}
\begin{center}
\framebox{\parbox[t]{5.in} {
\hspace*{.10in}    read sentence \\
\hspace*{.10in}    {\bf for} each word W {\bf do} \\
\hspace*{.25in}add all categories to which W belongs to the PEQ \\
\hspace*{.10in}    {\bf od} \\ [1mm]

\hspace*{.10in}    {\bf while} PEQ not empty {\bf do} \\
\hspace*{.25in}{\em elt} = dequeue(PEQ) \\
\hspace*{.25in}{\bf if }({\em elt} is a sentence) {\bf then} print\_sentence {\bf fi} \\
\hspace*{.25in}add {\em elt} to the chart \\
\hspace*{.25in}expand {\em elt} by applying the expansion rule. Add new arcs to PEQ. \\
\hspace*{.25in}combine {\em elt} with left and right neighbours of its \\
\hspace*{.25in}source and destination node. Add new arcs to PEQ. \\
\hspace*{.10in}{\bf od}
}}
\end{center}
\caption{The Serial ACP algorithm}
\label{mainloop}
\end{figure}
\normalsize

\section{Serial Implementation of ACP in Orca}

Most of the data structures used in the serial C version can be
translated directly into their Orca equivalents. The dynamic
structures, including the PEQ, are implemented using
Orca {\em graphs} \cite{orca}. However, one problem arises when trying to
translate the {\em arc} data structure (see section 3.1) to Orca. One
of the main problems of Orca is that it has no direct equivalent of
pointers. In particular, there is no such thing as a dereferencing
mechanism to access information associated with a pointer. In Orca
``pointers'' are implemented as elements (called {\em nodes}) of the most general
structure found in Orca, the graph. Information on a node can be accessed by
using the graph to which that node belongs, using a construct
analogous to array element selection \cite {orca}.

To implement pointers we must therefore introduce an object of type graph
and implement functions for creating nodes, assigning values to
nodes, and retrieving information from nodes. The Orca implementation
of the arc data structure is given in Figure~\ref{orcastruct}.

\subsection{Comparing the Serial-Orca ACP and the Serial-C ACP}

When first translating the serial C version into serial Orca,
one is inclined to use software engineering principles, like
the use of ADTs and information hiding. In Orca these principles are
implemented by means of {\em objects}. Each object has its private data
which can only be accessed by means of {\em operations} defined on
that object. As elegant as this may be, in the current
implementation of Orca objects in Orca are slow due to an enormous amount of
copying being done. Therefore, the serial Orca version of ACP was implemented
without the use of Orca objects.

Since most constructs used in C were directly implemented in Orca,
there was no significant difference in the number of lines of code.
The serial Orca version was about 5 times slower than the serial C
version, which can be explained by run-time system overhead and the heavy use
of the graph construct to implement the many dynamic data structures.

\footnotesize
\begin{figure}
\begin{center}
\framebox{\parbox[t]{4.in} {
\hspace*{.10in}TYPE edge\_p = \\
\hspace*{.25in}	       NODENAME OF edge\_t; \\ [1mm] \\

\hspace*{.10in}TYPE edge\_info\_t = \\
\hspace*{.25in}        RECORD \\
\hspace*{.35in}                from\_node :     integer; \hspace{.10in} \# source node \\
\hspace*{.35in}                to\_node :       integer; \hspace{.30in}\# destination node \\
\hspace*{.35in}                symbol\_name :   integer; \# label of the edge \\
\hspace*{.35in}                comp\_1 :        edge\_p; \hspace{.35in}\# left component \\
\hspace*{.35in}                comp\_2 :        edge\_p; \hspace{.35in}\# right component \\
\hspace*{.25in}        END; \\ [1mm] \\

\hspace*{.10in}TYPE edge\_t =   \\
\hspace*{.25in}        GRAPH \\
\hspace*{.25in}        NODES \\
\hspace*{.35in}                data : edge\_info\_t; \\
\hspace*{.25in}        END; \\ [1mm]
}}
\end{center}
\caption{Arc data structure in Orca}
\label{orcastruct}
\end{figure}
\normalsize

\section{Parallel implementation of ACP in Orca}

Parallelization of the Active Chart Parser is done by creating one
worker process for each node (i.e. word) and allocating each worker to
one processor (in a pure parallel version) or allocating several
workers to one processor (in a multi-tasking version). Note that in the
parallel version there will be as many workers as there are words, since
the last node does not represent any word. Each worker keeps track of
its incoming and outgoing edges.

In the parallel version, each worker executes the algorithm shown
in Figure~\ref{parloop}. After the local initializations have finished,
each worker adds one edge to its PEQ for each category to which
the word it handles belongs. Since edges connect {\em two} nodes,
the destination node (i.e. worker) should somehow also be able to add
the new edge to its PEQ. In a message passing system
this would be done by explicitly sending the information to the
destination worker. Since communication in Orca is implemented by
means of {\em shared objects} \cite {orca}, the most obvious way to
implement the communication is to pass a ``mailbox'' structure to all
workers as a shared parameter. The mailbox is implemented as an array
of PEQs (one for each worker) and is passed as a (shared)
parameter to all workers (the third parameter in
Figure~\ref{parloop}). Each time a higher-order edge is generated by a
worker, it updates the PEQs of the source and
destination nodes by means of the mailbox. Note that in the parallel
version of ACP, the PEQ has to be an object, since only objects may be
shared between processes.

The {\em EdgeCollection} object implements the {\em graph} from which edge
pointers can be created, queried and assigned, as discussed in section
4. 

\footnotesize
\begin{figure}
\begin{center}
\framebox{\parbox[t]{5.in} {
{\bf PROCESS} worker( \\
\hspace*{.10in}W: word\_t; \hspace{1.05in} \# the word this node represents\\
\hspace*{.10in}node\_nr  : integer; \hspace{.70in} \# the node number \\
\hspace*{.10in}M: SHARED mailbox\_t; \hspace{.30in} \# mailbox of pending edges queues\\
\hspace*{.10in}EC: SHARED EdgeCollection; \# the collection of edges \\
\hspace*{.10in}work: SHARED Work; \hspace{.40in} \# information on termination \\
\hspace*{.10in}...\\
); \\ [3mm]
\hspace*{.10in}do local initializations \\
\hspace*{.10in}add all categories to which W belongs to M[node\_nr] and M[node\_nr + 1] \\ [1mm]
\hspace*{.10in}{\bf while} M[node\_nr] not empty {\bf do} \\
\hspace*{.25in}elt = dequeue(M[node\_nr]) \\
\hspace*{.25in}{\bf if} (elt is a sentence) {\bf then} print\_sentence {\bf fi} \\
\hspace*{.25in}add elt to local incoming/outgoing edges \\
\hspace*{.25in}add new edges that can be created by the addition of elt to local list \\
\hspace*{.10in}{\bf od} \\
{\bf END}
}}
\end{center}
\caption{The Parallel ACP algorithm}
\label{parloop}
\end{figure}
\normalsize

Special care has to be taken when the {\em expansion} rule is applied
on an edge in the parallel version. Say that edge $Y$ connecting nodes
$n_{src}$ and $n_{dest}$ is expanded by rule $X$ {$\rightarrow$} $Y$ and
that edge $X$ is expanded by rule $Z$ {$\rightarrow$} $X$, then both
nodes $n_{src}$ and $n_{dest}$ would contain double entries of edge
$Z$, since both nodes will (eventually) expand edge $X$ in the same
way. To prevent this situation, workers will only expand {\em
outgoing} edges.


\subsection{Termination Detection}

In both of the serial versions of the ACP, the algorithm could safely
terminate once there were no more edges in the PEQ,
since there was only one process dealing with this data structure. 
In the parallel version, the algorithm stops when all workers are
done. However, it is not trivial for a worker $W$ to determine when to
stop. The fact that its PEQ is empty is no reason to
stop, since other workers may still be busy, which can possibly lead
to new edges being added to the PEQ of $W$. This
problem is analogous to termination detection in the Arc
Consistency Problem \cite {acp}. Termination of all workers (and
therefore the algorithm) can safely be done only when both of the following
conditions are met:

\begin{enumerate}

\item
The number of active workers is 0.

\item
No other worker has any edge in its PEQ (which
results in new edges being added).

\end{enumerate}

The information on the number of active workers and contents of the
PEQs is stored in an object shared by all the workers
(the fifth parameter in Figure~\ref{parloop}). The number of active
workers is maintained as an integer and is initialized to the number
of words in the sentence. The information on the contents of the
PEQs is maintained as an array of booleans, one
element for each PEQ. Each time a worker 
adds a new edge to the PEQ of another worker, the
corresponding field in the array of the {\em Work} object is set to
{\em true}. Each time a worker, representing node $N$ enters the while loop in
Figure~\ref{parloop} it sets the $N^{th}$ field of the array to {\em
false} (not shown in Figure~\ref{parloop}).

Whenever the worker exits the while loop in Figure~\ref{parloop}, it
decreases the number of active workers by calling an operation from
the {\em Work} object. Then it calls an operation of the {\em Work}
object which checks for the termination condition. If the termination
condition is false, the number of active workers is incremented and
the worker process re-enters the while loop to see if new work has to be done.
If the termination condition is true, the worker process (and all other
workers) can safely terminate. 

\subsection{Optimizations}

There are two optimizations which can be included in the parallel
version of ACP in order to reduce communication. It
should be noted that the EdgeCollection object and the Work object
have a major impact on the number of messages sent. Therefore
optimizations can be found by adapting these objects.

First of all the EdgeCollection implements a shared address space
between all processors. Although this approach is the most natural
way to implement the storage of the parse tree (see Section 3), it
will be slow, since there will be a lot of communication involved.
Furthermore for a large number of processes, this object can become a
bottleneck. What we would like is to implement ``local pointers'',
i.e. let each node have its own address space with respect to its
``own'' (outgoing) edges. Thus edges can now be identified by a
$<${\em node\_nr, index}$>$-pair, where {\em index} is an index in the
outgoing list of node {\em node\_nr}. Thus, instead of storing
``pointers'' to constituent edges (see Figure~\ref{orcastruct}), we
use the tuples as defined above to identify these edges.

There is one problem in the parallel version when
trying to implement this strategy: since an edge is added to both the
PEQs of its source and destination node, we would like to know in
advance its index in the outgoing edgelist of its source node.
However, since an edge is first added to the {\em PEQ} of the source
node, it is impossible to know in advance (or without any communication)
what this index will (eventually) be. In order to still implement this
strategy, the following scheme is used: each node $N$ maintains an
array {\em o\_cnt} of integers, one for each node. {\em o\_cnt[i]}
indicates the number of edges that $N$ has added (i.e. sent) to node
$N_i$.  The triplet $<$ {\em src\_node, N, N[src\_node]} $>$ then
uniquely identifies
an edge added by node $N$ to node {\em src\_node}. At the end of the
algorithm, all nodes store their outgoing edges in an object storing
this information for all nodes. By scanning through
the outgoing edge list of the first node, one can find all sentence arcs
and print the corresponding parse tree using the
identification mechanism described above. Although this scheme is
slightly more expensive in memory usage, the advantage of less
communication is expected to be significant. 

The next optimization that can be made is the adaption of
termination detection control. As explained in section 5.1, a 
{\em Work} object was introduced that kept information on
termination (an array of booleans indicating which PEQs were
still non-empty and an integer indicating the number of active
workers). 
Each time the worker entered its main loop and each time a worker announced a
new edge to the source and destination node of that new edge, the Work
object (and in particular the boolean array) had to be updated.

The following strategy is expected to decrease the communication
involved in updating the boolean array: let each node $N$ maintain an
array {\em work\_arr} of integers, one for each other node. 
{\em work\_arr[i]} denotes the number of pending edges that have been
added (or extracted) to node $i$ by $N$ during its main loop. After its main loop, each
worker sends the {\em work\_arr} to the {\em Work} object which keeps
track of the number of pending edges for each of the PEQs (i.e. also
an array of integers, one for each node). The {\em Work} object then
adds each of the elements in the {\em work\_arr} array to the
corresponding entries in its array. The second part of the termination
condition (see section 5.2) then becomes:

\begin{itemize}

\item
All entries in the integer array maintained by the {\em Work} object
should be 0 (i.e. there are no outstanding pending edges)

\end{itemize}

With this implementation, we do not have to send a message to the {\em Work}
object each time a PEQ is updated. Instead all the updates on the
PEQs are stored locally and send to the {\em Work} object after the
worker exits its main loop.

\section{Test results}

\subsection{Discussion}

Before starting to give the test results, we will begin
with a short discussion on the parallelization of the ACP. First of
all it should be noted that the ACP is very fine grained. That is,
there is a lot of communication,  
in comparison with the number of computations performed. We
therefore do not expect to have any speedup when running
the parallel version of ACP on top of Amoeba. Besides the fact that a
large number of edges have to be transmitted between nodes, the shared
address space (the {\em EdgeCollection} object) in the non-optimized
version also contributes to
the amount of communication; each time the EdgeCollection object is
changed, all possible replicas 
of the object must be uniformly updated. This means that a broadcast has to
be done in order to keep all copies of the object (when replicated)
consistent. One can influence the amount of communication by {\em
pinning} the {\em EdgeCollection} object to one processor, which means
that only one master copy of the object has to be kept up to date.
However, this strategy (which can be obtained by using the Orca {\em
Strategy()} library call) has one major disadvantage: it can become
a potential bottleneck since many processes are
simultaneously trying to access the same object.

The use of the {\em Strategy()}-call can also be used to pin each
pending edges queue $P$ (i.e. the $P^{th}$ element of the mailbox
array) to the processor which handles node $P$ in the sentence. It is
expected that this strategy will also reduce the amount of communication.

As a final remark, one has to note that the parallel ACP has a fixed
number of processes, since the number of words in the sentence
determines the number of workers and therefore the number of
processes. ``Speedup'' in our case will have to be measured by {\em
multitasking}, i.e. by allocating several workers to one processor.


\subsection{Timing Results}

For testing the parallel version of ACP, we used a sentence containing
29 words, resulting in 29 processes in total. An extended version of
the grammar shown in Figure~\ref{grammar} was used. Using this
extended grammar, a total of 6760 derivations could be found.
The serial C version on top of UNIX (running on a SPARC), the 6760
derivations were found in 5.4 seconds. The serial Orca
version on top of UNIX, using the {\em oc\_unix\_proc} Orca run-time
system needed 26.9 seconds. 

The figures for the parallel version of ACP are depicted in
Figure~\ref{results}. The tests were executed on top of Amoeba
\cite{amoeba1}. Two versions of ACP were tested, 
one with the EdgeCollection object and one without. Both versions
implemented the revised termination detection algorithm. The strategy call
was {\em not} used, since this caused a segmentation fault on top of
Amoeba when multitasking.

As can be seen and as explained in the previous section, the parallel
version is much slower than the serial Orca
version. There are a number of points one should keep in mind when
looking at the graph in Figure~\ref{results}. First of all, one should note that the
ACP implicitly has a fixed number of processes. Since processes are
assigned round-robin to the processors, there will be a lot of load
imbalance which is reflected in the timing results. Furthermore,
increasing the number of processors implies an increase in the number
of messages to be sent and therefore an increase in the time needed
by the parser.

Finally, to illustrate the fact that assigning each mailbox entry to
its corresponding processor (using the {\em Strategy()}-call) will
reduce communication: the time needed by the versions with and
without the EdgeCollection using 29 workers are 238.958 and 152.534 seconds. 

\begin{figure}
\begin{center}
% GNUPLOT: LaTeX picture
\setlength{\unitlength}{0.240900pt}
\ifx\plotpoint\undefined\newsavebox{\plotpoint}\fi
\begin{picture}(1349,900)(0,0)
\font\gnuplot=cmr10 at 10pt
\gnuplot
\sbox{\plotpoint}{\rule[-0.200pt]{0.400pt}{0.400pt}}%
\put(220.0,113.0){\rule[-0.200pt]{4.818pt}{0.400pt}}
\put(198,113){\makebox(0,0)[r]{400}}
\put(1265.0,113.0){\rule[-0.200pt]{4.818pt}{0.400pt}}
\put(220.0,193.0){\rule[-0.200pt]{4.818pt}{0.400pt}}
\put(198,193){\makebox(0,0)[r]{500}}
\put(1265.0,193.0){\rule[-0.200pt]{4.818pt}{0.400pt}}
\put(220.0,273.0){\rule[-0.200pt]{4.818pt}{0.400pt}}
\put(198,273){\makebox(0,0)[r]{600}}
\put(1265.0,273.0){\rule[-0.200pt]{4.818pt}{0.400pt}}
\put(220.0,353.0){\rule[-0.200pt]{4.818pt}{0.400pt}}
\put(198,353){\makebox(0,0)[r]{700}}
\put(1265.0,353.0){\rule[-0.200pt]{4.818pt}{0.400pt}}
\put(220.0,433.0){\rule[-0.200pt]{4.818pt}{0.400pt}}
\put(198,433){\makebox(0,0)[r]{800}}
\put(1265.0,433.0){\rule[-0.200pt]{4.818pt}{0.400pt}}
\put(220.0,512.0){\rule[-0.200pt]{4.818pt}{0.400pt}}
\put(198,512){\makebox(0,0)[r]{900}}
\put(1265.0,512.0){\rule[-0.200pt]{4.818pt}{0.400pt}}
\put(220.0,592.0){\rule[-0.200pt]{4.818pt}{0.400pt}}
\put(198,592){\makebox(0,0)[r]{1000}}
\put(1265.0,592.0){\rule[-0.200pt]{4.818pt}{0.400pt}}
\put(220.0,672.0){\rule[-0.200pt]{4.818pt}{0.400pt}}
\put(198,672){\makebox(0,0)[r]{1100}}
\put(1265.0,672.0){\rule[-0.200pt]{4.818pt}{0.400pt}}
\put(220.0,752.0){\rule[-0.200pt]{4.818pt}{0.400pt}}
\put(198,752){\makebox(0,0)[r]{1200}}
\put(1265.0,752.0){\rule[-0.200pt]{4.818pt}{0.400pt}}
\put(220.0,832.0){\rule[-0.200pt]{4.818pt}{0.400pt}}
\put(198,832){\makebox(0,0)[r]{1300}}
\put(1265.0,832.0){\rule[-0.200pt]{4.818pt}{0.400pt}}
\put(345.0,113.0){\rule[-0.200pt]{0.400pt}{4.818pt}}
\put(345,68){\makebox(0,0){5}}
\put(345.0,812.0){\rule[-0.200pt]{0.400pt}{4.818pt}}
\put(502.0,113.0){\rule[-0.200pt]{0.400pt}{4.818pt}}
\put(502,68){\makebox(0,0){10}}
\put(502.0,812.0){\rule[-0.200pt]{0.400pt}{4.818pt}}
\put(659.0,113.0){\rule[-0.200pt]{0.400pt}{4.818pt}}
\put(659,68){\makebox(0,0){15}}
\put(659.0,812.0){\rule[-0.200pt]{0.400pt}{4.818pt}}
\put(815.0,113.0){\rule[-0.200pt]{0.400pt}{4.818pt}}
\put(815,68){\makebox(0,0){20}}
\put(815.0,812.0){\rule[-0.200pt]{0.400pt}{4.818pt}}
\put(972.0,113.0){\rule[-0.200pt]{0.400pt}{4.818pt}}
\put(972,68){\makebox(0,0){25}}
\put(972.0,812.0){\rule[-0.200pt]{0.400pt}{4.818pt}}
\put(1128.0,113.0){\rule[-0.200pt]{0.400pt}{4.818pt}}
\put(1128,68){\makebox(0,0){30}}
\put(1128.0,812.0){\rule[-0.200pt]{0.400pt}{4.818pt}}
\put(1285.0,113.0){\rule[-0.200pt]{0.400pt}{4.818pt}}
\put(1285,68){\makebox(0,0){35}}
\put(1285.0,812.0){\rule[-0.200pt]{0.400pt}{4.818pt}}
\put(220.0,113.0){\rule[-0.200pt]{256.558pt}{0.400pt}}
\put(1285.0,113.0){\rule[-0.200pt]{0.400pt}{173.207pt}}
\put(220.0,832.0){\rule[-0.200pt]{256.558pt}{0.400pt}}
\put(45,472){\makebox(0,0){Number of seconds}}
\put(752,23){\makebox(0,0){Number of processors}}
\put(752,877){\makebox(0,0){Timing results}}
\put(220.0,113.0){\rule[-0.200pt]{0.400pt}{173.207pt}}
\put(721,752){\makebox(0,0)[r]{Without EdgeCollection }}
\put(743.0,752.0){\rule[-0.200pt]{15.899pt}{0.400pt}}
\put(283,458){\usebox{\plotpoint}}
\multiput(283.58,454.32)(0.497,-0.988){59}{\rule{0.120pt}{0.887pt}}
\multiput(282.17,456.16)(31.000,-59.159){2}{\rule{0.400pt}{0.444pt}}
\multiput(314.58,391.44)(0.497,-1.559){59}{\rule{0.120pt}{1.339pt}}
\multiput(313.17,394.22)(31.000,-93.221){2}{\rule{0.400pt}{0.669pt}}
\multiput(345.00,301.59)(1.834,0.489){15}{\rule{1.522pt}{0.118pt}}
\multiput(345.00,300.17)(28.841,9.000){2}{\rule{0.761pt}{0.400pt}}
\multiput(377.00,308.92)(0.675,-0.496){43}{\rule{0.639pt}{0.120pt}}
\multiput(377.00,309.17)(29.673,-23.000){2}{\rule{0.320pt}{0.400pt}}
\multiput(408.58,284.17)(0.497,-0.727){59}{\rule{0.120pt}{0.681pt}}
\multiput(407.17,285.59)(31.000,-43.587){2}{\rule{0.400pt}{0.340pt}}
\multiput(439.58,242.00)(0.497,0.673){61}{\rule{0.120pt}{0.637pt}}
\multiput(438.17,242.00)(32.000,41.677){2}{\rule{0.400pt}{0.319pt}}
\multiput(471.58,285.00)(0.497,0.858){59}{\rule{0.120pt}{0.784pt}}
\multiput(470.17,285.00)(31.000,51.373){2}{\rule{0.400pt}{0.392pt}}
\multiput(502.58,332.66)(0.497,-1.494){59}{\rule{0.120pt}{1.287pt}}
\multiput(501.17,335.33)(31.000,-89.329){2}{\rule{0.400pt}{0.644pt}}
\multiput(533.00,246.58)(0.515,0.497){59}{\rule{0.513pt}{0.120pt}}
\multiput(533.00,245.17)(30.935,31.000){2}{\rule{0.256pt}{0.400pt}}
\multiput(565.58,277.00)(0.497,0.695){59}{\rule{0.120pt}{0.655pt}}
\multiput(564.17,277.00)(31.000,41.641){2}{\rule{0.400pt}{0.327pt}}
\multiput(596.58,320.00)(0.497,0.515){59}{\rule{0.120pt}{0.513pt}}
\multiput(595.17,320.00)(31.000,30.935){2}{\rule{0.400pt}{0.256pt}}
\multiput(627.58,342.35)(0.497,-2.806){61}{\rule{0.120pt}{2.325pt}}
\multiput(626.17,347.17)(32.000,-173.174){2}{\rule{0.400pt}{1.163pt}}
\multiput(659.00,174.58)(0.866,0.495){33}{\rule{0.789pt}{0.119pt}}
\multiput(659.00,173.17)(29.363,18.000){2}{\rule{0.394pt}{0.400pt}}
\multiput(690.58,192.00)(0.497,0.939){59}{\rule{0.120pt}{0.848pt}}
\multiput(689.17,192.00)(31.000,56.239){2}{\rule{0.400pt}{0.424pt}}
\multiput(721.00,248.92)(1.642,-0.491){17}{\rule{1.380pt}{0.118pt}}
\multiput(721.00,249.17)(29.136,-10.000){2}{\rule{0.690pt}{0.400pt}}
\multiput(753.58,234.50)(0.497,-1.543){59}{\rule{0.120pt}{1.326pt}}
\multiput(752.17,237.25)(31.000,-92.248){2}{\rule{0.400pt}{0.663pt}}
\multiput(784.58,145.00)(0.497,2.147){59}{\rule{0.120pt}{1.803pt}}
\multiput(783.17,145.00)(31.000,128.257){2}{\rule{0.400pt}{0.902pt}}
\multiput(815.58,277.00)(0.497,1.331){59}{\rule{0.120pt}{1.158pt}}
\multiput(814.17,277.00)(31.000,79.596){2}{\rule{0.400pt}{0.579pt}}
\multiput(846.58,356.35)(0.497,-0.673){61}{\rule{0.120pt}{0.637pt}}
\multiput(845.17,357.68)(32.000,-41.677){2}{\rule{0.400pt}{0.319pt}}
\multiput(878.58,316.00)(0.497,0.613){59}{\rule{0.120pt}{0.590pt}}
\multiput(877.17,316.00)(31.000,36.775){2}{\rule{0.400pt}{0.295pt}}
\multiput(909.58,354.00)(0.497,0.776){59}{\rule{0.120pt}{0.719pt}}
\multiput(908.17,354.00)(31.000,46.507){2}{\rule{0.400pt}{0.360pt}}
\multiput(940.58,402.00)(0.497,0.594){61}{\rule{0.120pt}{0.575pt}}
\multiput(939.17,402.00)(32.000,36.807){2}{\rule{0.400pt}{0.288pt}}
\multiput(972.00,440.59)(3.382,0.477){7}{\rule{2.580pt}{0.115pt}}
\multiput(972.00,439.17)(25.645,5.000){2}{\rule{1.290pt}{0.400pt}}
\multiput(1003.00,445.58)(0.706,0.496){41}{\rule{0.664pt}{0.120pt}}
\multiput(1003.00,444.17)(29.623,22.000){2}{\rule{0.332pt}{0.400pt}}
\multiput(1034.58,467.00)(0.497,1.115){61}{\rule{0.120pt}{0.988pt}}
\multiput(1033.17,467.00)(32.000,68.950){2}{\rule{0.400pt}{0.494pt}}
\multiput(1066.58,519.59)(0.497,-5.476){59}{\rule{0.120pt}{4.435pt}}
\multiput(1065.17,528.79)(31.000,-326.794){2}{\rule{0.400pt}{2.218pt}}
\multiput(1097.00,202.58)(0.820,0.495){35}{\rule{0.753pt}{0.119pt}}
\multiput(1097.00,201.17)(29.438,19.000){2}{\rule{0.376pt}{0.400pt}}
\put(765,752){\raisebox{-.8pt}{\makebox(0,0){$\Diamond$}}}
\put(283,458){\raisebox{-.8pt}{\makebox(0,0){$\Diamond$}}}
\put(314,397){\raisebox{-.8pt}{\makebox(0,0){$\Diamond$}}}
\put(345,301){\raisebox{-.8pt}{\makebox(0,0){$\Diamond$}}}
\put(377,310){\raisebox{-.8pt}{\makebox(0,0){$\Diamond$}}}
\put(408,287){\raisebox{-.8pt}{\makebox(0,0){$\Diamond$}}}
\put(439,242){\raisebox{-.8pt}{\makebox(0,0){$\Diamond$}}}
\put(471,285){\raisebox{-.8pt}{\makebox(0,0){$\Diamond$}}}
\put(502,338){\raisebox{-.8pt}{\makebox(0,0){$\Diamond$}}}
\put(533,246){\raisebox{-.8pt}{\makebox(0,0){$\Diamond$}}}
\put(565,277){\raisebox{-.8pt}{\makebox(0,0){$\Diamond$}}}
\put(596,320){\raisebox{-.8pt}{\makebox(0,0){$\Diamond$}}}
\put(627,352){\raisebox{-.8pt}{\makebox(0,0){$\Diamond$}}}
\put(659,174){\raisebox{-.8pt}{\makebox(0,0){$\Diamond$}}}
\put(690,192){\raisebox{-.8pt}{\makebox(0,0){$\Diamond$}}}
\put(721,250){\raisebox{-.8pt}{\makebox(0,0){$\Diamond$}}}
\put(753,240){\raisebox{-.8pt}{\makebox(0,0){$\Diamond$}}}
\put(784,145){\raisebox{-.8pt}{\makebox(0,0){$\Diamond$}}}
\put(815,277){\raisebox{-.8pt}{\makebox(0,0){$\Diamond$}}}
\put(846,359){\raisebox{-.8pt}{\makebox(0,0){$\Diamond$}}}
\put(878,316){\raisebox{-.8pt}{\makebox(0,0){$\Diamond$}}}
\put(909,354){\raisebox{-.8pt}{\makebox(0,0){$\Diamond$}}}
\put(940,402){\raisebox{-.8pt}{\makebox(0,0){$\Diamond$}}}
\put(972,440){\raisebox{-.8pt}{\makebox(0,0){$\Diamond$}}}
\put(1003,445){\raisebox{-.8pt}{\makebox(0,0){$\Diamond$}}}
\put(1034,467){\raisebox{-.8pt}{\makebox(0,0){$\Diamond$}}}
\put(1066,538){\raisebox{-.8pt}{\makebox(0,0){$\Diamond$}}}
\put(1097,202){\raisebox{-.8pt}{\makebox(0,0){$\Diamond$}}}
\put(1128,221){\raisebox{-.8pt}{\makebox(0,0){$\Diamond$}}}
\put(721,707){\makebox(0,0)[r]{With EdgeCollection}}
\multiput(743,707)(20.756,0.000){4}{\usebox{\plotpoint}}
\put(809,707){\usebox{\plotpoint}}
\put(283,737){\usebox{\plotpoint}}
\multiput(283,737)(5.274,-20.074){6}{\usebox{\plotpoint}}
\multiput(314,619)(4.487,-20.265){7}{\usebox{\plotpoint}}
\multiput(345,479)(13.789,15.513){3}{\usebox{\plotpoint}}
\multiput(377,515)(14.676,-14.676){2}{\usebox{\plotpoint}}
\multiput(408,484)(8.716,-18.837){3}{\usebox{\plotpoint}}
\multiput(439,417)(9.519,18.444){4}{\usebox{\plotpoint}}
\multiput(471,479)(5.776,19.936){5}{\usebox{\plotpoint}}
\multiput(502,586)(4.427,-20.278){7}{\usebox{\plotpoint}}
\multiput(533,444)(12.579,16.510){3}{\usebox{\plotpoint}}
\multiput(565,486)(15.403,13.912){2}{\usebox{\plotpoint}}
\multiput(596,514)(7.419,19.384){4}{\usebox{\plotpoint}}
\multiput(627,595)(2.584,-20.594){12}{\usebox{\plotpoint}}
\multiput(659,340)(17.184,11.641){2}{\usebox{\plotpoint}}
\multiput(690,361)(7.666,19.288){4}{\usebox{\plotpoint}}
\multiput(721,439)(13.369,-15.876){2}{\usebox{\plotpoint}}
\multiput(753,401)(13.762,15.537){3}{\usebox{\plotpoint}}
\put(801.39,441.61){\usebox{\plotpoint}}
\multiput(815,446)(5.232,20.085){6}{\usebox{\plotpoint}}
\multiput(846,565)(8.629,18.877){4}{\usebox{\plotpoint}}
\multiput(878,635)(7.186,-19.472){4}{\usebox{\plotpoint}}
\multiput(909,551)(2.821,20.563){11}{\usebox{\plotpoint}}
\multiput(940,777)(6.563,-19.690){5}{\usebox{\plotpoint}}
\multiput(972,681)(20.585,2.656){2}{\usebox{\plotpoint}}
\put(1018.53,698.52){\usebox{\plotpoint}}
\multiput(1034,712)(7.546,19.335){5}{\usebox{\plotpoint}}
\multiput(1066,794)(1.381,-20.710){22}{\usebox{\plotpoint}}
\multiput(1097,329)(13.543,15.728){2}{\usebox{\plotpoint}}
\put(1128,365){\usebox{\plotpoint}}
\put(765,707){\makebox(0,0){$+$}}
\put(283,737){\makebox(0,0){$+$}}
\put(314,619){\makebox(0,0){$+$}}
\put(345,479){\makebox(0,0){$+$}}
\put(377,515){\makebox(0,0){$+$}}
\put(408,484){\makebox(0,0){$+$}}
\put(439,417){\makebox(0,0){$+$}}
\put(471,479){\makebox(0,0){$+$}}
\put(502,586){\makebox(0,0){$+$}}
\put(533,444){\makebox(0,0){$+$}}
\put(565,486){\makebox(0,0){$+$}}
\put(596,514){\makebox(0,0){$+$}}
\put(627,595){\makebox(0,0){$+$}}
\put(659,340){\makebox(0,0){$+$}}
\put(690,361){\makebox(0,0){$+$}}
\put(721,439){\makebox(0,0){$+$}}
\put(753,401){\makebox(0,0){$+$}}
\put(784,436){\makebox(0,0){$+$}}
\put(815,446){\makebox(0,0){$+$}}
\put(846,565){\makebox(0,0){$+$}}
\put(878,635){\makebox(0,0){$+$}}
\put(909,551){\makebox(0,0){$+$}}
\put(940,777){\makebox(0,0){$+$}}
\put(972,681){\makebox(0,0){$+$}}
\put(1003,685){\makebox(0,0){$+$}}
\put(1034,712){\makebox(0,0){$+$}}
\put(1066,794){\makebox(0,0){$+$}}
\put(1097,329){\makebox(0,0){$+$}}
\put(1128,365){\makebox(0,0){$+$}}
\end{picture}

\end{center}
\caption{Timing results for Parallel ACP}
\label{results}
\end{figure}

\section {Some Remarks on Orca}

Using Orca to implement the parallel version of ACP was relatively
easy since communication in Orca is done implicitly using the shared
object data model \cite {orca}. Since parallel ACP is a highly
asynchronous application, the implicit communication is a great relief
for the programmer. However, as said before there are also some
criticisms which can be made on Orca.

First of all, an enormous amount of data copying is done in Orca
which makes things slow. To illustrate this fact, one bug
in the serial version of ACP turned out to be the fault of the
compiler which forgot to make copies of an object which was passed as a
parameter to an operation. After the bug was fixed the program became
35 times slower. To overcome this slow-down, the object was removed
and implemented in the traditional way (as in the C version) which
resulted in the serial Orca version which was ``only'' 5 times slower than the
serial C version. 

Furthermore, using profiling information on the serial Orca version,
it turned out that explicitly assigning fields of a record to
copy the record was faster than using the record assignment operator. 

As a last example of the inefficient implementation of Orca,
consider the implementation of the dictionary, which is a list of
lists. Since the dictionary can dynamically grow, this had to be
implemented using the graph construct of Orca. In the initialization
phase of the serial Orca version, we would like to know all categories
to which a word belongs (see section 3). Since the dictionary stores
exactly this information per word, the dictionary contains a function
{\em give\_ith\_elt\_dict(i, symbollist)}, which returns the list of
non-terminals to which the $i^{th}$ word in the dictionary belongs. To
speed things up, this can be efficiently done by returning a {\em
pointer} to the symbollist. However, in Orca this is done by returning
a {\em copy} of the list which is again very slow and inefficient,
since we will use the list only for reading.

So even though Orca takes away the burden of explicit message passing
and provides the user with indivisible operations on shared data,
there is still a lot to be done to enhance the efficiency of the Orca
implementation and to the programmer concentrate on its own
problem(s) instead of concentrating on the ways to avoid the slow Orca
constructs.

\bibliographystyle{unsrt}
\begin{thebibliography}{99}

\bibitem{greg}
   G.V.~Wilson.
   {\em Assessing the Usability of Parallel Programming Systems : The
   Cowichan Problems.}
   ftp.cs.vu.nl:/pub/gvw/cowichan.ps.Z.
   VU, Amsterdam, October 1993.

\bibitem{terry}
   T.~Winograd.
   {\em Language as a Cognitive Process.}
   Addison-Wesley, 1983.

\bibitem{amoeba1}
  S.J.~Mullender, G.~van~Rossum, A.S.~Tanenbaum, R.~van~Renesse, H.~van~Staveren.
  {\em Amoeba -- A Distributed Operating System for the 1990s.}
  IEEE Computer, 1990.

\bibitem{orca}
  H.E.~Bal, A.S.~Tanenbaum, M.F.~Kaashoek.
  {\em Orca: A Language for Distributed Programming.}
  SIGPLAN Notices 25, May 1990.

\bibitem{acp}
  H.E.~Bal, I. Athanasiu.
  {\em The Arc Consistency Problem: a Case Study in Parallel
  Programming with Shared Objects}
  Unpublished paper, 1993.

\end{thebibliography}

\end{document}

