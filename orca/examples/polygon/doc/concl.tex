\sectionLabel{Conclusions}

The main conclusion is that the Overlay Problem is hard to parallelize,
because of the bad ratio between communication and computation.
This make the Overlay Problem a killer application for parallel
programming systems.

The outcome of the experiments comparing the two algorithms for solving the
Overlay Problem is that Patchwised Overlay is better than 
Data-Parallel Overlay for the following reasons:
\begin{itemize}
  \item Patchwised Overlay is faster than Data-Parallel Overlay
  as the problem size increases, because sorting of the B map takes less time.
  In Patchwised Overlay the B map is sorted in parallel
  by the workers; each worker sorts a sub-map.
  In contrast, in Data-Parallel Overlay the B map is primarily
  sorted by the master; each worker only merges sorted parts of the
  B map.

  \item The workers in Patchwised Overlay use less memory than the
  workers in Data-Parallel Overlay, because the B map is partitioned in
  the former case, whereas it is replicated in the latter case. The
  difference in memory usage is especially important for large input
  maps.

  \item Performance tuning in Patchwised Overlay was easier
  than in Data-Parallel Overlay where it was hard to find a
  good solution for balancing the sorting load between Master
  and workers. Note, however, that
  the algorithm of Data-Parallel Overlay itself is less complex.
\end{itemize}

Writing the parallel Overlay programs in Orca was easy because of
the shared data objects, which supports communication and
synchronization in parallel programs at a high level. 
Performance tuning of the programs, however, was rather difficult since the
Orca system hides important low level details like communication
overhead, object distribution, and message fragmentation.

Orca programs perform well as was demonstrated by the comparison
between sequential solutions of the Overlay problem written both in
C and Orca; the observed difference between Orca and C was less than
3\%.
