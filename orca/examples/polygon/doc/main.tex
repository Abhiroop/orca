\documentstyle[a4wide,11pt,times,sty/subfigure,epsf]{article}
%\documentstyle[a4wide,12pt,times,sty/subfigure,epsf]{article}

\title{
    {\bf Parallelizing the Polygon}
    Overlay \hspace{-7.5ex}\raise.15ex\hbox{Overlay}
    {\bf Problem Using Orca}
}
\author{
    H.F. Langendoen \\
    Faculteit der Wiskunde en Informatica\\
    Vrije Universiteit Amsterdam
}
\date{August 1995}

\begin{document}

\bibliographystyle{sty/henkm}
\newcommand{\newindex}[1]{\index{#1}#1}               % Text and index ref.
\newcommand{\refAndCapt}[1]{\caption{#1}\label{#1}}   % Ref and caption.
\newcommand{\refByCapt}[1]{Figure~\ref{#1}}           % `figure' + reference.
\newcommand{\sectionLabel}[1]{\section{#1}\label{#1}} % Section and label.
\newcommand{\subsectionLabel}[1]{\subsection{#1}\label{#1}} % Subsection + lbl.
\newcommand{\refSection}[1]{Section~\ref{#1}}         % 'Section' + reference.

\maketitle

\parindent 0pt
\noindent
\parskip 1ex

\begin{abstract}
  \parindent 0pt
  \noindent
  \parskip 1ex
  \vspace{-4ex}

  This report discusses the implementation of the Polygon Overlay
  problem in Orca on an architecture consisting of 80 SPARC processors
  connected by an Ethernet.
  Orca is based on shared, consistent objects, which reduces the concern
  about transmission and data consistency.
  This should make the life of a parallel programmer simpler.

  The report deals with two aspects of implementing the Polygon Overlay
  problem,
  firstly to study the characteristics of and solutions to the problem,
  and secondly to study the usability of Orca in this problem domain.

  The main result of this report is that the Polygon Overlay problem
  is hard to parallelize, because of the bad ratio between communication
  and computation.
  Furthermore the Orca language is well suited to implement the found
  algorithms that solve the Polygon Overlay problem.

\end{abstract}

\section{Introduction}
\label{sIntroduction}

\begin{sloppypar}
Parallel programming is becoming more important as the speed of
single-processor
systems approach the limit imposed by the speed of light. While
communication in multi-computers (such as hypercubes and transputer
grids) is still faster than communication on a local area network such
as an Ethernet, distributed
systems based on LANs of workstations are becoming more attractive for
coarse-grained parallel applications because of their wide availability
and because they can easily be expanded with off-the-shelf components.
\end{sloppypar}

Parallel programming, however, raises new difficulties for the
programmer. Since multiple CPUs work at the same time, a program's
behaviour cannot simply be deduced by reading its lines of code
step-by-step. It is difficult to determine which processor is doing
what at a certain moment. So synchronisation of the
processes is something the programmer has to think about. Another
problem is keeping data consistent between processors. As different
processors all apply different operations to copies of the same data,
these copies become different. To ensure that the copies remain consistent,
the same operations should be applied to all copies in the same
order.

These problems seem to be more easily solved on systems which offer
shared memory than on systems which use message passing as a model.
Distributed systems, however, do not have shared memory. This is why the
Orca programming language \cite{tse92,bal} provides
{\em logically shared data}, even on
systems that do not have physical shared memory. This model is called
{\em Distributed Shared Memory}.

Orca was designed at the Vrije Universiteit for implementing parallel
applications on distributed systems. It is not an extension to an
existing programming language, but is designed from scratch to integrate
sequential and parallel constructs cleanly. As part of an effort to
investigate the suitability of Orca for larger software projects, a
discrete simulation of a chaotic dynamic system, the Turing
Ring, was implemented. During the development of the program, special
attention was given to the ease of programming. Performance results
were then gathered to examine the performance of Orca-based programs.

This paper is organised as follows. First, the Turing Ring is described
in detail, and some background information is given. Next, the Orca
programming language and its implementation on top of the 
Amoeba operating system is described. Several possible
implementations of the Turing
Ring are considered, and the final program is described. Finally,
performance results are presented and the usability of Orca is
discussed.

The work presented here has been performed at the Vrije Universiteit,
Amsterdam as an afstudeerproject under the supervision of Gregory V.
Wilson \nocite{Cowichan}. Thanks for the other things he did, too.
Thanks to Marc Baehr for his numerous comments;
I would also like to thank everyone who listened to me and then sent me back to
work again.


\sectionLabel{Problem Description}

Polygon Overlay is commonly done in spatial information systems
\cite{White:1978}.
Suppose two maps have been supplied,
$A$ and $B$.
Each map covers the same geographical area,
and is decomposed into a set of non-overlapping polygons
(\refByCapt{Example of Polygonal map Overlay}).
The aim is then to overlay the maps,
i.e.\ to generate a new map consisting of
the non-empty elements in the geometric intersection of $A$
and $B$.
This problem frequently arises in geographical information
systems,
in which the first map might represent soil type,
and the second,
vegetation.
Their overlay then shows how combinations of soil type and
vegetation
are distributed.

\begin{figure}[hbtp]
  \begin{centering}
    \hspace{0cm}
    \epsffile{polygonExample.eps}
    \refAndCapt{Example of Polygonal map Overlay}
  \end{centering}
\end{figure}

Although this problem has been chosen to serve as a 'large' benchmark,
the problem as it is now is far too complicated.
In order to simplify the implementation of the problem,
the following requirements are made:
\begin{enumerate}
  \item
    All polygons are non-empty rectangles.
    The polygons are chosen to be rectangles because
    they are easy to store
    (only the upper-left and the lower-right corner are needed)
    and they are convex.
    Calculating an overlay of non-convex polygons is much
    harder.
  \item
    The vertices of the polygons lie on an integer grid
    $[1{\ldots}N] {\times} [1{\ldots}M]$.
    This is done to simplify arithmetic,
    and generation of a polygon map.
    Note that if the polygons were not rectangular it might
    not be possible to put every vertex of every output
    polygon on the same integer grid.
  \item
    The input maps have identical extents.
  \item
    Each map is completely covered by its rectangular
    decomposition.
    Holes in a map are not allowed as they are hard to trace;
    overlapping polygons are not allowed either.
  \item
    Input and output maps are not sorted in any fashion.
  \item
    Both of the input maps and the output map can be held
    in memory. 
  \item
    The program should take two input files and deliver an
    output file, all of the same format.
  \item
    Each line in each file should contain four integers
    representing the upper-left and the lower-right corner of
    a single polygon.
    The first polygon in the file should represent the outer
    surrounding rectangle.
    The rest of the file should hold the decomposition of the
    first polygon.
\end{enumerate}



\sectionLabel{Sequential Algorithms}

Several sequential programs for solving the polygon overlay problem
were developed in order to try out and refine different algorithms.
Only the simplest and fastest algorithms are discussed here.


\subsection{All Against All}

This method checks every polygon of the A map with all the polygons
of the B map.
To do so, each B polygon is checked against all polygons of the A map.
This method is the most straightforward algorithm to solve
the polygon overlay problem, and is easy to implement.
Unfortunately it one of the slowest.
Its time complexity is typically of order $O ( n_A \cdot n_B )$, where
$n_A$ and $n_B$ denote the number of polygons in the A, respectively B,
map.


\subsectionLabel{The Sort-and-Delete Method}

In the previous method, many checks are done of
polygons that do not overlap and therefore produce no resulting
polygons.
In order to reduce these unnecessary checks the Sort-and-Delete
method exploits the geographical-locality of the maps.

Geographical-locality means that a polygon from the A map
can only overlap with a polygon from the B map if they both
lie in the same area.
To make use of this fact, the polygons of the B map are
sorted on their upper-left-corner (first sort on the x- and
than on the y-coordinate).
If a polygon of the A map is being checked against the
polygons of the B map, then this process can be stopped when
the lower right corner of the A polygon is smaller than the
upper left corner of the B polygons.
In other words it is useless to search any further if an area
is entered where there is no chance to find an overlap.

Each time when there is a new polygon taken from the A map
there is an area in the beginning of the B map where there
are no overlaps possible.
To be able to skip polygons of the B map whilst checking, it is
necessary to sort the polygons of the A map too on their 
upper-left-corner.
Now there are two options to find out what to skip in the
B map.

In the first option it is necessary to sort the B map also
on the lower-right corner.
The map can be represented as a double linked list.
Now the search in the B map can start at the point where the
right-lower corners of the B polygons are larger than the
left-upper corner of the A polygon.
In other words the search can start there where it becomes
possible to find an overlap.
Note that the starting point, when checking a new A polygon,
is always the same as the previous point or further on in
the B map, because the A map is sorted.

The second option makes use of the definition that the map
may not contain any holes or overlapping polygons.
Each time a polygon of the A map is being checked against
the polygons of the B map and an overlap is found,
then the surface that is overlapped is subtracted from the
surface of the B polygon.
If all the surface of this B polygon is overlapped it can be
deleted from the B map.

It is clear that the second option is better, because
it requires less sorting, and often a polygon in the B map is
taken out sooner than the same polygon is skipped when applying
the first option.

There is one final optimization that can be done.
Instead of stopping when the lower-right-corner of the A
polygon is smaller than the upper-left corners of the B
polygons, it is possible to stop when the surface of the A
polygon as been given away.
Just like a record is kept of the surfaces of the B polygons,
a record can be kept of the surface of an A polygon.
Now each time an overlay is found the surface of the A
polygon is also reduced with the surface of the overlap.
If the surface of the A polygon has dropped to zero, then it
is useless to search any further and the next A polygon can
be taken from the map.
Stopping when the surface is zero always happens at the
same time or mostly sooner as in the case where is checked
whether the border has been past.
Also the checks in the improved version are only done when
an overlap has been found, instead of before each check.

The Sort-and-Delete method incorporates the above mentioned
optimizations as follows.
Both the A an B map are sorted on the upper-left-corner.
A record of the surface area is kept of the B polygons and
the current A polygon.
When an overlap is found, both the surfaces of the
concerning A polygon and B polygon are reduced with the
surface of the overlap.
If the surface of the B polygon is zero, then the polygon is
deleted from the B map.
If the surface from the A polygon is zero, then the search
proceeds with the next A polygon.

The Sort-and-Delete method has a typical time complexity of order
$O ( n_A \log n_B )$, where $n_A$ and $n_B$ denote the number of polygons
in the A, respectively B, map.
It must be stated tough, that there are exceptional input maps that result
in checking all against all polygons, see
\refByCapt{worst case}.

\begin{figure}[hbtp]
  \begin{centering}
    \hspace{0cm}
    \epsffile{worstCase.eps}
    \refAndCapt{worst case}
  \end{centering}
\end{figure}



%\input{parprog}

\newpage
\sectionLabel{Parallel Algorithms}

This section describes three different methods to solve the
Polygon Overlay problem in parallel. 
The outline of the methods were suggested by \cite{Wilson:1994}.
The main difference between the three methods is the way of
distributing the maps among the processors.



\subsection{Pipelined Overlay}

In a pipelined control-parallel implementation,
each of the $n_A$ polygons of the A map is allocated to one
of the $p$ processors, $1 \leq p \leq n_A$.
The polygons of the B map are then streamed through the
pipeline;
results are either passed through the same pipeline,
stored locally for collection upon termination,
or "bled off" as the pipeline executes
(\refByCapt{Pipelined Overlay}).

\begin{figure}[hbtp]
  \begin{centering}
    \hspace{0cm}
    \epsffile{pipelined.eps}
    \refAndCapt{Pipelined Overlay}
  \end{centering}
\end{figure}

This method is suited for architectures in which
processors are arranged in a grid, where each pipelined message
is transmitted over a different communication channel.
This method, however, is not appropriate for the Zoo's architecture,
because the pipelined messages would all be transmitted over
the same Ethernet channel.
Contention between them would limit performance.

But even if the right architecture was used, 
Pipelined Overlay would still suffer from load balancing
problems, which emerge during optimization.
The way to optimize the method would be by combining it with
the Sort-and-Delete method.
The Sort-and-Delete method is already discussed in the sequential
case, see
\refSection{The Sort-and-Delete Method}.
Combining Sort-and-Delete with Pipelined Overlay is straight
forward.
The consequence is that a polygon can be taken out of the pipeline if
its surface has been overlapped totally by the polygons it has
been checked with by the previous processor(s).
This would cause serious load balancing problems, since the last
processors in the pipeline have magnitudes less of polygons to check
compared to what the first processors have to check.



\subsection{Data-Parallel Overlay}

In a pure data-parallel version, the whole of map A is
read into a parallel structure in memory, or, equivalently,
each polygon is associated with a single virtual processor
(\refByCapt{Data-Parallel Overlay}).
Polygons from the B map are then read in one at a time,
and the intersection of a B polygon with each A polygon is
calculated simultaneously at each step.
Non-empty intersections are collected using a scan or gather
operation, and written to file.

\begin{figure}[hbtp]
  \begin{centering}
    \hspace{0cm}
    \epsffile{dataParallel.eps}
    \refAndCapt{Data-Parallel Overlay}
  \end{centering}
\end{figure}

The method as described above cannot be implemented directly on the Zoo's
architecture, because the fine grain manner of handling polygons one by
one causes too much communication overhead.
By enlarging the grain size, the Data-Parallel method can be
implemented as follows.

Map A is partitioned in equally sized sets of polygons,
which are called sub-maps.
Each sub-map of A is placed on a different processor.
Note that since map A has no ordering,
polygons from a sub-map can be scattered over the whole map, and
therefore the sub-maps cannot be classified as being polygon maps
since they violate the requirement that they contain no holes, see
\refSection{Problem Description}.
Map B is replicated on each processor.
Now each processor overlays its A sub-map with its
copy of the B map, and sends all the resulting polygons to the
collector.

This method has two advantages:
\begin{enumerate}
  \item
    It is relatively simple to implement.
  \item
    The number of polygons per processor is practically equal.
    This guarantees a good load balance.
\end{enumerate}



\subsection{Patchwised Overlay}

This method exploits geographical locality in contrast to
the previous method, which partitions the map in random sets.

Map A is partitioned in so called "patches", see 
\refByCapt{Patchwised Decomposition}.
The number of patches is equal to the number of participating
processors.
All polygons which are situated inside the borders of a certain
patch are put in a corresponding sub-map.
Each sub-map is placed on a different processor.
Map B is either replicated on all the processors or is 
partitioned similarly to the A map, with the same patch borders.
Now all intersections are calculated by the processors and the
results are gathered.

\begin{figure}[hbtp]
  \begin{centering}
    \hspace{0cm}
    \epsffile{patchwised.eps}
    \refAndCapt{Patchwised Decomposition}
  \end{centering}
\end{figure}

The method above suggests that the B map might be replicated
on all processors.
Although this simplifies distribution of the B map,
replication is not good for performance as the number of comparisons
is dramatically higher compared to the alternative of partitioning
the B map in patches.
The amount of work to partition the map patches will be
significantly less than the overhead of all the useless comparisons
done in the case of replication.

For simplicity the partitioning is changed from a gridwise one,
as suggested in
\refByCapt{Patchwised Decomposition}, to a columnwise partitioning.
The advantage of columnwise partitioning is threefold:
\begin{enumerate}
  \item
    It simplifies understanding of the algorithms.
  \item
    Making a gridwise partitioning is not a trivial process,
    and if there is an algorithm for it, then it is unlikely to be simple
    and fast.
    Making a columnwise partitioning, on the other hand, is simple and fast.
  \item
    It makes the methods for dealing with 'border polygons',
    mentioned below, less complex.
\end{enumerate}

One important aspect of Patchwised Overlay, not mentioned  by
\cite{Wilson:1994}, is the handling of 'border polygons'.
A border polygon is a polygon who's surface is spread over more
than one patch.

There are three ways to deal with these border polygons.
The first method puts the border polygons in all the
concerning sub-maps, as 
\refByCapt{Patchwised Decomposition} already suggested.
The second method adjusts the borders of the patches to the polygons,
hence, a border polygon is placed in only one of the sub-maps.
The third method adjusts the border polygons to the patches by
clipping them, hence, a border polygon is split into sub-polygons,
one per sub-map.
All three methods have difficulties to ensure that the correct resulting
maps are produced.
How the three methods work in detail, what their specific problems
are, and how those problems are dealt with is discussed below.

\begin{description}
  \item[Duplicate-and-Filter]
    Map A is partitioned in such a way that each border polygon is 
    duplicated in those sub-maps that correspond to the 
    patches that overlap the border polygon.
    Map B is similarly divided in patches, and with
    the same procedure for border polygons.

    The consequence of duplicating border polygons is that
    two neighboring processors can generate the same resulting
    polygon.
    This problem emerges when both the processors get the same A
    and B border polygon, and those border polygons are also
    overlapping each other.
    According to the requirements made in
    \refSection{Problem Description} this is not allowed,
    so these duplicates should be filtered out.
    This could be done at the end by the collector, but that would be
    a time consuming sequential process, which reduces speedup.
    Filtering can also be done on each processor during
    the creation of duplicate polygons.
    This is accomplished by marking all the copies of every border
    polygon of map A and B, except those copies that are laying in the
    patch that contains their upper left corner, see
    \refByCapt{Patchwised decomposition with marked border polygons}.
    If during overlay a nonempty intersection of two marked polygons
    is found, then it should be thrown away.
    It should be thrown away because the patch with at least one
    of the unmarked polygons finds it too, and therefore keeps it.

    Thus generating duplicates is not such a problem as long as they
    are filtered out locally and it doesn't involve too much work.

    \begin{figure}[hbtp]
      \begin{centering}
        \hspace{0cm}
	\epsffile{duplicateAndFilter.eps}
        \refAndCapt{Patchwised decomposition with marked border polygons}
      \end{centering}
    \end{figure}


  \item[Extending-Borders]
    Map A is partitioned, in such a way that border polygons are not
    duplicated.
    Instead a border polygon is placed in the sub-map that corresponds
    to the patch which contains the polygon's upper left corner.
    See for example 
    \refByCapt{farthest border polygons in the A sub-maps}, where border
    polygon $a_{1}$ is placed in the sub-map that corresponds with
    patch $A_{1}$, $a_{2}$ in the corresponding sub-map of $A_{2}$,
    and $a_{3}$ in the corresponding sub-map of $A_{3}$.
    Each right B patch border is now extended to cover the A border
    polygon sticking out farthest of the corresponding A patch.
    This is done to ensure that every A border polygon will produce
    all possible intersections with the B polygons.
    Now the B map will be partitioned according to the extended B
    patches, and B border polygons are duplicated in those sub-maps
    which correspond to the patches that overlap the border polygon.
    An example is given in
    \refByCapt{overlapping borders of the B sub-maps} where the border
    polygons $a_{1}, a_{2},$ and $a_{3}$ are sticking out the farthest of
    their A patches. The $B_{1}, B_{2},$ and $B_{3}$ patches are
    extended to cover the $a_{1}, a_{2},$ and $a_{3}$ border polygons
    respectively.
    Note that a B sub-map may contain polygons which do not overlap
    with any polygon of the corresponding A sub-map, which means that
    there are too many B polygons duplicated.
    Correcting this by keeping track of all the right sides of
    the A border polygons, would involve more work than is gained.

    \vspace*{-0.4cm}
    \begin{figure}[hbt]
      \begin{centering}
        \hspace*{1cm}
        \hspace*{-1cm}
        \subfigure [farthest border polygons in the A sub-maps] {
          \hspace*{2cm}
          \epsffile{extendingBorders_a.eps}
          \hspace*{2cm}
        \label{farthest border polygons in the A sub-maps}
        }
        \hspace*{-0.5cm}
        \subfigure [overlapping borders of the B sub-maps] {
          \hspace*{1.5cm}
          \epsffile{extendingBorders_b.eps}
          \hspace*{2cm}
        \label{overlapping borders of the B sub-maps}
        }
      \end{centering}
      \vspace{-0.8cm}\refAndCapt{Extending-Borders}
    \end{figure}

    Instead of looking at the upper left corner of a border polygon
    during the partitioning of the A map, it is also possible to put
    the polygon in the sub-map for which the patch contains the most
    surface of the polygon.
    This would give a better distribution of polygons per sub-map,
    what should result in better load balancing.
    The costs for doing this is that the left borders of the B patches
    have to be extended as well.
    In the general case, when the input maps are large, this is not
    worth the trouble, because then the number of polygons is much
    larger than the number of patches.
    This implies that the chances are small that border polygons are
    sticking out far, which could have caused the unbalance

  \item[Clip-and-Merge]
    During the partitioning of map A, each border polygon is
    duplicated in all the sub-maps that correspond to the
    overlapping patches, with the adjustment that each duplicate is
    clipped such that it fits into the concerning patch, see
    \refByCapt{Patchwised decomposition with clipped border polygons}.
    Map B is partitioned similarly, with the same patch structure.
    After the overlay is done, the result map can contain
    disjoint polygons that must be merged into one polygon.
    This can only be done at the end, when all the results are gathered.
    As the merging of polygons is a considerably expensive sequential
    process, it will reduce speedup, and therefore this method is not
    worthwhile to implement.

    \begin{figure}[hbtp]
      \begin{centering}
        \hspace{0cm}
	\epsffile{clipAndMerge.eps}
        \refAndCapt{Patchwised decomposition with clipped border polygons}
      \end{centering}
    \end{figure}

\end{description}

It is not clear beforehand which of the two remaining methods
Duplicate-and-Filter or Extending-Borders will be the best.
The difference in performance will not be dramatic, since,
in the general case where the maps are large, the number of border
polygons are relatively small.
As performance will not be dramatic, only one of the two methods is
chosen.
In this case Duplicate-and-Filter is chosen as the method is simplest
to understand.



\subsection{Data-Parallel Versus Patchwised}

If both the Patchwised overlay and the Data-Parallel overlay methods are
not optimized then it is clear that Patchwised Overlay is much faster,
because in the Patchwised method the B maps is partitioned, where in the
in Data-Parallel method the whole B map is replicated on each
processor.
This makes the number of comparisons in the Patchwised method
approximately $p$ times less than in the Data-Parallel method,
where $p$ is the number of processors.
But if both methods are optimized, which will be discussed in the
next section, then this dramatic difference will disappear, and it
is not so clear anymore which method will be the best.

The main disadvantage of Patchwised overlay is that the number of polygons
may differ strongly per sub-map, which may cause load balancing problems.
There are even exceptions that make patchwised partitioning of the maps 
useless.
An example was already given in 
\refByCapt{worst case}.
In the Duplicate-and-Filter method all the processors get all the A
and B polygons, and all processors will calculate the same resulting
polygons, and all processors, except the first, will throw all the
results away.
In the Extending-Borders method only the first processor will get all
the A and B polygons, and the rest of the processors will be idle.
In both cases there will only be 'slowdown' instead of speedup.
An other disadvantage of Patchwised Overlay is that the presence of
border polygons causes extra work.


\sectionLabel{Parallel Optimizations}

In the sequential algorithm the Sort-and-Delete method was used to
reduce the number of comparisons, which improved the overlay process
drastically.
In order to apply the Sort-and-Delete method to the parallel algorithms
both the Sort-and-Delete method and the parallel algorithms have to be
adapted slightly, which is discussed below.

This section also addresses another optimization in
\refSection{Message Fragmentation}, which involves message fragmentation.


\subsectionLabel{Sort-and-Delete}

The Sort-and-Delete method has already been discussed in
\refSection{The Sort-and-Delete Method} for the sequential algorithm.

In both the remaining parallel methods it is possible to apply
the Sort-and-Delete method separately to each processor's A and B
sub-maps.
The only difference with the sequential case is that the two sub-maps
are not completely overlapping.
This means that it is no longer valid, to 
stop checking a polygon if all its surface has been overlapped.
In each parallel method this problem is solved differently.

\subsubsection{Data-Parallel Overlay}

To increase parallelism, the A sub-maps are sorted on each
processor.
The B map is sorted before replicating it on all processors,
because then this sorting can already start at the master processor
while the slave processors are sorting their A sub-maps.
Note that sorting introduces a sequential aspect, which reduces
speedup, but the gain is that the number of comparisons is reduced
dramatically.

On each processor each polygon of the A sub-map will be completely
overlapped by the polygons of the B map by definition, but the
converse is definitely not true.
The consequence of this is that a B polygon must be deleted from the
map not only if it has been completely overlapped, but also if
its lower right corner is smaller than the upper left corner of the A
polygon currently checked.


\subsubsection{Patchwised Overlay}

The A and B sub-maps are sorted locally at each processor.

In the case of Duplicate-and-Filter, the surface of each A and B border
polygon is adjusted to only be the size of what falls within the borders of
the underlying patch.
If during overlay a border polygon has a non-empty intersection with
an other border polygon, then only the resulting surface that lays
within the underlying patch is subtracted from both polygons.
Therefore the search per polygon in the other map can stop if its surface
has been completely overlapped, because now both maps have the same surface
to be intersected.


\subsectionLabel{Message Fragmentation}

At first Orca did not do message fragmentation, which imposed a
restriction on the amount of data for operation parameters. 
As a consequence the distribution of larger (sub)maps was impossible.
Therefore the Orca code was modified to send a large map as a sequence of
chunks.
In essence, the code was modified to do message fragmentation by hand,
but having implemented the message fragmentation the following
optimizations could be realized.

Instead of at the end sending all the resulting polygons to the
collector, which writes them to a file,
the sending is done when enough resulting polygons have been produced
to fill a result sub-map to the maximum message size.
The advantage of this approach is that it reduces contention by
overlapping communication and computation.

An other optimization can be done in the Data-Parallel case during the
sorting and sending of the B map.
Instead of first sorting the entire map and than cutting it up and
sending it to all the processors, the map is first cut up in pieces
and than successively each part is sorted and sent to the other
processors.
At the processor side the parts are merged. 
In this way the sorting is done by the process that sends and all the
processes that receive the parts.


\sectionLabel{Parallel Implementations in Orca}

To determine which parallel algorithm is best, Data-Parallel or Patchwised, we
have coded both versions in Orca. The major issue in writing parallel Orca
programs is how to structure an application such that it can make efficient use
of the shared data objects provided for communication. Below we discuss the
objects used in both parallel overlay implementations.

\subsection{Data-Parallel Overlay}

\begin{figure}[hbtp]
  \begin{centering}
    \hspace{0cm}
    \epsffile{dataParObjects.eps}
    \refAndCapt{objects used in the Data-Parallel implementation}
  \end{centering}
\end{figure}

\refByCapt{objects used in the Data-Parallel implementation} shows that
the Data-Parallel Overlay implementation uses three types of
shared objects: queue-objects, a buffer-object, and counter-objects.
The queues and buffer are used to transport (partial) maps (i.e. lists of
polygons) between the Master and Worker processes.

The distribution of the A map is handled by the Master through 
GenericJobQueue objects from the standard Orca library.  Each Worker
receives its part of the A map through a private $\rm{Queue_{A}}$ that
is connected to the Master.  The Master distributes the parts of the A
map by cycling all the queues connected to the workers.

The replication of the B map is also handled by the Master.
Unfortunately, the Master could not use a standard object to replicate
the B map at all the workers, because the B map is too large to be sent
in one operation (i.e. broadcast message). Therefore, the B map is
fragmented into smaller pieces. Putting these pieces in a standard
GenericJobQueue shared by all workers has the undesirable effect that all get
operations modify the queue to signal that the next piece can be
appended. Consequently, all operations on the queue are broadcast by the Orca
system, which has a negative impact on performance.

The solution is to artificially separate receiving of data and signaling
the master into operations on different objects. Each slave is
connected to the master through a single shared buffer ($\rm{Buffer_{B}}$)
for receiving parts of the B map, and an individual counter
($\rm{nbr\_free_{B}}$) to signal the receipt of a piece to the Master.
To increase performance, the buffer can hold multiple pieces, so the Master can
work ahead. This is particularly useful at the beginning since the Master
starts handing out the B map immediately after distributing the A map,
while the workers are still busy sorting their part of the A map.

Recall that the master sorts each piece of the B map, while the workers merge
them together to get a completely sorted B map. Merging a new piece with the
partial B map takes progressively more time during the distribution of the B
map. To avoid the master running idle, the buffer is filled with pieces of
different sizes. A heuristic based on the number of free slots in the shared
buffer is used by the master to determine the size of the next piece. Initially
many buffers are free, so the master sorts small pieces of the B map; at the
end, larger pieces are generated. In practice the heuristic manages to balance
the load fairly well.

Finally, the Master collects the result map through a single queue
($\rm{Queue_{Result}}$) that is shared by all workers. Again the standard
GenericJobQueue object is used for convenience.

\subsection{Patchwised Overlay}

The structure of the Patchwised Overlay implementation is shown in
\refByCapt{objects used in the Patchwised implementation}.  Note that
the Patchwised Overlay implementation just uses one type of shared
object, the standard GenericJobQueue.

The master communicates both parts of the A and B map to each worker
through a private queue ($\rm{Queue}$). The results are collected
through a single shared queue ($\rm{Queue_{Result}}$) just like with
the Data-Parallel version.

\begin{figure}[hbtp]
  \begin{centering}
    \hspace{0cm}
    \epsffile{patchObjects.eps}
    \refAndCapt{objects used in the Patchwised implementation}
  \end{centering}
\end{figure}

One important observation is that the object structure of the Patchwised
Overlay implementation is much less complicated than that of the
Data-Parallel implementation. Consequently, getting the Patchwised
Overlay implementation to work was a lot easier than for the
Data-Parallel implementation. Note, however, that the algorithmic
complexity of the Patchwised Overlay method is higher than that of the
Data-Parallel method.


\sectionLabel{Performance}

Before testing the overlay algorithms,
it is necessary to have some maps to overlay.
Since the problem has deliberately been simplified,
it is not possible to simply digitize real images.
Instead, the polygon maps must be generated synthetic.
The way the map is generated influences properties, like
the deviation of the average size of polygons in a map.
This in turn has an effect on the measurements.

In 
\refSection{Generation of Polygon Maps} the method to generate the
polygon maps is described.
Then the performances of the sequential programs
written in C and Orca are compared followed by the performance figures
of the parallel programs.
Finally the usability of Orca is discussed.


\subsectionLabel{Generation of Polygon Maps}

The method to generate the polygon maps uses a bitmap.
First an $n {\times} m$ bitmap is defined which represents the
surface what is going to be divided among the polygons.
Then $q$ $1 {\times} 1$ boxes are placed randomly in the
bitmap.
Each box is represented as a point in the bitmap, see for
example
\refByCapt{random placed boxes}.
The generation program then repeatedly selects a random box
and "grows" it.
This "growing" is done by randomly choosing
a free side of the selected box,
i.e. a side where all of the bits in the bitmap
next to this side are not taken by any other box.
As there is now a free side selected,
the box will be enlarged by extending the free side by
length one.
For instance in 
\refByCapt{4 is about to expand}
box 4 is selected and can only grow to the left,
as has been done in
\refByCapt{4 cannot expand any further}.
Box 4 cannot be selected any more because it is now enclosed by
all its neighbors.
The selecting and growing is done until non of the boxes are
able to grow any further.

At this point the surface of the bitmap may contain holes
as in
\refByCapt{there are holes left}.
In such a case a new $1 {\times} 1$ box is placed in the hole
and grows until it has filled up the hole.
To find the holes a scan must be made through the bitmap.
The map is completed once there are no more holes,
as in
\refByCapt{9 and 10 make the map complete}.
The resulting polygon map is of size $[n+1] {\times} [m+1]$
and the number of boxes in the map is always greater or
equal to $q$.
Note that the boxes may not be put in the map in the order
they were found,
because,
during the filling of the holes,
the generator produces the boxes in an ordered way.
This is not allowed as one of the requirements is that a map
should not be ordered in any way (see
\refSection{Problem Description}).

\begin{figure}[hbt]
   \begin{center}
      \subfigure [random placed boxes] {
        \epsffile{bitmap_a.eps}
        \label{random placed boxes}
      }
      \hspace*{4ex}
      \subfigure [4 is about to expand] {
        \epsffile{bitmap_b.eps}
        \label{4 is about to expand}
      }
      \hspace*{4ex}
      \subfigure [4 cannot expand any further] {
        \epsffile{bitmap_c.eps}
        \label{4 cannot expand any further}
      } 
      \subfigure [there are holes left] {
        \epsffile{bitmap_d.eps}
        \label{there are holes left}
      }
      \hspace*{6ex}
      \subfigure [9 \& 10 make the map complete] {
        \epsffile{bitmap_e.eps}
        \label{9 and 10 make the map complete}
      }
   \end{center}
   \refAndCapt{Creation of a Polygon Map Using a Bitmap}
\end{figure} 

Apart from the properties which were required in
\refSection{Problem Description},
generating the polygon map like described above has four more
consequences:
\begin{enumerate}
  \vspace*{-1ex}
  \item Most of the boxes will have the average size or a
    value very close to it because of the uniform distribution
    of the random generator.

  \item When there is a large amount of polygons in a map, it
    is unlikely that there is a very long polygon which covers
    the whole length of a map.
    Again this is a consequence of the uniform distribution of
    the random generator.

  \item Usually the distribution of polygons over the map will be
    fairly equal.
    Yet again this is due to the uniform distribution of the random
    generator.

  \item It will almost be impossible to generate a map that has
    exactly the same number of polygons as wanted.
    This is because prior to the map generation it is not known how
    many holes will be created and filled during generation.
\end{enumerate}



\subsection{ Sequential Performance (Orca versus C) }

All the measured times presented in this report are without 
file access times.
This is done because on the Amoeba system it is unacceptably high at
the moment.
For instance, the sequential C program take 2.2 times longer to
finish for two input maps both containing 40,000 polygons.
The sequential Orca program needs 3.6 times more time.

Furthermore all the measurements, both sequential and parallel, are done
for two problem sizes.
The first problem size consists of two input maps of 40,000 polygons (40K)
each, and the second problem size consists of input maps of 400,000
polygons (400K) each.

All the maps are square, this gives the worst kind of columnwise
distribution.

\begin{table}[hbt]
  \centering
  \begin{tabular}{|l|c|c|}
    \hline
		& 40K  		& 400K     \\
    \hline
    C 		& 25997 ms	& 758824 ms   \\
    Orca	& 26037 ms 	& 738841 ms   \\
    \hline
  \end{tabular}
  \caption{Orca versus C}
  \label{Orca versus C}
\end{table}

Table 
\ref{Orca versus C} shows that difference in execution time between the
sequential C program and the Orca program is negligible. 
Considering that Orca uses C as intermediate code, it must be noted
that the performance of Orca is good.

\subsection {Parallel Performance}

\refByCapt{Speedup for different problem sizes} shows the speedup
for two different problem sizes, and for the two Overlay variants.
In both cases the performance of Patchwised Overlay is better than
that of Data-Parallel Overlay.
Relatively the performance difference gets larger as the problem size
increases.
This is because in Patchwised Overlay the B map is sorted in
parallel by the workers; each worker sorts a sub-map.
This is in contrast to Data-Parallel Overlay, where the B map is mainly
sorted by the master, and each worker does only merge sorted parts.

\begin{figure}[hbt]
  \begin{center}
    \leavevmode
    \epsfysize=8cm
    \epsfbox{timeSpeedup.eps}
    \refAndCapt{Speedup for different problem sizes}
  \end{center}
\end{figure}

\begin{figure}[hbt]
  \begin{center}
    \leavevmode
    \epsfysize=8cm
    \epsfbox{time400.eps}
    \refAndCapt{Execution times of Patchwised Overlay}
  \end{center}
\end{figure}

It can be seen that speedup is poor.
The reason for this is communication overhead.
Calculation shows that in the case of overlaying two 400K maps,
the number of bytes to send to the workers is approximately
19MB, that is 400K polygons * 2 maps * 6 integers(4 coordinates, 1 surface,
1 list pointer) * 4 bytes(size of an integer).
The number of resulting polygons produced by the workers is approximately
1480K, and have to be sent back to the master.
So the total amount of data that must be transferred by the network is
in this case 19 + 36 = 55MB.
What kind of, quantitative, influence this has on the performance can be
seen in
\refByCapt{Execution times of Patchwised Overlay}.
Note that only Patchwised Overlay is considered as it is the
fastest of the two methods.
\refByCapt{Execution times of Patchwised Overlay}  shows next to the total
execution time also the total time spent in communication, measured at
the Orca level.
Note that the amount of time spend in communication increases with the
number of workers, because the number of border polygons will increase
relatively to the number of patches.
The Ethernet has a transfer rate of 10 Mbits/sec,
which means that it takes 44 seconds to send all the data across the
network.
This implies that Orca needs approximately 56 seconds.
This time is spent in copying of data in and out objects, in and out of
message buffers and in and out of the Ethernet card.

All in all the poor speedup thus is not surprising.
To improve the speedup, the following suggestions are given:
\begin{enumerate}
  \item Using arrays of polygons instead of lists.
    This would mean 4 bytes less per polygon-entry.
  \item Calculating the surfaces of the polygons at the worker side.
    This would mean 4 bytes less per polygon-entry.
  \item Using integers of 2 bytes instead of integers of 4 bytes.
    The only problem is that Orca does not provide these short integers so
    they have to be implemented by putting two short integers into one
    Orca-integer.
    This would reduce the amount of bytes per polygon-entry to a
    total of 2 bytes.
\end{enumerate}

So the total speedup time could be increased by a factor 3.
Furthermore some kind of compression could be applied to decrease the
message size even more, but this would inevitably introduce a
sequential factor.

Still these optimizations will not take away the fact that the time
spent in communication is a sequential aspect and does not decrease
with the amount of workers used.
In all the possible solutions the network capacity will be the
bottleneck, which makes the Polygon Overlay problem a benchmark to
test the communication performance of parallel program systems.



\subsection {The Usability of Orca}

The aim of implementing the Polygon Overlay problem is not only to
see how fast Orca is and how well speedups are, but also to find out
how easy or hard programming with the language is.
Below some advantages and disadvantages of the Orca language are
discussed, which the author encountered during implementation of the
Polygon Overlay problem.

The Orca language has been found to be very straight forward, what
made the time to get acquainted with it very short.
The use of the shared data-object model makes parallel programming
simple, because its basics are easy to understand, without loss
of controlling the parallelism.
Therefore you can concentrate on the main issues instead of worrying
about communication details.
The standard libraries have been found very useful, in particular, it
was not necessary to use guards directly.
The drawback of the high level of abstraction arises during tuning
and optimizing, since all of a sudden the programmer needs to know
how shared objects are implemented in order to be able to make
improvements, or to explain program behavior.
For example, implementing the broadcast of map B in the Data
Parallel algorithm was difficult. 
The idea was to make an object that is written by the master and
subsequently read by all workers.
But as message fragmentation had to be done by hand on the B map, all the
workers had to read a part of map B before the master replaced it with
the next part.
The synchronization between master and workers is accomplished by
putting a counter in this shared object which counted the number of
workers which had read the current part.
The problem now was that the compiler saw that each operation on the
object was a write operation, because the counter was adjusted by
each worker.
The decision of the compiler was therefore not to replicate the
object on each worker, but only to put it on the master.
A strategy call on the object, to replicate it on each worker,
improved performance.
But now each worker was copying n times the data and each time raising
the counter, because each worker does the same operations on its
replicate to keep the object consistent.
The final solution was that the counter was placed in a separate
object, so that the compiler took the right decision of replicating
the buffer object, and that the workers only once did the copying of
the data, and that they only had to adjust the counter which the compiler
placed on the master.
Without knowing what the compiler decides about replicating, and not
knowing that objects do operation replication, this important optimization
could never be done.

The example above would have been difficult to detect without a proper
tracing tool.
The one provided at the moment is called Upshot.
Upshot gives graphical information about when and where events happen.
Important events are when processes block to do RPCs or broadcasts, and
when user defined events occur.
Without Upshot the knowledge about the behavior of the programs would
have been significantly less, and tuning would have taken
magnitudes of time longer.
Still you need help of a friend (read someone of the Orca team) to
point out what could cause certain behavior.

What still is missing is a good debugging tool.
At the moment it is possible to debug the generated C code, but this
is no fun, as half of the code, if not all, that passes is meaningless
to the Orca user.

There are no conditional compilation capabilities.
C's preprocessor was used instead to maintain a large amount of versions.
The advantage of this is that only one program has to be maintained
instead of all the different versions.

At the time of starting this research, it was not wise to use the
objects in a software engineering way.
Instead you should only use objects to communicate with other processes.
In the meantime there have been some changes, and it is not so
expensive anymore to just use objects locally.
Still it is questionable because the objects are not fully object
oriented.
It is also questionable if this is desired because sequential object
oriented programming tents to be slow.
Also is it not possible to do the following:
object\$foo().data where the operation foo() returns a record.

It was already mentioned that strategy calls had to be done by hand,
because the compiler sometimes takes the wrong decision.
It is questionable if this problem will ever be solved,  because
the programmer knows more about what the program has to do and how he
wants to use the objects than the compiler ever will.

Although message fragmentation is available it was still necessary to
do it by hand, because if large messages are sent (i.e. more than a
few MBs), inevitably the program crashes due to lack of memory.

Many of the problems with Orca that occured during this research have
been solved already.
In particular, the lack of message fragmentation and command-line
arguments have been solved, and the efficiency of nonshared (local)
objects has been improved.
Solving the problems has also disadvantages, that the program would have
looked different if implementation started now.
Fortunately it would have looked better.
Though it must be stated that the sorting of the B map, done by master
and workers, in the Data Parallel implementation would probably not
have been thought of if message fragmentation had already been
implemented.

As major problems have already been solved during this research,
the author does not doubt that most of the other problems mentioned
above will be solved too.


\section{Conclusion}

Parallelising the Turing Ring in Orca proved to be worthwhile.
Programming in Orca is not very difficult, although the lack of 
debugging tools was annoying. The Distributed Shared Memory model
works well in practice. It offers a clean interface to parallel
systems without diminishing the functionality.
Programming in parallel is still more complicated than serial, since
the
programmer must be aware of the fact that many things are happening at
the same time, and must take care of synchronisation.

\begin{sloppypar}
Although parallel
Orca programs do have communication overhead, speedups are achieved at medium
problem sizes, assuming a practical decomposition of the problem is
available. Finding such a decomposition is a specific difficulty
of parallel programming.
\end{sloppypar}

Programming in Orca can be done without much knowledge about the
underlying distributed system. However, tuning the performance of the program
cannot be done without this knowledge. 
For example, to minimise the number of messages send over the Ethernet
one must understand the run-time system in detail. The compiler cannot
optimise communication used for different operations. 

Some features found in other modern languages would be very welcome.
Currently there is no way of reading command line parameters or
environment variables. Conditional compilation (like C's {\tt
\#ifdef}/{\tt \#endif} would help to make larger projects easier to
handle and to enhance efficiency on different hardware. These changes
are currently under development.

The Turing Ring is well-suited for parallelisation. Because the
communication overhead grows as the number of processors grow, rather
than the problem size,
reasonable speedups can be achieved.
Although the speedup drops as the number of processors is increased,
the efficiency grows with the job size. The optimal number of
processors (i.e.\ the number of processors to be used to get a minimal
execution time) thus becomes larger as the jobs get larger. 
This kind of application is sensible to parallelise even for relatively
small problems.
A well-designed load balancing
strategy is necessary, though, to achieve a better hardware
utilisation. 

A suggestion for future work would be to optimise the synchronisation
at the end of each iteration. Currently, each worker waits until all
workers are ready. In reality, a worker only has to wait until its
immediate 
neighbours are finished. This can effectively put one worker process ahead
of others. 
This could accomplish some natural load balancing, and could also reduce network
congestion as some workers might be communicating earlier than
others.

Another optimisation could be useful if different hardware is used.
Currently it is assumed that shared variables are replicated, as is
the case on Amoeba, and thus read operations do not 
involve any network traffic. Although the read operations on the same
object which always follow each other are always implemented as a
single
operation, there are some read operations followed by a conditional
read operation. These could be optimised by implementing a combined
operation which is invoked if the condition yields true, and calling
the single operation otherwise. This might reduce network traffic on
systems which do not use replication.

One other possibility for optimisation lies in random number
generation. As discussed in Section \ref{sImpTur}, a 32-bit random
number is not necessary for the calculations made. Using bit-operators
(which were added to the language while this work was under way)
a more economic way of
handling random numbers could be implemented. This might improve
performance dramatically, because currently random number generation
still takes a large portion of the calculation time.



\bibliography{bib/cowichan}


\end{document}
