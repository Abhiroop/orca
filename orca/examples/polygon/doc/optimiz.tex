\sectionLabel{Parallel Optimizations}

In the sequential algorithm the Sort-and-Delete method was used to
reduce the number of comparisons, which improved the overlay process
drastically.
In order to apply the Sort-and-Delete method to the parallel algorithms
both the Sort-and-Delete method and the parallel algorithms have to be
adapted slightly, which is discussed below.

This section also addresses another optimization in
\refSection{Message Fragmentation}, which involves message fragmentation.


\subsectionLabel{Sort-and-Delete}

The Sort-and-Delete method has already been discussed in
\refSection{The Sort-and-Delete Method} for the sequential algorithm.

In both the remaining parallel methods it is possible to apply
the Sort-and-Delete method separately to each processor's A and B
sub-maps.
The only difference with the sequential case is that the two sub-maps
are not completely overlapping.
This means that it is no longer valid, to 
stop checking a polygon if all its surface has been overlapped.
In each parallel method this problem is solved differently.

\subsubsection{Data-Parallel Overlay}

To increase parallelism, the A sub-maps are sorted on each
processor.
The B map is sorted before replicating it on all processors,
because then this sorting can already start at the master processor
while the slave processors are sorting their A sub-maps.
Note that sorting introduces a sequential aspect, which reduces
speedup, but the gain is that the number of comparisons is reduced
dramatically.

On each processor each polygon of the A sub-map will be completely
overlapped by the polygons of the B map by definition, but the
converse is definitely not true.
The consequence of this is that a B polygon must be deleted from the
map not only if it has been completely overlapped, but also if
its lower right corner is smaller than the upper left corner of the A
polygon currently checked.


\subsubsection{Patchwised Overlay}

The A and B sub-maps are sorted locally at each processor.

In the case of Duplicate-and-Filter, the surface of each A and B border
polygon is adjusted to only be the size of what falls within the borders of
the underlying patch.
If during overlay a border polygon has a non-empty intersection with
an other border polygon, then only the resulting surface that lays
within the underlying patch is subtracted from both polygons.
Therefore the search per polygon in the other map can stop if its surface
has been completely overlapped, because now both maps have the same surface
to be intersected.


\subsectionLabel{Message Fragmentation}

At first Orca did not do message fragmentation, which imposed a
restriction on the amount of data for operation parameters. 
As a consequence the distribution of larger (sub)maps was impossible.
Therefore the Orca code was modified to send a large map as a sequence of
chunks.
In essence, the code was modified to do message fragmentation by hand,
but having implemented the message fragmentation the following
optimizations could be realized.

Instead of at the end sending all the resulting polygons to the
collector, which writes them to a file,
the sending is done when enough resulting polygons have been produced
to fill a result sub-map to the maximum message size.
The advantage of this approach is that it reduces contention by
overlapping communication and computation.

An other optimization can be done in the Data-Parallel case during the
sorting and sending of the B map.
Instead of first sorting the entire map and than cutting it up and
sending it to all the processors, the map is first cut up in pieces
and than successively each part is sorted and sent to the other
processors.
At the processor side the parts are merged. 
In this way the sorting is done by the process that sends and all the
processes that receive the parts.
