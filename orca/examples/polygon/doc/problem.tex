\sectionLabel{Problem Description}

Polygon Overlay is commonly done in spatial information systems
\cite{White:1978}.
Suppose two maps have been supplied,
$A$ and $B$.
Each map covers the same geographical area,
and is decomposed into a set of non-overlapping polygons
(\refByCapt{Example of Polygonal map Overlay}).
The aim is then to overlay the maps,
i.e.\ to generate a new map consisting of
the non-empty elements in the geometric intersection of $A$
and $B$.
This problem frequently arises in geographical information
systems,
in which the first map might represent soil type,
and the second,
vegetation.
Their overlay then shows how combinations of soil type and
vegetation
are distributed.

\begin{figure}[hbtp]
  \begin{centering}
    \hspace{0cm}
    \epsffile{polygonExample.eps}
    \refAndCapt{Example of Polygonal map Overlay}
  \end{centering}
\end{figure}

Although this problem has been chosen to serve as a 'large' benchmark,
the problem as it is now is far too complicated.
In order to simplify the implementation of the problem,
the following requirements are made:
\begin{enumerate}
  \item
    All polygons are non-empty rectangles.
    The polygons are chosen to be rectangles because
    they are easy to store
    (only the upper-left and the lower-right corner are needed)
    and they are convex.
    Calculating an overlay of non-convex polygons is much
    harder.
  \item
    The vertices of the polygons lie on an integer grid
    $[1{\ldots}N] {\times} [1{\ldots}M]$.
    This is done to simplify arithmetic,
    and generation of a polygon map.
    Note that if the polygons were not rectangular it might
    not be possible to put every vertex of every output
    polygon on the same integer grid.
  \item
    The input maps have identical extents.
  \item
    Each map is completely covered by its rectangular
    decomposition.
    Holes in a map are not allowed as they are hard to trace;
    overlapping polygons are not allowed either.
  \item
    Input and output maps are not sorted in any fashion.
  \item
    Both of the input maps and the output map can be held
    in memory. 
  \item
    The program should take two input files and deliver an
    output file, all of the same format.
  \item
    Each line in each file should contain four integers
    representing the upper-left and the lower-right corner of
    a single polygon.
    The first polygon in the file should represent the outer
    surrounding rectangle.
    The rest of the file should hold the decomposition of the
    first polygon.
\end{enumerate}

