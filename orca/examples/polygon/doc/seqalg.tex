\sectionLabel{Sequential Algorithms}

Several sequential programs for solving the polygon overlay problem
were developed in order to try out and refine different algorithms.
Only the simplest and fastest algorithms are discussed here.


\subsection{All Against All}

This method checks every polygon of the A map with all the polygons
of the B map.
To do so, each B polygon is checked against all polygons of the A map.
This method is the most straightforward algorithm to solve
the polygon overlay problem, and is easy to implement.
Unfortunately it one of the slowest.
Its time complexity is typically of order $O ( n_A \cdot n_B )$, where
$n_A$ and $n_B$ denote the number of polygons in the A, respectively B,
map.


\subsectionLabel{The Sort-and-Delete Method}

In the previous method, many checks are done of
polygons that do not overlap and therefore produce no resulting
polygons.
In order to reduce these unnecessary checks the Sort-and-Delete
method exploits the geographical-locality of the maps.

Geographical-locality means that a polygon from the A map
can only overlap with a polygon from the B map if they both
lie in the same area.
To make use of this fact, the polygons of the B map are
sorted on their upper-left-corner (first sort on the x- and
than on the y-coordinate).
If a polygon of the A map is being checked against the
polygons of the B map, then this process can be stopped when
the lower right corner of the A polygon is smaller than the
upper left corner of the B polygons.
In other words it is useless to search any further if an area
is entered where there is no chance to find an overlap.

Each time when there is a new polygon taken from the A map
there is an area in the beginning of the B map where there
are no overlaps possible.
To be able to skip polygons of the B map whilst checking, it is
necessary to sort the polygons of the A map too on their 
upper-left-corner.
Now there are two options to find out what to skip in the
B map.

In the first option it is necessary to sort the B map also
on the lower-right corner.
The map can be represented as a double linked list.
Now the search in the B map can start at the point where the
right-lower corners of the B polygons are larger than the
left-upper corner of the A polygon.
In other words the search can start there where it becomes
possible to find an overlap.
Note that the starting point, when checking a new A polygon,
is always the same as the previous point or further on in
the B map, because the A map is sorted.

The second option makes use of the definition that the map
may not contain any holes or overlapping polygons.
Each time a polygon of the A map is being checked against
the polygons of the B map and an overlap is found,
then the surface that is overlapped is subtracted from the
surface of the B polygon.
If all the surface of this B polygon is overlapped it can be
deleted from the B map.

It is clear that the second option is better, because
it requires less sorting, and often a polygon in the B map is
taken out sooner than the same polygon is skipped when applying
the first option.

There is one final optimization that can be done.
Instead of stopping when the lower-right-corner of the A
polygon is smaller than the upper-left corners of the B
polygons, it is possible to stop when the surface of the A
polygon as been given away.
Just like a record is kept of the surfaces of the B polygons,
a record can be kept of the surface of an A polygon.
Now each time an overlay is found the surface of the A
polygon is also reduced with the surface of the overlap.
If the surface of the A polygon has dropped to zero, then it
is useless to search any further and the next A polygon can
be taken from the map.
Stopping when the surface is zero always happens at the
same time or mostly sooner as in the case where is checked
whether the border has been past.
Also the checks in the improved version are only done when
an overlap has been found, instead of before each check.

The Sort-and-Delete method incorporates the above mentioned
optimizations as follows.
Both the A an B map are sorted on the upper-left-corner.
A record of the surface area is kept of the B polygons and
the current A polygon.
When an overlap is found, both the surfaces of the
concerning A polygon and B polygon are reduced with the
surface of the overlap.
If the surface of the B polygon is zero, then the polygon is
deleted from the B map.
If the surface from the A polygon is zero, then the search
proceeds with the next A polygon.

The Sort-and-Delete method has a typical time complexity of order
$O ( n_A \log n_B )$, where $n_A$ and $n_B$ denote the number of polygons
in the A, respectively B, map.
It must be stated tough, that there are exceptional input maps that result
in checking all against all polygons, see
\refByCapt{worst case}.

\begin{figure}[hbtp]
  \begin{centering}
    \hspace{0cm}
    \epsffile{worstCase.eps}
    \refAndCapt{worst case}
  \end{centering}
\end{figure}

