\documentclass{article}

\usepackage{a4}
\usepackage{times}
\usepackage{graphicx}
\usepackage{epsfig}
\usepackage{twocolumn}

\textwidth 6.5in
\oddsidemargin -0.2in

\newcommand{\rem}[1] {}

\title{Implementation of a Portable Runtime System Supporting Mixed
Task- and Data-Parallelism}

\author{Saniya Ben Hassen \\ Vrije Universiteit,
Amsterdam \\ saniya@cs.vu.nl \and Tim R\"uhl \\ Vrije Universiteit,
Amsterdam \\ tim@cs.vu.nl }

\date{}

\begin{document}

\maketitle

\begin{abstract}
  This paper describes the portable implementation of a programming
  model based on shared objects that integrates task- and
  data-parallelism. As a starting point, the implementation uses two
  runtime systems that are already available: the Orca RTS, for
  task-parallel programming, and Hawk, for data-parallel programming.
  To achieve our goal, we implemented the Orca RTS and Hawk on top of
  Panda, a virtual machine that was designed to facilitate porting
  parallel programming systems to new architectures. With the
  resulting system, an application program may use the task-parallel
  model of Orca and the data-parallel model of Hawk by calling the
  appropriate primitives in either the Orca RTS or in Hawk.

  \vspace{0.5cm}
  \textbf{keywords:} Data Parallel Programming.  Partitioned Shared
  Objects. Portable Runtime System. Portable Virtual Machine. Shared
  Objects. Task Parallel Programming.
\end{abstract}


\section{Introduction}
\label{sec:intro}

Supporting both task and data parallelism in one programming system is
useful, since many applications need both types of parallelism. We are
implementing a programming model based on shared objects that supports
both task- and data-parallelism~\cite{benhassen96b}. In this model,
there are two types of objects: \emph{shared objects}, that are either
stored on one processor or replicated on multiple processors and
\emph{partitioned shared objects} (partitioned objects for short), for
which the state is distributed over multiple processors. In a parallel
application, concurrent processes may communicate by invoking objects
of either type.

To implement this integrated model, we start with two available
runtime systems (RTSs), one that supports shared objects and one that
supports partitioned objects. Orca is a language based on
shared objects for programming task-parallel applications~\cite{bal92}
and its RTS provides an interface for managing processes and shared
objects. On the other hand, Hawk is a language independent RTS that
implements partitioned objects~\cite{benhassen96}. Our goal is
to integrate the Orca RTS and Hawk into a portable RTS that allows
both models to be used together at application level.

To achieve our goal, we use Panda~\cite{bhoedjang93}, a portable
virtual machine that was designed to facilitate porting parallel
programming systems to new architectures. Panda provides threads,
communication, and synchronization primitives to programming
systems. Only a small part of Panda needs to be rewritten to be used
on a new architecture.

In this paper, we describe the portable implementation of the mixed
task- and data-parallel object model on top of Panda. The Orca RTS
and Hawk were written independently on top of Panda. Each required a
different set of communication and synchronization tools for the
implementation of its object model. With the resulting system, an
application program may use both shared objects and partitioned
objects by calling the appropriate primitives in the Orca RTS or in
Hawk.

\section{Programming Model}

In our object-based parallel programming model, multiple threads of
control (processes or tasks) can communicate and synchronize through
operations on shared objects or partitioned objects.

Shared objects are used by task-parallel applications. A shared object
encapsulates a shared data structure and the operations on the data,
so it is essentially an instance of an Abstract Data Type (ADT). Its
state is either stored on one processor or replicated on multiple
processors. Processes may explicitely be created on different
processors and these processes may communicate by calling the
operations defined on shared objects.

In contrast, partitioned objects are used for data-parallel
applications. As in the shared object model, a partitioned
object is an instance of an ADT. The state of the ADT, however, is
restricted to be a multidimensional array that the programmer must
partition using directives. These partitions may then be distributed
arbitrarily over several processors (see Figure~\ref{fig:object}).

\begin{figure*}[htbp]
  \begin{center}
    \leavevmode
    \input{object.pstex_t}
    \caption{Example of a partitioned object. The state of the object 
      is partitioned column-wise and the columns are distributed over three
      processors.}
    \label{fig:object}
  \end{center}
\end{figure*}

After the state of an object has been partitioned and distributed, the
object may be invoked. An operation on an object can be either
sequential or parallel. A sequential operation is executed on one
machine and may update the entire state of the object sequentially. A
parallel operation is executed according to the ``owner-computes
rule,'' i.e., it is applied to all partitions owned by the processors in
parallel. The RTS guarantees that all partitions accessed and updated
are consistent. 

The integration of both types of objects leads to a flexible model
that is suitable for general-purpose parallel programming. It supports
a general process model with arbitrary interactions between processes
and mappings to physical machines.

\section{Implementation of the Programming Model}

Two runtime systems are now available for both shared objects and
partitioned objects. The Orca RTS provides all necessary primitives
for the dynamic creation of processes and for the creation and
invocation of single-copy or replicated objects. It requires from the
underlying operating system several communication and synchronization
primitives: Remote Procedure Call, Group
Communication~\cite{kaashoek92}, and threads.

Next to the Orca RTS, Hawk is a language-independent runtime system
that was written to support partitioned
objects~\cite{benhassen96}. Using the Hawk primitives, a program may
instantiate an object by specifying the number of its dimensions, its
size, and the operations that can be applied to it. Hawk also provides
routines to partition an object and distribute partitions over several
processors. Hawk requires from the underlying operating system a
different set of communication protocols than the Orca RTS does. It
uses collective communication on top of unreliable unicast and
multicast protocols in addition to Group Communication.

In order to integrate shared objects and partitioned objects
into one model, we have implemented the Orca RTS and Hawk on top of
Panda.  The portable implementation of both systems is briefly
described in the next section.

\section{Portable Implementation}

Panda provides threads, communication primitives, and synchronization
primitives. To facilitate porting Panda, a low-level \emph{system
interface} is defined (see Figure~\ref{fig:overview}). This system
interface includes threads, messages, and low-level communication
primitives. On top of this interface, a number of \emph{modules} may
be built. To support multiple communication modules, the system layer
provides message multiplexing and demultiplexing. Only the system
layer of Panda needs to be rewritten to be used on a new architecture;
high-level communication modules are architecture independent.

\begin{figure*}[htbp]
  \begin{center}
    \leavevmode
    \input{overview.pstex_t}
    \caption{System overview. }
    \label{fig:overview}
  \end{center}
\end{figure*}

We implemented the Orca RTS and Hawk on top of Panda by creating three
independent communication modules: Group Communication, Remote
Procedure Call, and Collective
Communication. Figure~\ref{fig:overview} gives an overview of the
resulting system. The Orca RTS uses Group Communication and Remote
Procedure Call. Independently from the Orca RTS, Hawk is implemented
on top of Group Communication and Collective Communication. Both the
Orca RTS and Hawk use the threads provided by Panda. At the highest
level, an application calls primitives of the Orca RTS to create and
use shared objects and primitives of Hawk to create and use
partitioned objects.

Since the entire system is implemented on top of Panda, it will be
available on all systems that Panda is ported to. Currently, Panda is
available on top of the Amoeba operating system on Ethernet and
Myrinet~\cite{Boden95-myrinet}. Furthermore, implementations exist for
Solaris~2.4, Thinking Machines CM-5, Parsytec GCEL, and Parsytec
PowerXplorer.

\section{Status and Future Work}
\label{sec:status}

The Group Communication and Remote Procedure Call modules have been
implemented on top of the Panda system layer and the portable Orca RTS
has been used for the implementation of many task-parallel
applications on various platforms. The collective communication module
for Hawk is currently being tested on the Amoeba system (through
Panda) with Ethernet. Once the testing phase is finished, the mixed
task- and data-parallel programming system will be available on all
platforms supported by Panda.

In the full version of this paper, we will describe in more detail the
shared and partitioned object models. We will also describe the
interfaces the Orca RTS and Hawk provide and some applications that
were implemented on top of them. Furthermore, we will fully describe
the properties of Panda that allowed the easy integration of the two
RTSs. Finally, we will provide performance measurements on a number of
different architectures at application level.

\bibliographystyle{unsrt}
\bibliography{/home/saniya/bibliography/biblio,/home/saniya/bibliography/myreferences,/home/tim/bib/head,/home/tim/bib/comm,/home/tim/bib/tail}

\end{document}
